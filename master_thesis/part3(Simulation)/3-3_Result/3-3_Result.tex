\documentclass[autodetect-engine,dvi=dvipdfmx,a4paper,ja=standard,oneside,openany,11pt,draft]{bxjsbook}
\usepackage{../../Preamble/mypackage}

\begin{document}
\section{数値計算結果}
\subsection{パターンの変化}

\begin{figure}[htbp]
  \centering
  \includegraphics[width=0.8\textwidth]{../../figure/part3/sim_result.png}
  \caption{数値計算によるパターンの結果。典型的なものを示す。横軸は応答の大きさ$\alpha$,縦軸は固着確率$P$である。}
  \label{fig:sim_result}
\end{figure}

図\ref{fig:sim_result}は数値計算の結果を示している。横軸は応答の大きさ$\alpha$,縦軸は固着確率$P$である。$\alpha$が増加してもはっきりとした違いは見られなかった。また,固着確率$P$の減少にともなって,$\alpha$の値によらずパターンの枝が太くなった。
\subsection{パターンの回転半径,フラクタル次元$D_f$の変化}

\begin{figure}[htbp]
  \centering
  \includegraphics[width=0.8\textwidth]{../../figure/part3/R_g_result.png}
  \caption{数値計算によるパターンの回転半径$R_g$の変化。横軸は応答の大きさ$\alpha$,縦軸は回転半径$R_g$である。}
  \label{fig:R_g_result}
\end{figure}

\begin{figure}
  \centering
  \includegraphics[width=0.6\textwidth]{../../figure/part3/fractal_dim_result.png}
  \caption{数値計算によるパターンのフラクタル次元$D_f$の変化。横軸は応答の大きさ$\alpha$,縦軸はフラクタル次元$D_f$である。}
  \label{fig:fractal_dim_result}
\end{figure}

図\ref{fig:R_g_result}は電場への応答$\alpha$と固着確率$P$を変化させたときの,回転半径の結果を示している。横軸は応答の大きさ$\alpha$,縦軸は回転半径$R_g$である。図\ref{fig:R_g_result}より,回転半径は$P>0.10$では$\alpha$の増加に対してほぼ単調に減少していた。逆に$P=0.10$ではほぼ一定値を取っていた。また,固着確率が減少するほど,回転半径の大きさも減少し,$\alpha$に対する減少割合は緩やかになった。特に$\alpha\lessapprox1.0$で顕著であった。また,固着確率が線形に減少(0.3ずつ)しているにもかかわらず,回転半径は線形に減少しておらず,$P=0.10$と$P=0.40$の間に大きなとびがあった。

図\ref{fig:fractal_dim_result}は電場への応答$\alpha$と固着確率$P$を変化させたときの,フラクタル次元の結果を示している。横軸は応答の大きさ$\alpha$,縦軸はフラクタル次元$D_f$である。フラクタル次元は,$P>0.10$では$\alpha$の増加に対してほぼ単調に増加しており,パターンが密になっていった。$P=0.10$ではほぼ一定値を取っていた。また,固着確率が減少するほど,フラクタル次元は2に近づき,$\alpha$に対する増加割合は,回転半径の場合と同様に緩やかになった。特に$\alpha\lessapprox1.0$で顕著であった。また,固着確率が線形に減少(0.3ずつ)しているにもかかわらず,フラクタル次元$D_f$は線形に増加しておらず,回転半径と同様に$P=0.10$と$P=0.40$の間に大きなとびがあった。
\ifdraft{
  \bibliographystyle{../../Preamble/Physics.bst}
  \bibliography{../../Preamble/reference.bib}
}{}
\end{document}