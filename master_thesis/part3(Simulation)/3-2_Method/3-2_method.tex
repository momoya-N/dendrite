\documentclass[autodetect-engine,dvi=dvipdfmx,a4paper,ja=standard,oneside,openany,11pt]{bxjsbook}
\usepackage{../../Preamble/mypackage}

\begin{document}
\chapter{数値計算}
\section{数値計算のモデル}
\subsection{RWの方向の決定方法}
\begin{figure}[htbp]
  \centering
  \includegraphics[width=0.5\textwidth]{../../figure/part3/RW_2dim.png}
  \caption{各方向へ移動する確率$p_i,q_i$と動かない確率$r_i$,電場$E_i$の模式図。}
  \label{fig:RW_2dim}
\end{figure}

金属樹のパターン形成における界面の影響や電場の影響を評価し,実験結果を再現するため,二次元DLA(空間次元$d=2$)をベースに,電場によるドリフトと界面活性剤による界面への固着の影響を取り込んだモデルを作成し,数値計算を行った。\ref{sec:RW}節と\ref{sec:Langevin}節の内容をもとに,RWとLangevin方程式の対応を考える。\ref{sec:RW}節より,外力(電場)について,
\begin{equation}
  \bar{F}_i=\mu q E_i=(p_i-q_i)\frac{a}{\Delta t}
  \label{eq:force}
\end{equation}が成り立つので,$p_i-q_i=\mu q E_i\Delta t/a$となる。また,拡散係数$D=a^2/(2d\Delta t)$より,$a^2=2dD\Delta t$となる。次に,式\eqref{eq:Langevin_overdamped_average}を離散化する。格子間隔$a$,$\Delta t$を用いると
\begin{equation}
  \begin{split}
    \bm{x} & =a\bm{X}, \qquad t=n\Delta t, \qquad (n\in\mathbb{N},\bm{X}\in\mathbb{Z}^d,a,\Delta t \in \mathbb{R}) \\
    \label{eq:discretization}
  \end{split}
\end{equation}
のように離散化できる。1ステップの時間発展を考えると,式\eqref{eq:discretization}で$n=1,X_i=\pm1$とすればよい。$\langle\cdot\rangle$を確率分布に関する平均とすれば,$\langle X_i\rangle=(+1)p_i+(-1)q_i=\mu q E_i\Delta t/a$,$\langle X_i^2\rangle=(+1)^2p_i+(-1)^2q_i=1/d$である。電場の平均操作について,1ステップの時間・空間スケールではほぼ一定であるとして動く前の電場を用いと,位置の平均は
\begin{equation}
  \begin{split}
    \langle\bm{x}\rangle & =a\langle\bm{X}\rangle                     \\
                         & =a\sum_{i=1}^{d}\bm{e}_i(p_i-q_i)          \\
                         & =\mu q \Delta t\sum_{i=1}^{d} \bm{e}_i E_i \\
                         & =\mu q \Delta t\bm{E}
  \end{split}
  \label{eq:discrete_average}
\end{equation}
となる。位置の分散は
\begin{equation}
  \begin{aligned}
    \langle(\bm{x}(t)-\langle\bm{x}(t)\rangle)^2\rangle
     & =a^2\langle(\bm{X}-\langle\bm{X}\rangle)^2\rangle                        \\
     & =a^2\{\langle\bm{X}^2\rangle-\langle\bm{X}\rangle^2\}                    \\
     & =a^2\sum_{i=1}^{d}\ab\{\frac{1}{d}-\frac{(\mu q \Delta t)^2}{a^2}E_i^2\} \\
     & =a^2\ab\{1-\frac{(\mu q \Delta t)^2}{a^2}\bm{E}^2\}                      \\
  \end{aligned}
  \label{eq:discrete_variance_3_2}
\end{equation}
式\eqref{eq:discrete_average}と式\eqref{eq:discrete_variance_3_2}は素朴にRWとLangevin方程式を対応させたものである。しかし,分散は電場の強さに依存しており,\ref{sec:Langevin}節で求めた,分散が外力に依存しない結果と異なる。このままだと,場所によって分散が異なる,つまり場所によって温度が異なる,物理的に不可解な状況になってしまう。

そこで,RWにおいて図\ref{fig:RW_2dim}のように\textbf{その場にとどまる確率 $r$}を導入し,電場の影響を打ち消すように調整した。物理的には電場と運動が拮抗し,動けない場合を想定している。まず,確率の保存則より,$r+\sum_{i=1}^{d}(p_i+q_i) =1$が成り立つ。そのため,第$i$成分に対して$p_i+q_i=(1-r)/d$となる。以降,$\bm{X}$空間での平均・分散を,$\langle X_i\rangle=2\alpha E_i$,$\langle\bm{X}^2\rangle-\langle\bm{X}\rangle^2=2dC$と置く。ここで,$\alpha=\mu q\Delta t/(2a)$であり,電場に対する応答の大きさを表す量である。式\eqref{eq:discrete_variance_3_2}より,分散は
\begin{equation}
  \begin{split}
    \langle(\bm{x}(t)-\langle\bm{x}(t)\rangle)^2\rangle & =a^2\ab\{(1-r)-\frac{(\mu q \Delta t)^2}{a^2}\bm{E}^2\} \\
                                                        & =a^2\ab\{(1-r)-4\alpha^2\bm{E}^2\}                      \\
                                                        & =a^2 2dC                                                \\
                                                        & =2dD\Delta t
  \end{split}
  \label{eq:discrete_variance}
\end{equation}
と修正される。式\eqref{eq:discrete_variance}より,$C=D\Delta t/a^2$,$r=1-2dC-4\alpha^2\bm{E}^2$となる。

ここで注意しなければならないのは,ここまでの議論が成り立つためには,少なくとも$r>0$でなければならない点である。$a^2=2dD\Delta t$のままならば,$C=1/(2d)$より,$r=-4\alpha^2\bm{E}^2<0$となってしまう。そのため,格子間隔$a$を固定すれば,時間間隔$\Delta t$を変化させることで,$a^2\neq2dD\Delta t$になり,$C\neq1/(2d)$より$r>0$を確保できる。

ここまでの議論より,RWにおいて電場の影響を打ち消す確率$r$を導入することで,Langevin方程式の分散を外力に依存しない形に変形できる。しかし,$r>0$を担保するため,時間間隔$\Delta t$を変化させなければならない。$\Delta t$を変えることは,パラメータである$\alpha=\mu q\Delta t/(2a),C=D\Delta t/a^2$を変えることに他ならない。今回の数値計算では分散$C$を固定して,$\alpha$を変化させて形状変化を調べた。

$C$を定めるために,まず各$i$成分の確率を求めた。各$i$成分の確率は
\begin{equation}
  \left\{
  \begin{aligned}
    p_i+q_i & =2C+4\alpha^2E_i^2 \\
    p_i-q_i & =2\alpha E_i
  \end{aligned}
  \right.
  \label{eq:prob}
\end{equation}
を満たす。式\eqref{eq:prob}の導出には,式\eqref{eq:discrete_variance_3_2}の分散に関して,総和を取る前の関係式を用いると$\langle X_i^2\rangle-\langle X_i\rangle^2=2C$となることより,
\begin{equation}
  \begin{split}
    \langle X_i^2\rangle-\langle X_i\rangle^2 & =\left\{\frac{1-r}{d}-\frac{(\mu q \Delta t)^2}{a^2}E_i^2\right\} \\
                                              & =(p_i+q_i)-4\alpha^2E_i^2                                         \\
                                              & =2C
  \end{split}
  \label{eq:prob_middle}
\end{equation}
となることを用いた。式\eqref{eq:prob_middle}より,各確率$p_i,q_i,r$は,
\begin{equation}
  \left\{
  \begin{aligned}
    p_i & =C+\alpha E_i+2\alpha^2 E_i^2 \\
    q_i & =C-\alpha E_i+2\alpha^2 E_i^2 \\
    r   & =1-2dC-4\alpha^2\bm{E}^2
  \end{aligned}
  \right.
  \label{eq:prob3}
\end{equation}
で与えられる。

$p_i,q_i$を$\alpha E_i$について平方完成すると,$\alpha E_i=\mp1/4$で最小値$C-1/8$を取る。また,$r$は$\alpha E_i=0$で最大値$1-2dC$を取る。$p_i,q_i$が0と1の間に入る条件は$d=2$の時,
\begin{align}
  0<C-\frac{1}{8}<1 & \Leftrightarrow  \frac{1}{8}<C<\frac{9}{8} \label{eq:condition_0} \\
  0<1-4C<1          & \Leftrightarrow  0<C<\frac{1}{4}
  \label{eq:condition}
\end{align}
式\eqref{eq:condition_0},\eqref{eq:condition}より,$C$の許される範囲は
\begin{equation}
  \frac{1}{8}<C<\frac{1}{4}
  \label{eq:condition2}
\end{equation}
となる。式\eqref{eq:condition2}より,上下端の中央の値である$C=3/16$を数値計算に用いる値とした。
\subsection{固着確率$P$の定義}
実験における,界面での界面活性剤によるイオンの析出阻害を再現するため,Brown運動してきた粒子を,パターンを構成する\textbf{クラスターに取り込む確率$P$}を導入した。本来のDLAでは,粒子がクラスターの隣に来た場合,そのまま新たなクラスターとして取り込むが,今回の数値計算では確率$P$でクラスターとして取り込んだ。取り込まれなかった場合は再びRWを行った。
\subsection{数値計算方法}
今回の数値計算では,パターンにより生じる電場と組み合わせてBrown運動する粒子の遷移確率を計算した。計算する系は,サイズ$L=\SI{512}{px}$四方の正方形である。系の中心座標 $\bm{x_c}=(x_c,y_c)=(L/2,L/2)=(256,256)$に粒子を一つ置き,粒子数 $N=15000$個として計算を行った。位置$\bm{x}$に対して電位の境界条件を
\begin{equation}
  V(\bm{x})=
  \begin{cases}
    1.0 & ||\bm{x}-\bm{x_c}||>L/2 \\
    0   & \bm{x}\in\text{パターン}
  \end{cases}
  \label{eq:boundary}
\end{equation}
と設定した。電場の計算は150粒子毎にSOR法(詳しくは付録\ref{sec:SOR}参照)を用いて行った。計算時の収束条件は前回のループとの誤差が$10^{-5}$未満とした。粒子の遷移確率は,式\eqref{eq:prob3}に従って計算した。

具体的な計算手順は次のとおりである。ただし,電場の計算は,金属樹の電解質溶液の電気的中性が期待されるため,電位$\phi$に対するLaplace方程式$\nabla^2\phi=0$を解いた。
\begin{samepage}
  \begin{enumerate}
    \item 現在のパターンの内,中心からの距離が最も遠い点を探索する。
    \item その点から30 px離れた円上にランダムに粒子を一つ置く。
    \item 式\eqref{eq:prob3}に従って粒子の移動方向を決定する。電位が1.0の領域(中心からの距離が$L/2$の円よりも外側)に来たらその粒子を棄却し,新たな粒子を2.に従って置く。
    \item 粒子が移動した結果,クラスター判定になっている位置に重なったら,ひとつ前の時間ステップの位置を,与えた固着確率$P$でクラスターとして取り込む。
    \item $150$粒子取り込むたびに電位$\phi$のLaplace方程式を解き,図\ref{fig:DLA_ex}のように電場を計算する。
    \item 手順2から5を15000粒子分繰り返す。
  \end{enumerate}
\end{samepage}

この計算を,$\alpha=0.0$から$3.0$まで(0.0から1.0まで0.1刻み,1.0から3.0までは0.2刻み),$P=0.1,0.4,0.7,1.0$の4つの固着確率について,各条件について21回行った。

\begin{figure}[htbp]
  \begin{minipage}{0.32\hsize}
    \subcaption{}
    \centering
    \includegraphics[width=0.8\textwidth]{../../figure/part3/DLA_alpha=0_P=1.png}
    \label{fig:DLA_alpha_0_P_1}
  \end{minipage}
  \begin{minipage}{0.32\hsize}
    \subcaption{}
    \centering
    \includegraphics[width=0.8\textwidth]{../../figure/part3/DLA_phi_alpha=0_P=1.png}
    \label{fig:DLA_phi_alpha_0_P_1}
  \end{minipage}
  \begin{minipage}{0.32\hsize}
    \subcaption{}
    \centering
    \includegraphics[width=0.8\textwidth]{../../figure/part3/DLA_E^2_alpha=0_P=1.png}
    \label{fig:DLA_E_alpha_0_P_1}
  \end{minipage}
  \caption{$C=3/16$,パラメータ$\alpha=0.0$固着確率$P=1.0$(通常のDLA)の例。\subref{fig:DLA_alpha_0_P_1}DLAパターンの形状。\subref{fig:DLA_phi_alpha_0_P_1}電位分布。\subref{fig:DLA_E_alpha_0_P_1}電場の強度分布。}
  \label{fig:DLA_ex}
\end{figure}

\subsection{解析方法}
各条件につき21個のデータについて最終形状の回転半径$R_g$とフラクタル次元を計測した。フラクタル次元の計測には密度相関関数法を用いた。まず密度相関関数$C(r)$を計算し,粒子間距離$2\leq r\leq R_g$以内の範囲でフィッティングを行い,\ref{sec:density_correlation}節に従いフラクタル次元を計測した。
$\alpha=2.6$では計算が収束しなかったため,解析には$\alpha\leq2.4$までのデータを用いた。

同じ条件で計算を行っても計算が収束する場合としない場合があったことより,原因として計算途中の粒子発生位置のランダム性が挙げられる。電場の影響が大きくなりすぎて電場に従った弾道的な運動になったために,たまたま中心付近を通るものしかクラスターに取り込まれなくなり,パターンの成長速度が極端に遅くなったためと考えられる。

\ifdraft{
  \bibliographystyle{../../Preamble/Physics.bst}
  \bibliography{../../Preamble/reference.bib}
}{}
\end{document}