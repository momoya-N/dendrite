\documentclass[autodetect-engine,dvi=dvipdfmx,a4paper,ja=standard,oneside,openany,11pt,draft]{bxjsbook}
\usepackage{../../Preamble/mypackage}

\begin{document}
\section{議論・考察}
以上の結果をまとめると,以下のようになる。
\begin{itemize}
  \item パターンの見た目は電場への応答の大きさ$\alpha$を大きくしてもさほど変化は見られないが,固着確率$P$を増加させるとパターンの枝が太くなることが確認された。
  \item 回転半径について,$\alpha$を大きくするとほぼ単調減少し,固着確率の減少により回転半径も減少することが分かった。また,変化の割合も緩やかになっていた。
  \item フラクタル次元については$\alpha$を大きくするとほぼ単調増加し,固着確率の減少によりフラクタル次元は2.00に近づくことが分かった。
\end{itemize}

まず,回転半径の変化とフラクタル次元の変化が逆の傾向を示している。これは回転半径がパターンの平均的な大きさを示していることより,粒子数が一定ならば,回転半径の減少でパターンが密になることは自明であり,その結果としてフラクタル次元が増加することも自明である。そのため,回転半径とフラクタル次元は逆の傾向を示す。

また,電場への応答$\alpha$の増加により,粒子がより電場の強い点,つまり,より細かい構造が多い部分に集まりやすくなるため,細かい構造が優先的に粒子で埋められることで密なパターンになったと思われる。クラスター形成の初期段階から細かい構造が埋められることで枝の広がりが抑制され,枝の遮蔽効果が弱くなり,内側まで粒子が入り込みやすくなった影響もあると思われる。

今回の数値計算からは,$\alpha$の影響以上に,固着確率$P$の方が影響が顕著であり,見た目の変化が大きかった点も含めて,パターン形成において固着確率の影響が大きいことがわかった。また,固着確率の低下は拡散律速凝集から界面での反応が律速段階になる凝集過程である反応律速凝集(Reaction Limited Aggregation:RLA)への遷移を意味するが,$P=0.10$とそれ以上の間にとびがあることから,この遷移は連続的でないことが示唆される。

また,実験結果と異なり,フラクタル次元の低下は再現できなかった。電場への応答よりも固着確率の方が影響が大きかったことより,固着確率の計算方法を工夫することが必要かと思われる。あるいは,粒子の数を増やして,より大きなパターンにすることで,太い枝を持つ,フラクタル次元の低い樹枝状パターンになる可能性もある。

RWを用いた数値計算は現実の原子($10^{23}$個)の大部分をひとまとめにして運動させている。今回の粒子数は,$1.5\times10^4$個なので,一個のブラウン粒子は約$10^{19}$個の原子の塊である。実験における界面の反応はイオン1粒子と,界面活性剤や高分子(長くとも$\SI{100}{nm}$程度,原子の大きさを$\SI{0.1}{nm}$程度とすれば,$10^3$個程度)の反応である。そのため,RW粒子は実験に比べて極端に大きい。そのため,界面での反応をうまく再現できなかったと思われる。

離散モデルのままではこの問題は付いて回る。計算の複雑さや計算量が増えてしまうが,界面形状を連続体として考えることで,より近い結果が得られるのではないかと思われる。

\ifdraft{
  \bibliographystyle{../../Preamble/Physics.bst}
  \bibliography{../../Preamble/reference.bib}
}{}
\end{document}