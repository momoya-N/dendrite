\documentclass[autodetect-engine,dvi=dvipdfmx,a4paper,ja=standard,oneside,openany,11pt,draft]{bxjsbook}
\usepackage{../../Preamble/mypackage}

\begin{document}
\section{議論・考察}
結果をまとめると,次のようになる。
\begin{itemize}
  \item パターンの見た目は電場への応答の大きさ$\alpha$を大きくしてもさほど変化は見られないが,固着確率$P$を増加させるとパターンの枝が太くなることが確認された。
  \item 回転半径について,$\alpha$を増加させるとほぼ単調減少し,固着確率の減少により回転半径も減少していた。また,変化の割合も緩やかになっていた。
  \item フラクタル次元については$\alpha$を増加させるとほぼ単調増加し,固着確率の減少によりフラクタル次元は2に近づいていた。
\end{itemize}

まず,回転半径の変化とフラクタル次元の変化が逆の傾向を示している。回転半径がパターンの平均的な大きさを示していることより,粒子数が一定ならば,回転半径の減少でパターンが密になることは自明であり,その結果としてフラクタル次元が増加することも自明である。そのため,回転半径とフラクタル次元は逆の傾向を示す。

また,電場への応答$\alpha$の増加により,粒子がより電場の強い点,つまり,より細かい構造が多い部分に集まりやすくなるため,細かい構造が優先的に粒子で埋められることで密なパターンになったと考えられる。クラスター形成の初期段階から細かい構造が埋められることで枝の広がりが抑制され,枝の遮蔽効果が弱くなり,内側まで粒子が入り込みやすくなった影響も考えられる。

今回の数値計算からは,$\alpha$の影響に比べて固着確率$P$の影響が顕著であり,見た目の変化がはっきりしていた点も含めて,パターン形成において固着確率の影響が大きいことが示唆された。また,固着確率の低下は拡散律速凝集から界面での反応が律速段階になる凝集過程である反応律速凝集(Reaction Limited Aggregation:RLA)への遷移を意味するが,$P=0.10$と$P>0.10$との間にとびがあることから,この遷移は連続的でないことが示唆された。

また,実験結果と異なり,フラクタル次元の低下は再現できなかった。電場への応答に比べて固着確率の影響が大きかったことより,固着確率の計算方法を工夫することが必要である。あるいは,粒子の数を増やして,より大きなパターンにすることで,太い枝を持つ,フラクタル次元の低い樹枝状パターンになる可能性もある。

今回の数値計算はRWを用いた離散的なモデルだが,添加物の拡散や粒子の拡散を考慮した連続体モデルに拡張することが考えられる。計算の複雑さや計算量が増えてしまうが,界面形状を連続体として考えることで,実験により近い結果が得られる可能性が高い。

\ifdraft{
  \bibliographystyle{../../Preamble/Physics.bst}
  \bibliography{../../Preamble/reference.bib}
}{}
\end{document}