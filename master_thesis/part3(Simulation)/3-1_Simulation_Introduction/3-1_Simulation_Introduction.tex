% \documentclass[autodetect-engine,dvi=dvipdfmx,a4paper,ja=standard,oneside,openany,11pt,draft]{bxjsbook}
\documentclass[autodetect-engine,dvi=dvipdfmx,a4paper,ja=standard,oneside,openany,11pt,draft]{bxjsarticle}
\usepackage{../../Preamble/mypackage}

\begin{document}

\part{数値計算}
\section{ランダムウォーク(RW)と拡散方程式}
\subsection{ランダムウォーク}
溶液中で粒子が拡散する際,その運動は\textbf{ランダムウォーク(Random Walk:RW)}でモデル化される。例えば空間次元$d=2$の場合,

\section{Langevin方程式}
\section{拡散律速凝集(Diffusion-limited aggregation, DLA)モデル}
\ifdraft{
  \bibliographystyle{../Preamble/Physics.bst}
  \bibliography{../Preamble/reference.bib}
}{}
\end{document}