\documentclass[autodetect-engine,dvi=dvipdfmx,a4paper,ja=standard,oneside,openany,11pt,draft]{bxjsbook}
\usepackage{../../Preamble/mypackage}

\begin{document}
\chapter{結論}
自然界に広くみられる樹枝状パターンは枝の先端の界面が不安定化することで分岐・成長していく。本研究の目的は,樹枝状パターンの形成過程における先端の分岐現象に着目し,先端の成長界面の微小な不安定性が枝の分岐角度や分岐の頻度にどのような影響を与え,その結果,成長界面に比べてよりスケールの大きな全体の樹枝状パターンをどのように変化させるか,実験的・理論的に明らかにすることである。

まず,第2章では電解析出によって生じる樹枝状パターンである金属樹を対称に,界面活性剤(Pluronic F-127)を加えて界面の析出メカニズムを変化させた。析出したパターンに対して,フラクタル次元だけでなく,従来あまり計測されていなかった分岐角度分布,枝長分布を計測した。その結果,界面活性剤の濃度が上昇するとフラクタル次元が小さくなり,細長い形状が増えることがわかった。分岐角度の分布は界面活性剤濃度の上昇によってあまり変化しなかった。一方で,枝長の分布は長い枝の出現確率が増加するように変化した。

第3章では添加物による界面成長の安定化を検証するため,levelerとされている界面活性剤(TWEEN20,Pluronic F-127)を加えて界面の二乗平均粗さと自己相関関数を計測した。その結果,TWEEN20に関しては濃度の上昇による界面の安定化が見られた。Pluronic F-127に関しても,界面の安定化が見られたが,その濃度依存性は濃度に対して線形に変化しておらず,$0.01 \mathrm{vol\%}$付近で最も安定化が見られた。

第4章では,RWする粒子にドリフトを加え,かつパターンへの固着確率を変化させ,その形状を解析した。その結果,電場への応答の大きさよりも,界面への固着確率のほうがパターン形成への影響が大きいことが示唆された。実験とは異なり,フラクタル次元が減少する結果は得られなかった。

実験結果より,界面活性剤によって分岐の頻度が減り,長い枝の割合が増加することが言える。また,線形に変化してはいないが,界面活性剤による界面の安定化が存在した。これより,界面の安定化が分岐の頻度に影響を与え,樹枝状パターンを枝の分岐が少なく長い枝を持ったパターンに変化させることが示唆された。Pluronic F-127に関する濃度依存性は線形では無かったが,フラクタル次元が$0.03 \mathrm{vol\%}$以降で低下していることと,$0.03 \mathrm{vol\%}$で界面粗さが増加していることより,この濃度で析出メカニズムが変化してる可能性があるが,詳細な実験や解析は今後の課題である。

また,数値計算結果より,界面での反応の方が樹枝状パターンへ与える影響が大きく,実験と同様に界面での反応がパターン形成に重要であることが示唆された。今回の数値計算では,実験結果は再現できなかったが,固着確率の決定方法を変えることで実験結果を再現できる可能性がある。どのように固着確率を決定するかは文献\cite{vicsek1984pattern}\cite{nittmann1986tip}などが提案されているが,フラクタル次元の低下を再現するためにどのようなモデルを構築するのかは不明瞭であり,今後の課題である。

\ifdraft{
  \bibliographystyle{../../Preamble/Physics.bst}
  \bibliography{../../Preamble/reference.bib}
}{}
\end{document}