% \documentclass[autodetect-engine,dvi=dvipdfmx,a4paper,ja=standard,oneside,openany,11pt,draft]{bxjsbook}
\documentclass[autodetect-engine,dvi=dvipdfmx,a4paper,ja=standard,oneside,openany,11pt,draft,textwidth=50zw]{bxjsbook}
\usepackage{../../Preamble/mypackage}


\begin{document}
\appendix
\chapter{Appendix}
\section{スケールフリー性を持つ関数がべき関数になることの証明}
\label{sec:scale_free}
スケールフリー関数はスケール$a$のみに依存する関数$g(a)$を用いて以下のように表される。
\begin{equation}
  \frac{f(ax)}{f(x)}=g(a) \quad (a>0)
\end{equation}
これを両辺$a$で微分し,変形していく。
\begin{equation}
  \begin{split}
    \frac{\d (ax)}{\d a}\frac{\d f(ax)}{\d (ax)}          & =\frac{\d g(a)}{\d a}f(x)                                                       \\
    \lim_{a\to 1}x\frac{\d f(ax)}{\d (ax)}                & =D_f f(x) \qquad\left(\left.\frac{\d f(ax)}{\d (ax)}\right|_{a\to1}:=D_f\right) \\
    \frac{1}{f(x)}\frac{\d f(x)}{\d x}                    & =\frac{D_f}{x}                                                                  \\
    \int_{x_0}^{x} \frac{1}{f(x)}\frac{\d f(x)}{\d x}\d x & =\int_{x_0}^{x} \frac{D_f}{x}\d x                                               \\
    \log f(x)-\log f(x_0)                                 & =D_f\log x-D_f\log x_0                                                          \\
    \log \frac{f(x)}{f(x_0)}                              & =D_f\log \frac{x}{x_0}                                                          \\
    f(x)                                                  & =f(x_0)\left(\frac{x}{x_0}\right)^{D_f}
  \end{split}
\end{equation}
以上よりスケールフリー性を持つ関数はべき関数になることが示された。
\section{フラクタル次元の求め方}
\section{MS不安定性の詳細な計算}
\section{ランダムウォーク(RW)からの拡散方程式の導出}
式\ref{eq:RW}
\begin{equation}
  c(\bm{x},t+\Delta t)=\sum_{i=1}^{d}\left[p_i c(\bm{x}-a\bm{e}_i,t)+q_i c(\bm{x}+a\bm{e}_i,t)\right]
  \label{eq:RW}
\end{equation}
を左辺は一次,右辺は二次までTaylor展開する。
\begin{equation}
  \begin{split}
    \mathrm{(左辺)} & =c(\bm{x},t)+\pdv{c(\bm{x},t)}{t}\Delta t+\mathcal{O}((\Delta t)^2)                                                                                                                 \\
    \mathrm{(右辺)} & =\sum_{i=1}^{d}\left[\underset{=1/d}{\uwave{(p_i+q_i)}}c(\bm{x},t)-(p_i-q_i)\pdv{c(\bm{x},t)}{x_i}a+\underset{=1/d}{\uwave{(p_i+q_i)}}\pdv[2]{c(\bm{x},t)}{x_i}\frac{a^2}{2}\right] \\
                  & =c(\bm{x},t)+\sum_{i=1}^{d}\left[-(p_i-q_i)\pdv{c(\bm{x},t)}{x_i}a+\pdv[2]{c(\bm{x},t)}{x_i}\frac{a^2}{2d}\right]+\mathcal{O}(a^3)
  \end{split}
  \label{eq:RW_taylor}
\end{equation}
以上の結果から,以下の式が成り立つ。
\begin{equation}
  \pdv{c(\bm{x},t)}{t}=-\sum_{i=1}^{d}(p_i-q_i)\frac{a}{\Delta t}\pdv{c(\bm{x},t)}{x_i}+\frac{a^2}{2d\Delta t}\sum_{i=1}^{d}\pdv[2]{c(\bm{x},t)}{x_i}
  \label{eq:RW_diffusion}
\end{equation}

\section{過減衰Langenvin方程式の分散の詳細な計算}
運動方程式
\begin{equation}
  \odv{\bm{x}}{t}=\mu q\bm{E}+\mu\bm{\xi}(t)
\end{equation}
より,時刻$t$での位置$\bm{x}(t)$は0から$t$までの積分を行い,以下のように表される。ただしランダム力は以下の関係式を満たす。
\begin{equation}
  \begin{split}
    \langle\bm{\xi}(t)\rangle                  & =0                                                   \\
    \langle\bm{\xi}(t)\cdot\bm{\xi}(t')\rangle & =2d\gamma k_B T\delta(t-t') \qquad (d:\mathrm{空間次元})
  \end{split}
  \label{eq:random_force}
\end{equation}
よって,時刻$t$での位置$\bm{x}(t)$は以下のように表される。
\begin{equation}
  \bm{x}(t)=\bm{x}(0)+\mu q\bm{E}t+\mu\int_0^t\bm{\xi}(t')\d t'
  \label{eq:Langevin_overdamped_integrated}
\end{equation}
式(\ref{eq:Langevin_overdamped_integrated})の両辺の二乗を取り,平均を取ると,以下のようになる。ただし,初期位置$\bm{x}(0)=0$とした。
\begin{equation}
  \begin{split}
    \langle\bm{x}(t)^2\rangle & =(\mu q \bm{E} t)^2+2\mu^2 q t\ab\langle\int_{0}^{t}\bm{E}(\bm{x})\cdot\bm{\xi}(t') \d t'\rangle+\mu^2\ab\langle\int_{0}^{t}\int_{0}^{t}\bm{\xi}(t')\cdot
    \bm{\xi}(t'') \d t' \d t''\rangle                                                                                                                                                                                                                            \\
                              & =(\mu q \bm{E} t)^2+2\mu^2 q t\ab\int_{0}^{t}\bm{E}(\bm{x})\cdot\underset{=0}{\uwave{\langle\bm{\xi}(t')\rangle}} \d t'+\mu^2\ab\int_{0}^{t}\int_{0}^{t}\underset{=2d\gamma k_B T\delta(t'-t'')}{\uwave{\langle\bm{\xi}(t')\cdot
    \bm{\xi}(t'')\rangle}} \d t' \d t''                                                                                                                                                                                                                          \\
                              & =(\mu q \bm{E} t)^2+2d\mu k_B T t                                                                                                                                                                                                \\
  \end{split}
  \label{eq:Langevin_overdamped_average}
\end{equation}

\section{DLAの数値計算の詳細}
\section{逐次加速緩和法(Successive Over Relaxation: SOR)の概要}

\ifdraft{
  \bibliographystyle{../../Preamble/Physics.bst}
  \bibliography{../../Preamble/reference.bib}
}{}
\end{document}