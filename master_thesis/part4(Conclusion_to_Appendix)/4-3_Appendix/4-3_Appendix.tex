\documentclass[autodetect-engine,dvi=dvipdfmx,a4paper,ja=standard,oneside,openany,11pt,draft]{bxjsbook}
\usepackage{../../Preamble/mypackage}


\begin{document}
\appendix
\chapter{スケールフリー性を持つ関数がべき関数になることの証明}
\label{sec:scale_free}
スケールフリー関数$f(x)$はスケール$a$のみに依存する関数$g(a)$を用いて以下のように表される。
\begin{equation}
  \frac{f(ax)}{f(x)}=g(a) \quad (a>0)
\end{equation}
これを両辺$a$で微分し,変形する。連鎖律より,
\begin{equation}
  \frac{\d (ax)}{\d a}\frac{\d f(ax)}{\d (ax)}          =\frac{\d g(a)}{\d a}f(x)
\end{equation}
となる。$a$を1に近づけると
\begin{equation}
  \lim_{a\to 1}x\frac{\d f(ax)}{\d (ax)}                 =D_f f(x)
\end{equation}
となる。ただし,
\begin{equation}
  D_f=\eval{\frac{\d g(a)}{\d a}}_{a=1}
\end{equation}
である。$x>0$かつ$f(x)>0$の範囲で,
\begin{equation}
  \frac{1}{f(x)}\frac{\d f(x)}{\d x}                    =\frac{D_f}{x}
\end{equation}
と変形できる。この式を$x_0$から$x$まで積分すると,
\begin{equation}
  \begin{split}
    \int_{x_0}^{x} \frac{1}{f(x)}\frac{\d f(x)}{\d x}\d x & =\int_{x_0}^{x} \frac{D_f}{x}\d x       \\
    \log f(x)-\log f(x_0)                                 & =D_f\log x-D_f\log x_0                  \\
    \log \frac{f(x)}{f(x_0)}                              & =D_f\log \frac{x}{x_0}                  \\
    f(x)                                                  & =f(x_0)\left(\frac{x}{x_0}\right)^{D_f}
  \end{split}
\end{equation}
以上よりスケールフリー性を持つ関数は冪関数になることが示された。
\chapter{Brown運動に関する理論計算}
\section{ランダムウォーク(RW)からの拡散方程式の導出}
\label{sec:RW_cal}
1ステップ後の粒子の濃度についての式
\begin{equation}
  c(\bm{x},t+\Delta t)=\sum_{i=1}^{d}\left[p_i c(\bm{x}-a\bm{e}_i,t)+q_i c(\bm{x}+a\bm{e}_i,t)\right]
  \label{eq:RW}
\end{equation}
を左辺は一次,右辺は二次までTaylor展開する。
\begin{equation}
  \begin{split}
    \mathrm{(左辺)} & =c(\bm{x},t)+\pdv{c(\bm{x},t)}{t}\Delta t+\mathcal{O}((\Delta t)^2)                                                                                                                 \\
    \mathrm{(右辺)} & =\sum_{i=1}^{d}\left[\underset{=1/d}{\uwave{(p_i+q_i)}}c(\bm{x},t)-(p_i-q_i)\pdv{c(\bm{x},t)}{x_i}a+\underset{=1/d}{\uwave{(p_i+q_i)}}\pdv[2]{c(\bm{x},t)}{x_i}\frac{a^2}{2}\right] \\
                  & =c(\bm{x},t)+\sum_{i=1}^{d}\left[-(p_i-q_i)\pdv{c(\bm{x},t)}{x_i}a+\pdv[2]{c(\bm{x},t)}{x_i}\frac{a^2}{2d}\right]+\mathcal{O}(a^3)
  \end{split}
  \label{eq:RW_taylor}
\end{equation}
以上の結果から,以下の式が成り立つ。
\begin{equation}
  \pdv{c(\bm{x},t)}{t}=-\sum_{i=1}^{d}(p_i-q_i)\frac{a}{\Delta t}\pdv{c(\bm{x},t)}{x_i}+\frac{a^2}{2d\Delta t}\sum_{i=1}^{d}\pdv[2]{c(\bm{x},t)}{x_i}
  \label{eq:RW_diffusion}
\end{equation}

\section{過減衰Langenvin方程式の分散の詳細な計算}
\label{sec:Langevin_cal}
運動方程式
\begin{equation}
  \odv{\bm{x}}{t}=\mu q\bm{E}+\mu\bm{\xi}(t)
  \label{eq:Langevin_overdamped}
\end{equation}
より,時刻$t$での位置$\bm{x}(t)$は0から$t$までの積分を行い,以下のように表される。ただしランダム力は以下の関係式を満たす。
\begin{equation}
  \begin{split}
    \langle\bm{\xi}(t)\rangle                  & =0                                                   \\
    \langle\bm{\xi}(t)\cdot\bm{\xi}(t')\rangle & =2d\gamma k_B T\delta(t-t') \qquad (d:\mathrm{空間次元})
  \end{split}
  \label{eq:random_force}
\end{equation}
よって,時刻$t$での位置$\bm{x}(t)$は以下のように表される。
\begin{equation}
  \bm{x}(t)=\bm{x}(0)+\mu q\bm{E}t+\mu\int_0^t\bm{\xi}(t')\d t'
  \label{eq:Langevin_overdamped_integrated}
\end{equation}
式\ref{eq:Langevin_overdamped_integrated}の両辺の二乗を取り,平均を取ると,以下のようになる。ただし,初期位置$\bm{x}(0)=0$とした。
\begin{equation}
  \begin{split}
    \langle\bm{x}(t)^2\rangle & =(\mu q \bm{E} t)^2+2\mu^2 q t\ab\langle\int_{0}^{t}\bm{E}(\bm{x})\cdot\bm{\xi}(t') \d t'\rangle+\mu^2\ab\langle\int_{0}^{t}\int_{0}^{t}\bm{\xi}(t')\cdot
    \bm{\xi}(t'') \d t' \d t''\rangle                                                                                                                                                                                                                            \\
                              & =(\mu q \bm{E} t)^2+2\mu^2 q t\ab\int_{0}^{t}\bm{E}(\bm{x})\cdot\underset{=0}{\uwave{\langle\bm{\xi}(t')\rangle}} \d t'+\mu^2\ab\int_{0}^{t}\int_{0}^{t}\underset{=2d\gamma k_B T\delta(t'-t'')}{\uwave{\langle\bm{\xi}(t')\cdot
    \bm{\xi}(t'')\rangle}} \d t' \d t''                                                                                                                                                                                                                          \\
                              & =(\mu q \bm{E} t)^2+2d\mu k_B T t                                                                                                                                                                                                \\
  \end{split}
  \label{eq:Langevin_overdamped_variance}
\end{equation}

\chapter{逐次加速緩和法(Successive Over Relaxation: SOR)の概要}
\label{sec:SOR}
\textbf{SOR法}はPoisson方程式の数値解法である\textbf{Gauss-Seidel法}を改良したものである。まず,Poisson方程式
\begin{equation}
  \nabla^2\phi(\bm{x})=-\rho(\bm{x})
  \label{eq:poisson}
\end{equation}
を数値的に解くために,離散化した方程式を考える。ここで,$\bm{x}=(x,y)$は二次元空間内の位置ベクトル,$\phi(\bm{x})$はポテンシャル,$\rho(\bm{x})$は適当な関数である。二次元空間を格子点で離散化する。格子点$(i,j)$におけるLaplasianは以下のように導かれる。格子間隔$\Delta x=\Delta y=h$とすると,
\begin{equation}
  \left\{
  \begin{aligned}
    \phi_{i\pm1,j} & =\phi_{i,j}\pm \pdv{\phi}{x}h+\pdv[2]{\phi}{x}\frac{h^2}{2}+\mathcal{O}(h^3) \\
    \phi_{i,j\pm1} & =\phi_{i,j}\pm \pdv{\phi}{y}h+\pdv[2]{\phi}{y}\frac{h^2}{2}+\mathcal{O}(h^3)
  \end{aligned}
  \right.
  \label{eq:discrete_laplasian}
\end{equation}
これより,Laplasianは以下のように離散化される。
\begin{equation}
  \begin{split}
    (\nabla^2\phi)_{i,j} & =\left\{\left(\pdv*[2]{}{x}+\pdv*[2]{}{y}\right)\phi\right\}_{i,j}                                            \\
                         & =\frac{1}{h^2}\left\{\phi_{i+1,j}+\phi_{i-1,j}+\phi_{i,j+1}+\phi_{i,j-1}-4\phi_{i,j}\right\}+\mathcal{O}(h^3) \\
                         & =-\rho_{i,j}
  \end{split}
  \label{eq:discrete_poisson}
\end{equation}
この式を以下のように更新して,前回ステップとの差が規定値以下になるまで計算していく。ここで$n$は計算ステップである。
\begin{equation}
  \phi_{i,j}^{n+1}=\frac{1}{4}\left(\phi_{i+1,j}^{n}+\phi_{i-1,j}^{n+1}+\phi_{i,j+1}^{n}+\phi_{i,j-1}^{n+1}+h^2\rho_{i,j}\right)
  \label{eq:Gauss-Seidel}
\end{equation}
式\ref{eq:Gauss-Seidel}のようなLaplace方程式の計算方法を\textbf{Gauss-Seidel法}という。ここに加速パラメータ$r$を導入し,以下のように更新式を変更する。
\begin{equation}
  \phi_{i,j}^{n+1}=(1-r)\phi_{i,j}^{n}+\frac{r}{4}\left(\phi_{i+1,j}^{n}+\phi_{i-1,j}^{n+1}+\phi_{i,j+1}^{n}+\phi_{i,j-1}^{n+1}+h^2\rho_{i,j}\right)
  \label{eq:SOR}
\end{equation}
式\ref{eq:SOR}の更新方法を\textbf{SOR法}という。SOR法はGauss-Seidel法よりも収束が速いことが知られている。実際,電場の計算にかかった実時間は,Gaiss-Seidel法に対しておおよそ1/10程度になっていた。

加速パラメータ$r$の値によって,効率が変化する。
\begin{itemize}
  \item $0<r<1$のとき,\textbf{under-relax}と呼ばれ,収束が速くなる。
  \item $r=1$のとき,Gauss-Seidel法と等価である。
  \item $1<r<2$のとき,\textbf{over-relax}と呼ばれ,収束が早くなる。
  \item $2<r$のとき,収束しない。
\end{itemize}
十分大きな系のサイズ$N$に対して,最適な$r=r_{\mathrm{optimaze}}$は,
\begin{equation}
  r_{\mathrm{optimaze}}=\frac{2}{1+\pi/N}
  \label{eq:optimaze_r}
\end{equation}
で与えられる。$r=r_{\mathrm{optimaze}}$の時,計算量はGauss-Seidel法が$\mathcal{O}(N^2)$なのに対して, SOR法は(厳密に)$2N$である。Gauss-Seidel法およびSOR法については文献\cite{hinch2020numerical}を参考にした。

\ifdraft{
  \bibliographystyle{../../Preamble/Physics.bst}
  \bibliography{../../Preamble/reference.bib}
}{}
\end{document}