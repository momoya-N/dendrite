\documentclass[autodetect-engine,dvi=dvipdfmx,a4paper,ja=standard,oneside,openany,11pt,draft]{bxjsbook}
\usepackage{../../Preamble/mypackage}

\begin{document}
\chapter*{謝辞}
本研究を進めるうえで指導教員である北畑裕之先生と伊藤弘明先生には,研究の進め方,実験・解析方法,学会での発表方法,理科系の文章の書き方などの実務的なことから,物理学をする上での考え方や思考法,研究に対する心構えなど,研究者としての姿勢や哲学を学ぶことができました。

自分の知識不足や誤解など,至らない点が多く,考えていることをうまく言葉に出来ないことが多々ありましたが,その都度助け舟を出していただいて根気強く丁寧に議論していただきました。また,思い込みや固定観念,不勉強などで,研究や議論が進まないときは,適切なアドバイスで研究の方向性を示していただくとともに,認知の歪みの矯正や新たな知識の獲得に努めるように指導していただきました。締め切り直前になっても何度も原稿の添削をしていただいたことなども含めて,この紙面の余白にも書ききれないほどの感謝の念に堪えません。

また,研究室の先輩,同期,後輩の方々にも,自分の興味とは異なる分野にもかかわらず積極的に質問やアドバイスをいただきました。ゼミだけでなく普段の何気ない雑談や冗談を通して,研究のアイデアや知識,プログラミング方法のみならず,研究への思いや物理学に対する考え方など,多くのことを学びました。また,1を聞いたら10を教えてくれるような同期や後輩にも恵まれ,大変有意義な研究室生活を送ることができました。

学部生時代からの友人,特に電磁気や流体力学,熱力学の輪講をしている他研究室の友人や,修士課程まで進んでいる高校時代の友人など,互いに励ましあえる仲間の存在は大変心強い存在でした。

最後に,家族には,生活費などの経済的な支援のみならず,大学院への進学,特に博士課程への進学の応援など,精神的な支えをいただきました。学部入学から現在まで,家族の支えがなければここまで研究を続けることはできなかったと思います。

研究室に所属してからの3年間,多くの方と関わり,助けていただき,学びを得ることができました。この場を借りて感謝の意を表すとともに,本修士論文の締めとさせていただきます。


\ifdraft{
  \bibliographystyle{../../Preamble/Physics.bst}
  \bibliography{../../Preamble/reference.bib}
}{}
\end{document}