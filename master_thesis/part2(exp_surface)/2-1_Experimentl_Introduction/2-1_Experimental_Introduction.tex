\documentclass[autodetect-engine,dvi=dvipdfmx,a4paper,ja=standard,oneside,openany,11pt,draft]{bxjsbook}
\usepackage{../../Preamble/mypackage}

\begin{document}
% \chapter{実験:界面成長}
\section{実験:界面成長}
\subsection{界面成長}
\textbf{界面成長}とは,異なる相の境界面が時間発展により成長していく現象である。たとえば,金属樹は液相内に含まれる$\ce{Zn^{2+}}$がすでに成長した$\ce{Zn}$の表面に付着して成長することで成長していく。
界面成長は,成長界面のランダムなゆらぎによりその形成過程が支配される。例えば,成長界面の高さ$h$の時間$t$に対する成長過程は,ランダムなゆらぎの項を含む次のような方程式で記述される\cite{kardar1987scaling}。
\begin{equation}
  \pdv{h}{t}=\nu\nabla^2h+\lambda(\nabla h)^2+\eta(\bm{x},t)
  \label{eq:KPZ}
\end{equation}
ここで,$\nu$は界面張力,$\lambda$は最低次の非線形項の係数,$\eta(\bm{x},t)$はランダムなゆらぎを表すホワイトノイズである。この方程式はKardar-Parisi-Zhang(KPZ)方程式と呼ばれる。KPZ方程式は,非線形項の係数$\lambda$が正のとき,成長界面はより粗い界面になり,負のときは滑らかな界面になることが知られている。

また,式\ref{eq:KPZ}により高次の微分項を加えたもの\cite{wolf1990growth}
\begin{equation}
  \pdv{h}{t}=\nu\nabla^2h-K\nabla^4h+\eta(\bm{x},t)+\mu\nabla^4h
  \label{eq:KPZ_higher}
\end{equation}
なども提案されている。Kは4次の勾配項の係数である。式\eqref{eq:KPZ_higher}の右辺一項目は界面への吸着・脱着プロセスによる界面緩和の効果を,二項目の線形項は界面拡散による緩和の効果を表しており,電解析出による界面形状を再現することが知られている。

\subsection{leveling効果}
工学分野や電気化学分野において,電界析出(メッキ)を行う際に重要なのが"いかに平坦な析出界面を形成するか"である。表面の粗さを抑えるということは成長過程での金属結晶の形成過程での格子欠陥の減少につながり高品質な製品を製造するうえで重要なためである。そのため,成長過程での界面粗さを抑制するため,電解質溶液に添加物を加えることで析出速度や,界面の微小な凹凸に対するイオンの析出位置を制御している。このように,電界析出における電解質溶液に添加剤を加えることで析出界面の粗さの成長を抑制できる効果のことを\textbf{leveling効果}と呼ぶ。leveling効果を起こす有機物質は様々なものが知られており,leveler(leveling剤)と呼ばれる。例えば亜鉛の電界析出においては第4級アンモニウム塩,ポリエチレングリコールなどの高分子,界面活性剤,イオン液体塩,有機酸\cite{sorour2017review}などが知られている。以下の図\ref{fig:leveling}はチオ尿素によるleveling効果の実験結果である\cite{schilardi1998evolution}。
\begin{figure}[H]
  \begin{minipage}
    {0.5\textwidth}
    \caption{}
    \centering
    \includegraphics[width=0.9\textwidth]{../../figure/part2(exp_surface)/el_dep_surface_no_TU_expfig.png}
    \label{fig:no_leveling_effect}
  \end{minipage}
  \begin{minipage}
    {0.5\textwidth}
    \caption{}
    \centering
    \includegraphics[width=0.9\textwidth]{../../figure/part2(exp_surface)/el_dep_surface_0.025M_TU_expfig.png}
    \label{fig:leveling_effect}
  \end{minipage}
  \caption{チオ尿素によるleveling効果に関する過去の報告(実験結果)\cite{schilardi1998evolution}。\subref{fig:no_leveling_effect}チオ尿素を加えない($\SI{0}{M}$)場合の実験結果。\subref{fig:leveling_effect}はチオ尿素を$\SI{0.025}{M}$加えた場合の実験結果。}
  \label{fig:leveling}
\end{figure}
図\ref{fig:leveling}からわかるように,\ref{fig:no_leveling_effect}では早い段階($<\SI{2400}{s}$)で析出界面が粗くなっているのに対し,\ref{fig:leveling_effect}では長時間($>\SI{6000}{s}$)経過しても析出界面が平坦になっていることがわかる。また,以下の結果は図\ref{fig:leveling}の成長界面の高さの平均値$\langle h\rangle$を右軸にその標準偏差$W_L$を左軸に表し,その時間変化をプロットしたものである。
\begin{figure}[H]
  \begin{minipage}
    {0.5\textwidth}
    \centering
    \includegraphics[width=0.9\textwidth]{../../figure/part2(exp_surface)/el_dep_surface_no_TU_resultfig.png}
    \caption{TU:$\SI{0}{M}$,解析結果}
    \label{fig:no_leveling_effect_result}
  \end{minipage}
  \begin{minipage}
    {0.5\textwidth}
    \centering
    \includegraphics[width=0.9\textwidth]{../../figure/part2(exp_surface)/el_dep_surface_0.025M_TU_resultfig.png}
    \caption{TU:$\SI{0.025}{M}$,解析結果}
    \label{fig:leveling_effect_result}
  \end{minipage}
  \caption{チオ尿素によるleveling効果に関する過去の報告(解析結果)\cite{schilardi1998evolution}\subref{fig:no_leveling_effect_result}チオ尿素を加えない($\SI{0}{M}$)場合の解析結果。\subref{fig:leveling_effect_result}チオ尿素を$\SI{0.025}{M}$加えた場合の解析結果。}
  \label{fig:leveling_result}
\end{figure}
図\ref{fig:leveling_result}からわかるように,TUを加えることで析出界面の高さの平均値$<h>$の成長速度が遅くなり,標準偏差$W_L$の時間発展も遅くなっていることがわかる。これは,TUによるleveling効果によって析出界面の粗さが抑制されていることを示している。\\
leveling効果のメカニズムは物質によって様々なものがある\cite{めっき添加剤の作用機構と表面形状制御}が,一般的には添加物の界面への吸着・拡散によって,界面成長速度の抑制や,金属イオンの析出阻害などが原因とされ\cite{oniciu1991some},界面付近の過飽和度の減少や界面張力の増加によってMS不安定性を抑制することで起こるとされている。
\ifdraft{
  \bibliographystyle{../../Preamble/Physics.bst}
  \bibliography{../../Preamble/reference.bib}
}{}
\end{document}