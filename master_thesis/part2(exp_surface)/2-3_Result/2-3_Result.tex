\documentclass[autodetect-engine,dvi=dvipdfmx,a4paper,ja=standard,oneside,openany,11pt]{bxjsbook}
\usepackage{../../Preamble/mypackage}

\begin{document}
\section{実験結果}
\subsection{TWEEN20を添加した際の実験・解析結果}
\begin{figure}[htbp]
  \begin{minipage}
    {0.5\textwidth}
    \subcaption{}
    \centering
    \includegraphics[width=0.9\textwidth]{../../figure/part2(exp_surface)/surface_exp_0mM_51s.png}
    \label{fig:surface_exp_0mM_51s}
  \end{minipage}
  \begin{minipage}
    {0.5\textwidth}
    \subcaption{}
    \centering
    \includegraphics[width=0.9\textwidth]{../../figure/part2(exp_surface)/surface_exp_0.005mM_547s.png}
    \label{fig:surface_exp_0.005mM_547s}
  \end{minipage}
  \caption{TWEEN20を添加した際の実験結果。\subref{fig:surface_exp_0mM_51s}$\SI{0}{mM}$(TWEEN20不添加)の実験結果,時刻$\SI{51}{s}$。\subref{fig:surface_exp_0.005mM_547s}TWEEN20濃度$\SI{0.005}{mM}$の実験結果,時刻$\SI{547}{s}$。}
  \label{fig:surface_exp}
\end{figure}
図\ref{fig:surface_exp}はTWEEN20を添加した際の円形極板での界面成長の実験結果である。見た目には似ているが,図\ref{fig:surface_exp}\subref{fig:surface_exp_0mM_51s}は時刻$\SI{51}{s}$の結果に対して,図\ref{fig:surface_exp}\subref{fig:surface_exp_0.005mM_547s}は時刻$\SI{547}{s}$の結果であり,同じような見た目になるまでおおよそ10倍程度の時間がかかった。

\begin{figure}[htbp]
  \begin{minipage}{0.43\textwidth}
    \subcaption{}
    \centering
    \includegraphics[width=0.9\linewidth]{../../figure/part2(exp_surface)/surface_roughness_TWEEN20.jpg}
    \label{fig:surface_roughness_TWEEN20}
  \end{minipage}
  \begin{minipage}{0.29\textwidth}
    \subcaption{}
    \centering
    \includegraphics[width=0.9\linewidth]{../../figure/part2(exp_surface)/roughness_correlation_function_0mM.jpg}
    \label{fig:roughness_correlation_function_0mM}
  \end{minipage}
  \begin{minipage}{0.26\textwidth}
    \subcaption{}
    \centering
    \includegraphics[width=0.9\linewidth]{../../figure/part2(exp_surface)/roughness_correlation_function_tween20.jpg}
    \label{fig:roughness_correlation_function_tween20}
  \end{minipage}
  \caption{界面粗さの時間発展と特徴波長。\subref{fig:surface_roughness_TWEEN20}二乗平均粗さ$W(t)$の時間発展。\subref{fig:roughness_correlation_function_0mM} $\SI{0}{mM}$(TWEEN20不添加)の時の成長高さの自己相関関数$C(\delta\theta,t)$の特徴波長。\subref{fig:roughness_correlation_function_tween20} TWEEN20濃度$\SI{0.005}{mM}$の時の成長高さの自己相関関数$C(\delta\theta,t)$の特徴波長。}
  \label{fig:surface_roughness}
\end{figure}

図\ref{fig:surface_roughness}はTWEEN20を$\SI{0.005}{mM}$加えた際の界面粗さの時間発展と界面の突出部の自己相関関数を表している。図\ref{fig:surface_roughness}\subref{fig:surface_roughness_TWEEN20}は時刻$t$における二乗平均粗さ$W(t)$の時間発展を示している。図中の横向き破線は,例として二乗平均粗さが$W(t)=\SI{40}{\mu m}$の線,縦向き破線はその時の時刻を表している。TWEEN20を添加していない時は$\SI{50}{s}$程度に対し,TWEEN20を$\SI{0.005}{mM}$加えた際は$\SI{500}{s}$弱で,TWEEN20を加えると界面が同程度に粗くなるまでおおよそ10倍程度の時間がかかった。図\ref{fig:surface_roughness}\subref{fig:roughness_correlation_function_0mM}\subref{fig:roughness_correlation_function_tween20}は図\ref{fig:surface_roughness}\subref{fig:surface_roughness_TWEEN20}の赤色の破線付近の時間の成長高さの自己相関関数$C(\delta\theta,t)$の特徴波長を示している。オレンジ色で示した点は小さい方から3つ分の極大値を表しており,数字は各極大値を取る$\delta \theta$の値である。
% ピークの位置は時間が経過してもほぼ同じ値を取っており(データを示す)
TWEEN20濃度が$\SI{0}{mM}$の時は$\delta\theta\sim\SI{0.6e-2}{rad}$程度であるのに対し,TWEEN20を$\SI{0.005}{mM}$加えた際は$\delta\theta\sim\SI{1.0e-2}{rad}$程度になっている。$r_0\sim\SI{40}{mm}$程度であり,$r_0$が$h(\theta,t)$に比べて十分大きいため,特徴波長は$\lambda\sim r_0\delta\theta$より$0.2\sim\SI{0.4}{mm}$程度になる。この結果より,TWEEN20濃度$\SI{0}{mM}$と$\SI{0.005}{mM}$とを比べると,特性波長$\lambda$がおおよそ2倍程度になっていた。

\subsection{Pluronic F-127を添加した際の実験・解析結果}
\begin{figure}[htbp]
  \begin{minipage}
    {0.32\textwidth}
    \subcaption{}
    \centering
    \includegraphics[width=0.9\textwidth]{../../figure/part2(exp_surface)/0_2695.98s_Pluronic_surface.png}
    \label{fig:0_2695.98 s_Pluronic_surface}
  \end{minipage}
  \begin{minipage}
    {0.32\textwidth}
    \subcaption{}
    \centering
    \includegraphics[width=0.9\textwidth]{../../figure/part2(exp_surface)/0.01_8147.37s_Pluronic_surface.png}
    \label{fig:0.01_8147.37 s_Pluronic_surface}
  \end{minipage}
  \begin{minipage}
    {0.32\textwidth}
    \subcaption{}
    \centering
    \includegraphics[width=0.9\textwidth]{../../figure/part2(exp_surface)/0.05_2645.03s_Pluronic_surface.png}
    \label{fig:0.05_2645.03 s_Pluronic_surface}
  \end{minipage}
  \caption{Pluronic F-127濃度を変化させた際の実験結果。\subref{fig:0_2695.98 s_Pluronic_surface} $\SI{0}{vol\%}$の実験結果, 時刻$\SI{2696}{s}$。\subref{fig:0.01_8147.37 s_Pluronic_surface} $\SI{0.01}{vol\%}$の実験結果,時刻$\SI{8147}{s}$。\subref{fig:0.05_2645.03 s_Pluronic_surface} $\SI{0.05}{vol\%}$の実験結果, 時刻$\SI{2645}{s}$。}
  \label{fig:Pluronic_surface}
\end{figure}

図\ref{fig:Pluronic_surface}はPluronic F-127を添加した際の円形極板での界面成長の実験結果である。TWEEN20と同様に,見た目には似ているが,図\ref{fig:Pluronic_surface}\subref{fig:0.01_8147.37 s_Pluronic_surface}は他の二つの結果に比べて,おおよそ3倍程度の時間がかかっていた。

\begin{figure}[htbp]
  \centering
  \includegraphics[width=0.7\textwidth]{../../figure/part2(exp_surface)/Pluronic_surface_roughness.png}
  \caption{二乗平均粗さ$W(t)$の時間発展。}
  \label{fig:surface_roughness_Pluronic}
\end{figure}

図\ref{fig:surface_roughness_Pluronic}はPluronic F-127を添加した際の界面粗さの時間発展を示している。TWEEN20の時とは異なり,界面活性剤濃度が$\SI{0.005}{vol\%},$\SI{0.01}{vol\%}の時に二乗平均粗さの時間発展が遅くなり,その後高濃度になると時間発展が速くなった。

\ifdraft{
  \bibliographystyle{../../Preamble/Physics.bst}
  \bibliography{../../Preamble/reference.bib}
}{}

\end{document}