\documentclass[autodetect-engine,dvi=dvipdfmx,a4paper,ja=standard,oneside,openany,11pt]{bxjsbook}
\usepackage{../../Preamble/mypackage}

\begin{document}
\chapter{実験:界面成長}
\section{実験・解析方法}
\subsection{界面成長の実験方法・実験系}
金属樹のパターン形成と析出界面の関係を調べるために,硫酸亜鉛七水和物\ce{ZnSO_4.7H_2O}(富士フイルム和光純薬)の$\SI{2}{M}$水溶液を作成し,そこにlevelerとされる非イオン性界面活性剤TWEEN20(東京化成工業)やPluronic F-127(フナコシ)を加えて電解析出を行った。実験系のセットアップについては図\ref{fig:surface_exp_system}。全体のセットアップは図\ref{fig:system_exp_whole}と同じである。

\begin{figure}[htbp]
  \begin{minipage}
    {0.55\textwidth}
    \subcaption{}
    \centering
    \includegraphics[width=0.9\textwidth]{../../figure/part2(exp_surface)/sys_side_surface.png}
    \label{fig:sys_side_surface}
  \end{minipage}
  \begin{minipage}{0.4\hsize}
    \subcaption{}
    \centering
    \includegraphics[width=\linewidth]{../../figure/part2(exp_surface)/surface_exp_system.png}
    \label{fig:sys_top_surface}
  \end{minipage}
  \caption{実験系の模式図。\subref{fig:sys_side_surface}横から見た模式図。\subref{fig:sys_top_surface}上から見た模式図。}
  \label{fig:surface_exp_system}
\end{figure}

実験手順は以下の通りである。
\begin{enumerate}
  \item 表面処理装置(Electro-Technic Products BD-20A)を用いてガラスシャーレ(直径$\SI{110}{mm}$)の表面を励起させ,溶液を広がりやすくした。
  \item  \ce{ZnSO_4.7H_2O} $\SI{2}{M}$水溶液に界面活性剤を加えた溶液を$\SI{12}{ml}$作成した。TWEEN20の濃度は$\SI{0.005}{mM}$,Pluronic F-127の濃度は$\SI{0}{\mathrm{vol}\%}, \SI{0.005}{\mathrm{vol}\%}, \SI{0.01}{\mathrm{vol}\%}, \SI{0.03}{\mathrm{vol}\%}, \SI{0.05}{\mathrm{vol}\%}$とした。
  \item ガラスシャーレに作成した溶液を入れた。(溶液厚:$\sim\SI{1.26}{mm}$)
  \item 陰極に半円形極板(直径約$\SI{63}{mm}$の半円部分と縦横約$\SI{63}{mm}\times \SI{25}{mm}$の四角形部分からなる \ce{Zn}板 。半円の部分で折り曲げて使用。),陽極に長方形極板(縦横約$\SI{15}{mm}\times$約$\SI{30}{mm}$の\ce{Zn}板。$\SI{15}{mm}$幅の方を溶液に設置。)を,二つの距離がおおよそ$\SI{30}{mm}$程度になるように設置した。半円形の極板は適当な大きさのシリコンゴムをガラスシャーレの中央付近に差し渡し,目玉クリップで固定した。長方形極板は目玉クリップでガラスシャーレにはさんで固定した。
  \item ライトビュアー(HAKUBA ライトビュアー 7000PRO)上にガラスシャーレを静置した。
  \item 電源装置(KENWOOD PA18-5B )とデジタルマルチメータ(ADCMT 7352A/E)を繋ぎ,電極間に$\SI{5}{V}$の電圧を印加した。室温は$\SI{22}{\degreeCelsius}$から$\SI{25}{\degreeCelsius}$程度だった。
  \item 電解析出していく過程を,USBカメラ(TWEEN20の実験時:オムロンセンテック STC-MBS43U3V 画素数 $\SI{720}{px} \times \SI{540}{px}$, 時間分解能 $\SI{527.1}{fps}$。Pluronic F-127の実験時:オムロンセンテック STC-MBS1242U3V 画素数 $\SI{4000}{px} \times \SI{3000}{px}$,時間分解能 $\SI{31.2}{fps}$。)にレンズ(Pixco PL2514 焦点距離$\SI{25}{mm}$)を取り付けて$\SI{1}{fps}$で撮影した。ただし,撮影範囲拡大のためマウント変換用のリングを複数個用いてカメラとの距離を伸ばしたため,スペックの性能が出ていない可能性がある。
\end{enumerate}

界面成長の実験系においては,バルク水溶液中での平坦な界面の成長を見るものが多い\cite{schilardi1998evolution}。しかし,今回の実験では金属樹の析出実験と条件を合わせるため,気液界面における界面成長を観察した。

また,極板を四角形のような角のある形状にすると,極板の角で電場が強くなりイオンが集中し,十分に界面成長する前に樹枝状結晶が発生してしまった。そのため,図\ref{fig:surface_exp_system}\subref{fig:sys_top_surface}の様に円形極板にすることで,角からの急速な樹枝状結晶の成長を抑制し,また,陽極の横幅を狭めることで,界面の成長範囲を制御した。
\subsection{解析方法}
\begin{figure}[htbp]
  \begin{minipage}
    {0.64\textwidth}
    \subcaption{}
    \centering
    \includegraphics[width=0.9\textwidth]{../../figure/part2(exp_surface)/polor_dif.png}
    \label{fig:surface_hight_def}
  \end{minipage}
  \begin{minipage}
    {0.32\textwidth}
    \subcaption{}
    \centering
    \includegraphics[width=0.9\textwidth]{../../figure/part2(exp_surface)/polor_changed.png}
    \label{fig:polor_changed}
  \end{minipage}
  \caption{成長高さの定義と極座標への変換後の例。緑色の線内の角度(青矢印)を座標変換し,解析に用いた。\subref{fig:surface_hight_def}円形極板における,極座標$r,\theta$の定義と界面高さ$h(\theta,t)$の定義。\subref{fig:polor_changed}極座標へ変換した界面の二値化画像。}
  \label{fig:surface_hight_def_polor_changed}
\end{figure}
まず,動画データをPythonのOpenCVを利用して,Otsuの二値化法を用いて二値化し,極板と析出した結晶の影(図\ref{fig:surface_hight_def_polor_changed}\subref{fig:surface_hight_def}の黒色の部分)を,面積でフィルタリングすることで抽出した。

実験において,界面はほぼ円の動径方向に成長していた。そのため,デカルト座標のままでは極板形状による歪みが生じてしまう。歪みの除去のため,まず動画の1フレーム目の画像の円形極板を円の一部と仮定して,図\ref{fig:surface_hight_def_polor_changed}\subref{fig:surface_hight_def}のような実験画像から,黒色領域と白色領域の境界をPythonのOpenCVを用いて抽出した。$y=-\sqrt{|r_0^2-(x-x_c)^2|}+y_c$でフィッティングを行い,極板の中心座標$(x_c,y_c)$,および極板半径$r_0$を取得した。それをもとに,極板の中心を原点とし,極板の中心からの距離$r=r(\theta,t)$を求めた。変換後は図\ref{fig:surface_hight_def_polor_changed}\subref{fig:polor_changed}のようになる。解析のしやすさのため,ビット反転して黒と白を入れ替えている。

初期フレームにおける$r(\theta,0)$の最小値$r_{0\mathrm{min}}$を初期界面と定義した(図\ref{fig:surface_hight_def_polor_changed}\subref{fig:surface_hight_def}の赤色破線)。成長高さ$h=h(\theta,t)$を初期界面からの高さ$h(\theta,t)=r(\theta,t)-r_{0\mathrm{min}}$と定義し解析を行った。時間の基準$t=0$は初期フレームとした。これは実験時において,電解析出開始時刻と撮影開始時刻にほぼ差がなく(高々数秒程度),全体の実験時間(1000秒以上)や反応の時間スケールに対して十分小さいためである。また,基準半径を$r_0$ではなく,$r_{0\mathrm{min}}$とした理由として,$r_0$はフィッティング結果のため,$h(\theta,0)<0$となる場合があったことと,$r_0$と$r_{0\mathrm{min}}$に大きな差がなかったことが挙げられる。

取得した各時刻の成長高さのデータ$h(\theta,t)$に対して,二乗平均粗さ$W(t)$と成長高さの自己相関関数$C(\delta\theta,t)$を計算した。

成長高さ$h(\theta,t)$に対する二乗平均粗さ$W(t)$は,$\langle\cdot\rangle$を$\cdot$に対する$\theta$による平均として,
\begin{equation}
  W(t) = \ab\langle[h(\theta,t)-\langle h(\theta,t)\rangle]^2\rangle^{1/2}
  \label{eq:W}
\end{equation}

と定義される。成長高さ$h(\theta,t)$に対する界面高さの自己相関関数$C(\delta\theta,t)$は,$\delta \theta$を角度のずれとして,
\begin{equation}
  C(\delta\theta,t) = \frac{\int \d \theta h(\theta,t)h(\theta+\delta\theta,t)}{\int \d \theta h(\theta,t)^2}
  \label{eq:C}
\end{equation}
と定義される。自己相関関数を計測することで,そのピークとなる$\delta \theta$を得ることができる。この$\delta \theta$によって,界面の突出部の波長,あるいは突出部の幅をある程度知ることができると思われる。

式\eqref{eq:W}で与えられる二乗平均粗さや,式\eqref{eq:C}で与えられる自己相関関数を用いると,界面の粗さの時間発展や突出部の幅を推定することができる。そのため,界面活性剤を加えたことによる界面の粗さの時間発展の変化を特徴づけることができ,金属樹の枝の分岐頻度や分岐角度,枝の太さの変化の原因を明らかにすることができると考えられる。
\ifdraft{
  \bibliographystyle{../../Preamble/Physics.bst}
  \bibliography{../../Preamble/reference.bib}
}{}
\end{document}