\documentclass[autodetect-engine,dvi=dvipdfmx,a4paper,ja=standard,oneside,openany,11pt,draft]{bxjsbook}
% \documentclass[autodetect-engine,dvi=dvipdfmx,a4paper,ja=standard,oneside,openany,11pt,draft]{bxjsarticle}
\usepackage{../../Preamble/mypackage}

\begin{document}
\section{実験・解析方法}
\subsection{界面成長の実験方法・実験系}
金属樹のパターン形成への界面への影響を調べるために,硫酸亜鉛七水和物\ce{ZnSO_4.7H_2O}(\textcolor{red}{富士フイルムワコー純薬})の$\SI{2}{mol/L}$水溶液を精製し,そこに界面活性剤(TWEEN20(\textcolor{red}{会社名}))を加えて,以下の条件で電界析出を行い,解析を行った。
\begin{table}[H]
  \begin{minipage}{0.5\hsize}
    \centering
    \includegraphics[width=\linewidth]{../../figure/part2(exp_surface)/surface_exp_system.png}
    \caption{実験系の模式図}
    \label{fig:surface_exp_system}
  \end{minipage}
  \begin{minipage}{0.5\hsize}
    \centering
    \caption{界面成長の実験条件}
    \begin{tabular}{c||c}
      \hline
      シャーレ直径     & $\phi = 11 \si{\cm}$ \\ \hline
      電圧         & $5 \si{\V}$          \\ \hline
      溶液量        & $12 \si{\mL}$        \\
      極板表面と電極の距離 & 約$\SI{3}{cm}$        \\
      \hline
    \end{tabular}
    \label{tab:surface_exp_condition}
  \end{minipage}
  \label{fig:surface_exp_system_condition}
\end{table}
実験手順は以下の通りである。
\begin{enumerate}
  \item シャーレに\textcolor{red}{プラズマをかけて表面を励起させ,溶液が広がりやすくする。}
  \item シャーレに硫酸亜鉛七水和物の水溶液を入れる。(溶液厚:$\sim\SI{1.26}{mm}$)
  \item 界面活性剤(TWEEN20,Pluronic F-127)を加える。界面活性剤の濃度は,TWEEN20は$\SI{0.005}{mM}$,Pluronic F-127は体積\%で,0\%,0.005\%,0.01\%,0.03\%,0.05\%の5つの濃度について実験を行った。
  \item シャーレの中心と側面にに電極を設置し,電極間に電圧を印可する。
  \item 電界析出していく過程を撮影し形態を観察する。
\end{enumerate}
界面成長の実験系においては,バルク中での平坦な界面の成長を見るものが多い(例えば\cite{schilardi1998evolution})。しかし,今回の実験では金属樹の析出実験と条件を合わせるため,気液界面における界面成長を観察した。また,極板を四角形のような角のある形状にすると,極板の角で電場が強くなりイオンが集中し,十分に界面成長する前に樹枝状結晶が発生してしまった。そのため,図\ref{fig:surface_exp_system}の用に円形極板にすることで,角からの急速な樹枝状結晶の成長を抑制し,また,陽極の位置を狭めることで,界面の成長範囲を制御した。
\subsection{解析方法}
\begin{figure}[H]
  \begin{minipage}
    {0.64\textwidth}
    \centering
    \includegraphics[width=0.9\textwidth]{../../figure/part2(exp_surface)/surface_hight_def.png}
    \caption{円形極板における,成長高さの定義と,曲座標$\theta$の定義}
    \label{fig:surface_hight_def}
  \end{minipage}
  \begin{minipage}
    {0.32\textwidth}
    \centering
    \includegraphics[width=0.9\textwidth]{../../figure/part2(exp_surface)/polor_changed.png}
    \caption{極座標へ変換後のデータ(界面成長中の変換画像)}
    \label{fig:polor_changed}
  \end{minipage}
  \caption{成長高さの定義と極座標への変換後の例}
  \label{fig:surface_hight_def_polor_changed}
\end{figure}
実験で撮影した実験データは円形極板での成長なので,そのままでは歪みが生じてしまう。歪みの除去のため,まず初期画像の円形極板を円の一部と仮定して黒色領域と白色領域の境界をPythonのOpenCVライブラリを用いて抽出し,$y=-\sqrt{|r_0^2-(x-x_c)^2|}+y_c$でフィッティングを行い,極板の中心座標$(x_c,y_c)$,および極板半径$r_0$を取得した。それをもとに,極板の中心を原点とし,極板の中心からの距離$r=r(\theta,t)$を求め,成長高さ$h=h(\theta,t)$を初期界面からの高さ$h(\theta,t)=r(\theta,t)-\min{\{r(\theta,0)\}}$と定義し解析を行った。ここで,基準半径を$r_0$ではなく,$r(\theta,0)$としたのは,$r_0$は初期界面の平均値として得られるため,$h(\theta,0)<0$となる場合があったことと,$r_0$と$\min{\{r(\theta,0)\}}$に大きな差がなかったためである。\\
取得した各時刻の成長高さのデータ$h(\theta,t)$に対して,以下で定義される二乗平均粗さ$W(t)$と成長高さの自己相関関数$C(\delta\theta,t)$を計算した。
\begin{defi}
  成長高さ$h(\theta,t)$に対する二乗平均粗さ$W(t)$は,$\langle\cdot\rangle$を$\cdot$に対する$\theta$による平均として,以下のように定義される。
  \begin{equation}
    W(t) = \ab\langle[h(\theta,t)-\langle h(\theta,t)\rangle]^2\rangle^{1/2}
  \end{equation}
  \label{def:W}
\end{defi}
\begin{defi}
  成長高さ$h(\theta,t)$に対する界面高さの自己相関関数$C(\delta\theta,t)$は,$\delta \theta$を角度のずれとして,以下のように定義される。
  \begin{equation}
    C(\delta\theta,t) = \frac{\int \d \theta h(\theta,t)h(\theta+\delta\theta,t)}{\int \d \theta h(\theta,t)^2}
  \end{equation}
  \label{def:C}
\end{defi}
自己相関関数を計測することで,そのピークが界面の突出部の波長,あるいは突出部の幅をある程度知ることができると思われる。\\
以上の二乗平均粗さや自己相関関数を用いて,界面の粗さの時間発展や突出部の幅を推定することで,leveler(界面活性剤)を加えたことによる金属樹の枝の分岐頻度や分岐角度,枝の太さの変化の原因を解明することを目的に実験を行った。
\ifdraft{
  \bibliographystyle{../../Preamble/Physics.bst}
  \bibliography{../../Preamble/reference.bib}
}{}
\end{document}