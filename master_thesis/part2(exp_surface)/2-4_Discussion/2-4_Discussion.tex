\documentclass[autodetect-engine,dvi=dvipdfmx,a4paper,ja=standard,oneside,openany,11pt,draft]{bxjsbook}
\usepackage{../../Preamble/mypackage}


\begin{document}
\section{議論・考察}
TWEEN20を添加した場合には,二乗平均粗さ$W(t)$の時間発展が10倍程度遅くなった(図\ref{fig:surface_roughness}\subref{fig:surface_roughness_TWEEN20})。また,成長高さの自己相関関数$C(\delta\theta,t)$の特徴波長は,TWEEN20を添加した場合には$\SI{0}{mM}$の時よりも2倍程度大きくなっていた(図\ref{fig:surface_roughness}\subref{fig:roughness_correlation_function})。

この結果よりTWEEN20を添加することでleveling効果が働き,界面の粗さが減少したことが示唆される。また,突出部の波長については$\lambda\sim\SI{0.2}{mm}$から$\SI{0.4}{mm}$で濃度により差が出たが,ミクロな段階での波長がそのまま金属樹の枝の太さに一致するかは非自明であり,濃度の影響がどの程度影響するのかの調査は今後の課題である。

金属樹の析出実験でも用いたPluronic F-127でも同様の実験を行い二乗平均粗さを計測したが,こちらはTWEEN20とは異なり,中間の濃度($\SI{0.005}{vol\%},\SI{0.01}{vol\%}$)で時間発展が遅くなり,高濃度になると時間発展が早くなる傾向が見られた。\Red{原因として,今回調べた濃度範囲においてleveling効果が最大になる濃度($\SI{0.005}{vol\%}$から$\SI{0.01}{vol\%}$の周辺)の存在が考えられる。}実際,銅にlevelerとしてチオ尿素を添加した際に,中間的な濃度で最もleveling効果が大きくなることが報告されている\cite{schilardi2000stable}。

\ifdraft{
  \bibliographystyle{../../Preamble/Physics.bst}
  \bibliography{../../Preamble/reference.bib}
}{}
\end{document}