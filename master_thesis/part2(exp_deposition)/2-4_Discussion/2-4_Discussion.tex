\documentclass[autodetect-engine,dvi=dvipdfmx,a4paper,ja=standard,oneside,openany,11pt,draft]{bxjsbook}
\usepackage{../../Preamble/mypackage}

\begin{document}
\section{議論・考察}
以上の結果をまとめると,界面活性剤の濃度を増やして金属樹を生成すると,濃度が上がるほど傾向としてフラクタル次元が減少する(図\ref{fig:fractal_dim})する結果が得られた。これは金属樹の枝分かれが少なくなり(図\ref{fig:branch}\subref{fig:branch_number}),パターンが疎になることで一次元形状(線)の割合が増えるためだと考えられる。

また,図\ref{fig:angle}, \ref{fig:branch_length}, \ref{fig:branch_length_edited}より,分岐角度は界面活性剤の濃度に依存しない一方で,枝の長さは濃度が上がるほど,長い枝が出現しやすくなることが示唆された。界面活性剤の添加によって,\ref{sec:tip_splitting}節でも言及したような,析出界面における界面張力の増加や,イオン流束の減少による過飽和度の減少などにより,MS不安定性における波長や析出速度が変化したためと考えられる。

このことは図\ref{fig:branch_length_exp}において,界面活性剤濃度が上昇すると確率密度関数$f(x)=ax^{-b}$の指数$b$が傾向として減少することからも裏付けられる。指数$b$が減少するということは,図\ref{fig:pow_b_func}のように確率密度関数の値が全体的に上昇し,ある範囲の値(図\ref{fig:pow_b_func}では$2<x<3$の網掛け部の面積)が出る確率が上昇することを示している。したがって,界面活性剤の添加によって,長い枝がより出現しやすくなる傾向があるといえる。

\begin{figure}[htbp]
  \centering
  \includegraphics[width=0.8\textwidth]{../../figure/part2(exp_deposition)/pow_b_func.png}
  \caption{確率密度関数(冪関数)と指数の大小の関係}
  \label{fig:pow_b_func}
\end{figure}

\ifdraft{
  \bibliographystyle{../../Preamble/Physics.bst}
  \bibliography{../../Preamble/reference.bib}
}{}
\end{document}