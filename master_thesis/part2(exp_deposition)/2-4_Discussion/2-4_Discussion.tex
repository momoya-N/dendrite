\documentclass[autodetect-engine,dvi=dvipdfmx,a4paper,ja=standard,oneside,openany,11pt]{bxjsbook}
\usepackage{../../Preamble/mypackage}

\begin{document}
\section{議論・考察}
界面活性剤の濃度を増加させて金属樹を生成すると,濃度の増加と共に,傾向としてフラクタル次元が減少する結果が得られた(図\ref{fig:fractal_dim})。この結果は金属樹の枝分かれが少なくなり(図\ref{fig:branch}\subref{fig:branch_number}),パターンが疎になることで一次元形状(直線の枝)の割合が増えたためだと考えられる。

また,図\ref{fig:angle}, \ref{fig:branch_length}, \ref{fig:branch_length_edited}より,分岐角度は界面活性剤の濃度に依存しない一方で,枝の長さは濃度が上がるほど,長い枝が出現しやすくなることが示唆された。界面活性剤の添加によって,\ref{sec:tip_splitting}節でも言及したような,析出界面における界面張力の増加や,イオン流束の減少による過飽和度の減少などにより,MS不安定性における波長や析出速度が変化したためと考えられる。

このことは図\ref{fig:branch_length_exp}において,界面活性剤濃度が上昇すると確率密度関数$f(x)=ax^{-b}$の指数$b$が傾向として減少することからも裏付けられる。指数$b$の減少は,図\ref{fig:pow_b_func}のように確率密度関数の値が全体的に上昇し,ある範囲の値(図\ref{fig:pow_b_func}では$2<x<3$の網掛け部の面積)が出る確率が上昇することを示している。したがって,界面活性剤の添加によって,長い枝がより出現しやすくなる傾向がある。ただし,図\ref{fig:branch_length_exp}\subref{fig:exp_b_branch_len}において,$\SI{0.005}{\mathrm{vol}\%}$は傾向から外れた低い値を取っていたが,これは図\ref{fig:branch_length},\ref{fig:branch_length_edited}で言及した,低濃度側で見られる極端に長い枝の影響である。また,図\ref{fig:branch_length_exp}\subref{fig:exp_b_branch_edited_len}において$\SI{0.005}{\mathrm{vol}\%}$の値が普通の枝だけの時に比べて相対的に大きい値となっているのは,$\SI{0.03}{\mathrm{vol}\%}$以上の濃度での幹の割合が増えたためである。

\begin{figure}[htbp]
  \centering
  \includegraphics[width=0.75\textwidth]{../../figure/part2(exp_deposition)/pow_b_func.png}
  \caption{確率密度関数(冪関数)と指数の大小の関係。}
  \label{fig:pow_b_func}
\end{figure}

\ifdraft{
  \bibliographystyle{../../Preamble/Physics.bst}
  \bibliography{../../Preamble/reference.bib}
}{}
\end{document}