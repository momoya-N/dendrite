% \documentclass[autodetect-engine,dvi=dvipdfmx,a4paper,ja=standard,oneside,openany,11pt,draft]{bxjsbook}
\documentclass[autodetect-engine,dvi=dvipdfmx,a4paper,ja=standard,oneside,openany,11pt,draft]{bxjsarticle}
\usepackage{../../Preamble/mypackage}

\begin{document}
\subsection{実験結果}
\subsubsection{フラクタル次元}

\subsubsection{枝の本数・太さ}
\begin{figure}[H]
  \begin{minipage}
    {0.5\textwidth}
    \centering
    \includegraphics[width=0.9\textwidth]{../../figure/part2(exp_deposition)/branch_num.png}
    \subcaption{枝の本数}
    \label{fig:branch_number}
  \end{minipage}
  \begin{minipage}
    {0.5\textwidth}
    \centering
    \includegraphics[width=0.9\textwidth]{../../figure/part2(exp_deposition)/branch_thickness_mean.png}
    \subcaption{枝の太さ}
    \label{fig:branch_thickness}
  \end{minipage}
  \caption{中心からの距離$r$の円と交わる枝の本数・太さ}
\end{figure}
\ref{fig:branch_number}は中心からの距離$r$の円と交わる枝の本数,\ref{fig:branch_thickness}は枝の太さを示している。図中黒破線(20本の線)をおおよそ境にして,
\begin{enumerate}
  \item 界面活性剤濃度0.03\%以上:枝の本数は20本を超えない。
  \item 界面活性剤濃度0.03\%未満:枝の本数は20本を超え,外側に連れて増加していく。
\end{enumerate}
という特徴が見られる。また,\ref{fig:branch_thickness}は枝の太さを示している。図中黒破線は太さ約$\SI{0.06}{cm}$を表しており,この線をおおよそ境にして,
\begin{enumerate}
  \item 界面活性剤濃度0.03\%以上:枝の太さは緩やかに大きくなり,ばらつきも大きくなる。
  \item 界面活性剤濃度0.03\%未満:枝の太さはおおむね$\SI{0.06}{cm}$以下であり,太さのばらつきも小さい。
\end{enumerate}
\subsubsection{枝の長さ,分岐角度}

\ifdraft{
  \bibliographystyle{../../Preamble/Physics.bst}
  \bibliography{../../Preamble/reference.bib}
}{}
\end{document}