\documentclass[autodetect-engine,dvi=dvipdfmx,a4paper,ja=standard,oneside,openany,11pt]{bxjsbook}
\usepackage{../../Preamble/mypackage}

\begin{document}
\section{実験結果}
\subsection{金属樹の概観・フラクタル次元}

\begin{figure}[htbp]
  \begin{minipage}
    {0.32\textwidth}
    \subcaption{}
    \centering
    \includegraphics[width=0.9\textwidth]{../../figure/part2(exp_deposition)/3620s_0.00sur.png}
    \label{fig:non_surfactant}
  \end{minipage}
  \begin{minipage}
    {0.32\textwidth}
    \subcaption{}
    \centering
    \includegraphics[width=0.9\textwidth]{../../figure/part2(exp_deposition)/3092s_0.005sur.png}
    \label{fig:0.005_surfactant}
  \end{minipage}
  \begin{minipage}
    {0.32\textwidth}
    \subcaption{}
    \centering
    \includegraphics[width=0.9\textwidth]{../../figure/part2(exp_deposition)/3548s_0.01sur.png}
    \label{fig:0.01_surfactant}
  \end{minipage}
  \\
  \begin{minipage}
    {0.32\textwidth}
    \subcaption{}
    \centering
    \includegraphics[width=0.9\textwidth]{../../figure/part2(exp_deposition)/3126s_0.03sur.png}
    \label{fig:0.03_surfactant}
  \end{minipage}
  \begin{minipage}
    {0.32\textwidth}
    \subcaption{}
    \centering
    \includegraphics[width=0.9\textwidth]{../../figure/part2(exp_deposition)/814s_0.05sur.png}
    \label{fig:0.05_surfactant}
  \end{minipage}
  \caption{界面活性剤濃度による金属樹の形態変化。典型的なものを抜粋。\subref{fig:non_surfactant}界面活性剤濃度$\SI{0}{\mathrm{vol}\%}$,実験開始後$\SI{3620}{s}$。\subref{fig:0.005_surfactant}界面活性剤濃度$\SI{0.005}{\mathrm{vol}\%}$,実験開始後$\SI{3092}{s}$。\subref{fig:0.01_surfactant}界面活性剤濃度$\SI{0.01}{\mathrm{vol}\%}$,実験開始後$\SI{3548}{s}$。\subref{fig:0.03_surfactant}界面活性剤濃度$\SI{0.03}{\mathrm{vol}\%}$,実験開始後$\SI{3126}{s}$。\subref{fig:0.05_surfactant}界面活性剤濃度$\SI{0.05}{\mathrm{vol}\%}$,実験開始後$\SI{814}{s}$。}
  \label{fig:surfactant}
\end{figure}

図\ref{fig:surfactant}は界面活性剤を加えた際の,各濃度ごとにおける金属樹の最終形状の結果である。界面活性剤濃度が高くなるほど定性的には枝分かれが減少していく。実験で得られたパターンのフラクタル次元をボックスカウンティング法を用いて計測したものが図\ref{fig:fractal_dim}である。図\ref{fig:fractal_dim}より,界面活性剤濃度が高くなるほどフラクタル次元が小さくなることが明らかになった。また,界面活性剤濃度が0.03\%以上の条件ではフラクタル次元は減少していた。また,界面活性剤濃度が高くなるほどばらつきが増加していた。
\begin{figure}[htbp]
  \centering
  \includegraphics[width=0.5\textwidth]{../../figure/part2(exp_deposition)/fractal_dim_result.png}
  \caption{界面活性剤濃度によるフラクタル次元の変化。}
  \label{fig:fractal_dim}
\end{figure}

\subsection{枝の本数・太さ}

\begin{figure}[htbp]
  \begin{minipage}
    {0.45\textwidth}
    \subcaption{}
    \centering
    \includegraphics[width=0.9\textwidth]{../../figure/part2(exp_deposition)/branch_num.png}
    \label{fig:branch_number}
  \end{minipage}
  \begin{minipage}
    {0.45\textwidth}
    \subcaption{}
    \centering
    \includegraphics[width=0.9\textwidth]{../../figure/part2(exp_deposition)/branch_thickness_mean.png}
    \label{fig:branch_thickness}
  \end{minipage}
  \caption{中心からの距離$r$の円と交わる枝の本数・太さの結果。\subref{fig:branch_number}枝の本数。\subref{fig:branch_thickness}枝の太さ。}
  \label{fig:branch}
\end{figure}

図\ref{fig:branch}\subref{fig:branch_number}は中心からの距離$r$の円と交わる枝の本数,図\ref{fig:branch}\subref{fig:branch_thickness}は枝の太さを示している。枝の太さは半径$r$の円と重なった何本かの枝の太さの平均値を表している。

まず,枝の本数について,界面活性剤濃度$\SI{0.03}{\mathrm{vol}\%}$未満では図中赤矢印のように$r$の増加と共に本数も増加していく。一方,$\SI{0.03}{\mathrm{vol}\%}$以上では枝の本数は$r$によらず同じ程度である。図\ref{fig:branch}\subref{fig:branch_number}の黒点破線は例として枝の本数20本を表している。低濃度側ではこの線を超えることが多い一方で,高濃度側ではこの線をあまり超えないことからも,高濃度側において枝の本数があまり増加しないことが理解できる。また,図\ref{fig:branch}\subref{fig:branch_thickness}は枝の太さを示しており,界面活性剤濃度$\SI{0.03}{\mathrm{vol}\%}$未満では,図中赤矢印のように$r$によらず枝の太さは比較的小さく一定で,ばらつきも小さい。一方,界面活性剤濃度$\SI{0.03}{\mathrm{vol}\%}$以上では,図中青矢印のように枝の太さが$r$とともに緩やかに大きくなり,ばらつきも大きくなる。図中黒点破線は例として太さ約$\SI{0.6}{mm}$を表している。高濃度側では$\SI{0.6}{mm}$を超えることが多い一方で,低濃度側では$\SI{0.6}{mm}$を超えないことが多いことからも,濃度による枝の太さの違いが見て取れる。$r$の増加と時間経過はおおむね比例することより,界面活性剤濃度が$\SI{0.03}{\mathrm{vol}\%}$未満では金属樹は頻繁に細い枝が発生,枝分かれをする。一方で$\SI{0.03}{\mathrm{vol}\%}$以上では枝分かれが抑制され,太い枝として成長していた。

\subsection{分岐角度}

\begin{figure}[htbp]
  \begin{minipage}
    {0.5\textwidth}
    \subcaption{}
    \centering
    \includegraphics[width=0.9\textwidth]{../../figure/part2(exp_deposition)/angle_in_result.png}
    \label{fig:angle_in}
  \end{minipage}
  \begin{minipage}
    {0.45\textwidth}
    \subcaption{}
    \centering
    \includegraphics[width=0.9\textwidth]{../../figure/part2(exp_deposition)/angle_out_result.png}
    \label{fig:angle_out}
  \end{minipage}
  \caption{分岐角度の界面活性剤濃度による分布の変化。ビン幅は$0.1\pi$。\subref{fig:angle_in}枝の内側の分岐角度$\theta_{\mathrm{in}}$の分布。\subref{fig:angle_out}枝の外側の分岐角度$\theta_{\mathrm{out}}$の分布。}
  \label{fig:angle}
\end{figure}

\begin{table}
  \centering
  \caption{界面活性剤濃度による$\theta_{\mathrm{in}}$の平均値。}
  \begin{tabular}{|c|c|}
    \hline
    濃度 vol\% & $\theta_{\mathrm{in}}$の平均値 rad \\
    \hline\hline
    0        & $0.392\pi$                     \\ \hline
    0.005    & $0.393\pi$                     \\ \hline
    0.01     & $0.389\pi$                     \\ \hline
    0.03     & $0.406\pi$                     \\ \hline
    0.05     & $0.422\pi$                     \\
    \hline
  \end{tabular}
  \label{tab:angle_average}
\end{table}

図\ref{fig:angle}は枝の分岐角度の分布である。図\ref{fig:angle}\subref{fig:angle_in}は枝の内側の分岐角度の分布で,表\ref{tab:angle_average}は各濃度における$\theta_{\mathrm{in}}$の平均値である。分岐角度の分布はフラクタル次元ほど明らかな変化は見られず,界面活性剤の濃度によらずほぼ同じような分布となっている。内側の分岐角度の平均値はおおよそ$2\pi/5$ radになっていた。

図\ref{fig:angle}\subref{fig:angle_out}は枝の外側の分岐角度の分布である。外側の分岐角度も濃度によらずほぼ同じような分布となっていた。$\theta_{\mathrm{out}}$は8割程度が$0.8\pi \ \si{rad}$ 以上であり,横方向($0.7\pi \ \si{rad}$未満)に曲がる分岐は少ないことが示唆された。
\subsection{枝の長さ}

\begin{figure}[htbp]
  \centering
  \includegraphics[width=0.6\textwidth]{../../figure/part2(exp_deposition)/branch_hist_relativ.png}
  \caption{普通の枝の長さ分布(相対度数表示)。}
  \label{fig:branch_length}
\end{figure}

図\ref{fig:branch_length}は普通の枝の長さ分布をカメラの解像度($\SI{1}{px}\approx \SI{0.16}{mm}$)未満を除いて相対度数で示しており,界面活性剤濃度が高くなるほど長い枝の割合が増加していた(図中赤丸)。また,極端に長い枝が低濃度($\SI{0}{\mathrm{vol}\%}$と$\SI{0.005}{\mathrm{vol}\%}$)でわずかな割合見られた(図中青丸)。この極端に長い枝は,低濃度のパターン内に数本だけ存在する長い枝が,相対度数を取ったために,割合として強調されたためである。

\begin{figure}[htbp]
  \centering
  \includegraphics[width=0.6\textwidth]{../../figure/part2(exp_deposition)/branch_edited_hist_relativ.png}
  \caption{幹と,幹に含まれない普通の枝の長さ分布(相対度数表示)。}
  \label{fig:branch_length_edited}
\end{figure}

図\ref{fig:branch_length_edited}は幹と,幹に含まれない普通の枝の長さの分布を解像度($\SI{1}{px}\approx \SI{0.16}{mm}$)未満を除いて相対度数で示している。図\ref{fig:branch_length}に比べて,高濃度における長い枝の割合が増加していた。また,極端に長いものも同様に低濃度側で見られた。この極端に長い枝も図\ref{fig:branch_length}と同様に,1,2本だけ存在する長い枝が相対度数で強調されたためである。

\begin{figure}[htbp]
  \begin{minipage}
    {0.51\textwidth}
    \subcaption{}
    \centering
    \includegraphics[width=0.9\textwidth]{../../figure/part2(exp_deposition)/exp_b_branch_len.png}
    \label{fig:exp_b_branch_len}
  \end{minipage}
  \begin{minipage}
    {0.49\textwidth}
    \subcaption{}
    \centering
    \includegraphics[width=0.9\textwidth]{../../figure/part2(exp_deposition)/exp_b_branch_edited_len.png}
    \label{fig:exp_b_branch_edited_len}
  \end{minipage}
  \caption{枝の長さの冪分布の指数$b$の測定結果。\subref{fig:exp_b_branch_len}普通の枝の分布の指数$b$。\subref{fig:exp_b_branch_edited_len}幹及び幹に含まれない普通の枝を合わせた分布の指数$b$。}
  \label{fig:branch_length_exp}
\end{figure}

\begin{table}[htbp]
  \begin{minipage}{0.45\textwidth}
    \centering
    \caption{界面活性剤濃度による,普通の枝の分布の指数$b$の平均値。}
    \begin{tabular}{|c|c|}
      \hline
      濃度 vol\% & 指数$b$の平均値 \\ \hline\hline
      0        & $4.29$    \\ \hline
      0.005    & $3.88$    \\ \hline
      0.01     & $4.41$    \\ \hline
      0.03     & $3.89$    \\ \hline
      0.05     & $3.76$    \\
      \hline
    \end{tabular}
    \label{tab:brnch_len_exp}
  \end{minipage}
  \hfill
  \begin{minipage}{0.45\textwidth}
    \centering
    \caption{界面活性剤濃度による,幹と,幹に含まれない枝の分布の指数$b$の平均値。}
    \begin{tabular}{|c|c|}
      \hline
      濃度 vol\% & 指数$b$の平均値 \\ \hline\hline
      0        & $3.33$    \\ \hline
      0.005    & $3.29$    \\ \hline
      0.01     & $3.46$    \\ \hline
      0.03     & $3.24$    \\ \hline
      0.05     & $2.92$    \\
      \hline
    \end{tabular}
    \label{tab:branch_len_exp_edited}
  \end{minipage}
\end{table}

図\ref{fig:branch_length_exp}は界面活性剤濃度ごとの枝の長さ分布の指数$b$を示している。図\ref{fig:branch_length_exp}\subref{fig:exp_b_branch_len}は普通の枝の分布の指数$b$であり,図\ref{fig:branch_length_exp}\subref{fig:exp_b_branch_edited_len}は幹と,幹に含まれない普通の枝を合わせた分布の指数$b$である。表\ref{tab:brnch_len_exp}は普通の枝の分布の指数$b$の平均値であり,表\ref{tab:branch_len_exp_edited}は幹と,幹に含まれない普通の枝を合わせた分布の指数$b$の平均値である。
図\ref{fig:branch_length_exp}\subref{fig:exp_b_branch_len}より,普通の枝の指数$b$の値は界面活性剤濃度に対して,
\begin{itemize}
  \item $\SI{0}{\mathrm{vol}\%}$, $\SI{0.01}{\mathrm{vol}\%}$では相対的に大きな値($b>4.00$)
  \item $\SI{0.03}{\mathrm{vol}\%}$以上では相対的に小さな値($b<4.00$)
  \item $\SI{0.005}{\mathrm{vol}\%}$は傾向から外れている
\end{itemize}
となっていた。
図\ref{fig:branch_length_exp}\subref{fig:exp_b_branch_edited_len}より,幹と,幹に含まれない普通の枝を合わせた分布の指数$b$の値は界面活性剤濃度に対して,
\begin{itemize}
  \item $\SI{0.03}{\mathrm{vol}\%}$以下では相対的に大きな値($b>3.20$)
  \item $\SI{0.05}{\mathrm{vol}\%}$では相対的に小さな値($b<3.20$)
\end{itemize}
となった。
\ifdraft{
  \bibliographystyle{../../Preamble/Physics.bst}
  \bibliography{../../Preamble/reference.bib}
}{}
\end{document}