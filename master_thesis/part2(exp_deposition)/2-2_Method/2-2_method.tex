\documentclass[autodetect-engine,dvi=dvipdfmx,a4paper,ja=standard,oneside,openany,11pt,draft]{bxjsbook}
\usepackage{../../Preamble/mypackage}

\begin{document}
\section{実験・解析方法}
\subsection{金属樹析出の実験方法・実験系}
本研究では亜鉛の金属樹を対象に実験・解析を行った。硫酸亜鉛七水和物\ce{ZnSO_4.7H_2O}(富士フイルム和光純薬)の$\SI{2}{mol/L}$水溶液を作成し,そこに非イオン性界面活性剤であるPluronic F-127(フナコシ)を加え,濃度を変えて電解析出を行った。実験器具については表\ref{tab:exp_condition},実験系のセットアップについては図\ref{fig:system_exp},\ref{fig:system_exp_whole}を参照\textcolor{red}{(解析には直接関わらないがライトボード,ジャッキ情報も必要か?)}。
\begin{table}[htbp]
  \centering
  \caption{実験系に用いた器具の情報}
  \begin{tabular}{|c||c|}
    \hline
    シャーレ直径     & $\SI{110}{mm}$                                                        \\ \hline
    シャーレ素材     & ガラス                                                                   \\ \hline
    陰極         & 直径$\SI{0.5}{mm}$ \ce{Zn}線                                             \\ \hline
    陽極         & \ce{Zn}板 約$\SI{20}{mm}\times$約$\SI{100}{mm}$                          \\ \hline
    表面処理装置     & Electro-Technic Products BD-20A                                       \\ \hline
    電源装置       & \textcolor{red}{KENWOOD (生産終了している)} PA18-5B                           \\ \hline
    デジタルマルチメータ & ADCMT 7352A/E                                                         \\
    \hline
    USBカメラ     & オムロンセンテック STC-MBS43U3V 画素数 $720 \times 540 \si{pix}, \SI{527.1}{fps}$ \\ \hline
    レンズ        & HOZAN L-600-12, 焦点距離$\SI{12}{mm}$                                     \\
    \hline
  \end{tabular}
  \label{tab:exp_condition}
\end{table}
\begin{figure}[htbp]
  \begin{minipage}
    {0.65\textwidth}
    \subcaption{}
    \centering
    \includegraphics[width=0.9\textwidth]{../../figure/part2(exp_deposition)/system_side.png}
    \label{fig:system_side}
  \end{minipage}
  \begin{minipage}
    {0.32\textwidth}
    \subcaption{}
    \centering
    \includegraphics[width=0.9\textwidth]{../../figure/part2(exp_deposition)/system_top.png}
    \label{fig:system_top}
  \end{minipage}
  \caption{実験系の模式図。\subref{fig:system_side}横から見た模式図。\subref{fig:system_top}上から見た模式図。}
  \label{fig:system_exp}
\end{figure}
\begin{figure}[htbp]
  \centering
  \includegraphics[width=0.5\textwidth]{../../figure/part2(exp_deposition)/exp_sys_whole.png}
  \caption{周辺機器も含めた実験系の外観}
  \label{fig:system_exp_whole}
\end{figure}
実験手順は以下の通りである。

\begin{enumerate}
  \item 表\ref{tab:exp_condition}の表面処理装置を用いてガラスシャーレの表面を励起させ,溶液を広がりやすくした。
  \item \ce{ZnSO_4.7H_2O} $\SI{2}{mM}$水溶液にPluronic F-127を加えた溶液を$\SI{15}{ml}$作成した。この時Pluronic F-127の濃度は$\SI{0}{\mathrm{vol}\%}, \SI{0.005}{\mathrm{vol}\%}, \SI{0.01}{\mathrm{vol}\%}, \SI{0.03}{\mathrm{vol}\%}, \SI{0.05}{\mathrm{vol}\%}$になるように調整した。
  \item $\SI{15}{ml}$の溶液のうち,$\SI{3}{ml}$用いてガラスシャーレの表面を洗浄した。
  \item 陽極として表\ref{tab:exp_condition}の\ce{Zn}板を3枚用意し,ガラスシャーレの側面に沿うように湾曲させて設置した。この時,極板が元の形状に戻るためにしなるため,3枚の極板を重ねて設置し,ガラスシャーレ側面に互いに押し付け合うように極板を設置,固定した。
  \item ガラスシャーレに残りの溶液$\SI{12}{ml}$を入れた。(溶液厚:$\sim\SI{1.26}{mm}$)
  \item 陰極の先が液面に触れるまでジャッキでライトプレートを上昇させ,ガラスシャーレの中心に電極を設置した。
  \item 表\ref{tab:exp_condition}の電源装置を用いて,電極間に$\SI{5}{V}$の電圧を印加した。室温は$\SI{21}{\degreeCelsius}$から$\SI{23}{\degreeCelsius}$程度だった。
  \item 電界析出していく過程を表\ref{tab:exp_condition}のUSBカメラで$\SI{10}{fps}$で撮影した。
\end{enumerate}

\subsection{解析方法}
\subsubsection{データの2値化処理}
取得したデータは以下の手順で二値化した。
\begin{enumerate}
\item 動画データをImageJ(Fiji)のMake Substack機能を用いて$\SI{2}{s}$(20 フレーム)毎に取り出し,動画化した。
\item 初期フレーム($\SI{0.10}{s}$)を全体から Difference 機能で引き算をし,背景画像を消去した。
\item 金属樹の部分のみをCrop機能で切り出し,金属樹以外の影はClear 機能で消去した。
\item 編集した動画を$\SI{30}{s}$ (15 フレーム)毎に取り出し,輝度値の閾値を30として二値化した。
\item 輝度値が0でないピクセルが初めて現れた時刻を$t=0$とした。そのフレームで輝度値を持つピクセルの重心を求め,その座標を中心(陰極線の位置)とした。

\subsubsection{金属樹の外観・フラクタル次元}
界面活性剤濃度による形態変化を定量化するために,金属樹の最終形状のフラクタル次元を求めた。フラクタル次元は\textbf{ボックスカウンティング法}を用いて求めた。ボックスカウンティング法とは与えられたパターンを様々なスケール(大きさ)の正方形で覆い,その正方形の中にパターンが含まれるか否かを数えることでフラクタル次元を求める方法である。
\begin{figure}[H]
  \centering
  \includegraphics[width=0.5\textwidth]{../../figure/part2(exp_deposition)/boxcounting.png}
  \caption{ボックスカウンティング法の模式図\cite{表面粗さ曲線のフラクタル解析}}
  \label{fig:box_counting}
\end{figure}
スケールを$\varepsilon$,パターンが少しでも含まれるボックスの個数を$N(\varepsilon)$とすると,フラクタル次元$D_f$とボックスの数の関係ははおおよそ$N(\varepsilon)\propt\varepsilon^{\D_f}$となる。

ボックスの大きさ$\varepsilon$を変え,$N(\varepsilon)$を求め,両対数プロットすることでフラクタル次元を求めた。
\subsubsection{枝の本数・太さのトラッキング}
金属樹の形態を特徴づけるために,枝やその太さがどのように変化するかを解析した。枝は陰極である\ce{Zn}線から等方的に広がっていると仮定し,最終形状について中心からの距離$r$の円と交差する枝の本数やその太さを求めた。
\begin{figure}[htbp]
  \begin{minipage}
    {0.5\textwidth}
    \subcaption{}
    \centering
    \includegraphics[width=0.9\textwidth]{../../figure/part2(exp_deposition)/0.00_circle.png}
    \label{fig:0.00_circle}
  \end{minipage}
  \begin{minipage}
    {0.5\textwidth}
    \subcaption{}
    \centering
    \includegraphics[width=0.9\textwidth]{../../figure/part2(exp_deposition)/0.05_circle.png}
    \label{fig:0.05_circle}
  \end{minipage}
  \caption{計測に用いた半径$r$の円(図中赤線)。交差する枝の本数と太さを計測した。\subref{fig:0.00_circle}界面活性剤濃度 0\%($\SI{3795}{s}$)の実験画像。\subref{fig:0.05_circle}界面活性剤濃度 0.05\%($\SI{821}{s}$)の実験画像}
\end{figure}
\subsubsection{枝の長さ・分岐角度の計測}
フラクタル次元で定量化できる全体形状の特徴よりもよりも細かい金属樹の構造を解析するために,最終形状について枝の長さや分岐角度を解析し,界面活性剤濃度による形態変化を定量化した。解析手順は以下の通りである。
\begin{enumerate}
  \item 陰極線の位置を原点とし,適当な半径$r$に対して動径方向に走査し,輝度値が$0\rightarrow1\rightarrow0$となる部分の値1の部分を一本の枝の太さとした。取得したピクセル位置の重心の曲座標上での半径$r$と角度$\theta$を計測した。
  \item $r$が大きい側から最近接の重心位置を結び,複数の点と最近接となる点を分岐した点とみなした。図\ref{fig:branch_def_input}\subref{fig:branch_def}で定義される角度$\theta_{\mathrm{in}}$,$\theta_{\mathrm{out}}$と,分岐点をつなぐ枝の長さ,$\theta_{\mathrm{out}}\geq 0.9\pi$となる枝どうしをつないだ枝の長さを計測し,分岐角度を求めた。
\end{enumerate}
\begin{figure}
  \begin{minipage}
    {0.32\textwidth}
    \subcaption{}
    \centering
    \includegraphics[width=0.9\textwidth]{../../figure/part2(exp_deposition)/branch_def.png}
    \label{fig:branch_def}
  \end{minipage}
  \begin{minipage}
    {0.32\textwidth}
    \subcaption{}
    \centering
    \includegraphics[width=0.9\textwidth]{../../figure/part2(exp_deposition)/den_input.png}
    \label{fig:den_input}
  \end{minipage}
  \begin{minipage}
    {0.32\textwidth}
    \subcaption{}
    \centering
    \includegraphics[width=0.9\textwidth]{../../figure/part2(exp_deposition)/den_analisys.png}
    \label{fig:den_analisys}
  \end{minipage}
  \label{fig:branch_def_input}
  \caption{枝の長さと角度の解析方法。\subref{fig:branch_def}枝の長さと角度の定義。\subref{fig:den_input}入力画像。\subref{fig:den_analisys}解析画像。オレンジ色の点が枝の分岐点,青色の点が端点になっている。}
\end{figure}
入力画像\ref{fig:branch_def_input}\ref{fig:den_input}に対して,解析結果は図\ref{fig:branch_def_input}\ref{fig:den_analisys}のように得られる。

$\theta_{\mathrm{out}}$の定義は,分岐点から生えた枝のうち,前の枝との角度が最大となるものとした。$\theta_{\mathrm{out}}$が$0.9\pi$以上になっている枝どうしを一本の枝とみなす定義は,一本の幹のような枝から複数の細かい枝が生える場合の特徴づけ,計測を目的としている。
\subsubsection{分布の推定}
取得した枝の長さのデータから,その分布を計測した。枝の長さについて,金属樹がフラクタル構造を持つことより,冪分布$f(x)=ax^{-b}$になっていると仮定して\textbf{相補累積分布関数}を用いて推定を行った。

確率密度関数$f(x)$が存在するとき,確率変数$X$がある値$x$以上になる確率を表す関数$\bar{F}(x)$を\textbf{相補累積分布関数}と呼び,
\begin{equation}
  \bar{F}(x) = P(X \geq x) = \int_{x}^{\infty} f(x')dx'
  \label{eq:complementary_cdf}
\end{equation}
と定義される。

\begin{wrapfigure}{r}[0pt]{0.33\textwidth}
  \begin{center}
    \includegraphics[scale=0.5]{../../figure/part2(exp_deposition)/fitting_exp.png}
  \end{center}
  \caption{冪分布のフィッティングの例}
  \label{fig:fitting_exp}
\end{wrapfigure}

フィッティング関数は式\eqref{eq:complementary_cdf}より以下のようになる。
\begin{equation}
  \begin{split}
    \bar{F}(x) & = \int_{x}^{\infty} ax'^{-b}dx' = \left[ \frac{a}{1-b}x'^{1-b} \right]_{x}^{\infty} \\
               & = \frac{a}{b-1}x^{1-b}
  \end{split}
\end{equation}

得られたデータのうち,枝の太さ程度($\sim0.15\si{cm}$)以下のものはノイズによるものとして除外し,冪分布のパラメータ$b$を推定し,界面活性剤濃度による分布の変化を調べた。今回の解析では,冪分布の変化を見たいという点と,フィッティング範囲がデータによって異なり,また一部分のみ(冪関数とみなせる範囲)のフィッティングであることより,パラメータ$a$の値はブレが大きいと考えられるため,パラメータ$b$のみを用いて解析を行った。

\ifdraft{
  \bibliographystyle{../../Preamble/Physics.bst}
  \bibliography{../../Preamble/reference.bib}
}{}
\end{document}