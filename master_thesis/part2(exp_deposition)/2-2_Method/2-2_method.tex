\documentclass[autodetect-engine,dvi=dvipdfmx,a4paper,ja=standard,oneside,openany,11pt]{bxjsbook}
\usepackage{../../Preamble/mypackage}

\begin{document}
\chapter{実験:金属樹}
\section{実験・解析方法}
\subsection{金属樹析出の実験方法・実験系}
硫酸亜鉛七水和物\ce{ZnSO_4.7H_2O}(富士フイルム和光純薬)の$\SI{2}{M}$水溶液を作成し,そこにlevelerとされる非イオン性界面活性剤のPluronic F-127(フナコシ)を加え,その濃度を変えて電解析出を行った。実験系のセットアップについては図\ref{fig:system_exp},\ref{fig:system_exp_whole}を参照。

\begin{figure}[htbp]
  \begin{minipage}
    {0.55\textwidth}
    \subcaption{}
    \centering
    \includegraphics[width=0.9\textwidth]{../../figure/part2(exp_deposition)/system_side.png}
    \label{fig:system_side}
  \end{minipage}
  \begin{minipage}
    {0.43\textwidth}
    \subcaption{}
    \centering
    \includegraphics[width=0.9\textwidth]{../../figure/part2(exp_deposition)/system_top.png}
    \label{fig:system_top}
  \end{minipage}
  \caption{実験系の模式図。\subref{fig:system_side}横から見た模式図。\subref{fig:system_top}上から見た模式図。}
  \label{fig:system_exp}
\end{figure}

\begin{figure}[htbp]
  \centering
  \includegraphics[width=0.5\textwidth]{../../figure/part2(exp_deposition)/exp_sys_whole.png}
  \caption{周辺機器も含めた実験系の外観}
  \label{fig:system_exp_whole}
\end{figure}

実験手順は以下の通りである。

\begin{enumerate}
  \item 表面処理装置(Electro-Technic Products BD-20A)を用いてガラスシャーレ(直径$\SI{110}{mm}$)の表面を励起させ,溶液を広がりやすくした。
  \item \ce{ZnSO_4.7H_2O} $\SI{2}{M}$水溶液に非イオン性界面活性剤であるPluronic F-127を加えた溶液を$\SI{15}{ml}$作成した。この時界面活性剤濃度は$\SI{0}{\mathrm{vol}\%}, \SI{0.005}{\mathrm{vol}\%}, \SI{0.01}{\mathrm{vol}\%}, \SI{0.03}{\mathrm{vol}\%}, \SI{0.05}{\mathrm{vol}\%}$になるように調整した。
  \item $\SI{15}{ml}$の溶液のうち,$\SI{3}{ml}$を用いてガラスシャーレの表面を洗浄した。
  \item 陽極として\ce{Zn}極板(約$\SI{20}{mm}\times$約$\SI{100}{mm}$)を3枚用意し,ガラスシャーレの側面に沿うように湾曲させて設置した。この時,極板が元の形状に戻るようにしなるため,3枚の極板を重ねて設置し,ガラスシャーレ側面に互いに押し付け合うように極板を設置,固定した。
  \item ガラスシャーレに残りの溶液$\SI{12}{ml}$を入れた。(溶液厚:$\sim\SI{1.26}{mm}$)
  \item 陰極である\ce{Zn}線(直径$\SI{0.5}{mm}$,長さ約$\SI{350}{mm}$)を穴をあけたアクリル板に通して固定し,陰極の先が溶液に接しないようにアクリル板をガラスシャーレの上に固定した。
  \item 陰極の先が液面に触れるまでジャッキでライトプレート(HAKUBA ライトビュアー 7000PRO)を上昇させ,ガラスシャーレの中心に電極を設置した。
  \item 電極と電源装置(KENWOOD PA18-5B),デジタルマルチメータ(ADCMT 7352A/E)を繋ぎ,電極間に$\SI{5}{V}$の電圧を印加した。室温は$\SI{21}{\degreeCelsius}$から$\SI{23}{\degreeCelsius}$程度だった。
  \item 電解析出していく過程を,USBカメラ(オムロンセンテック STC-MBS43U3V 画素数 $\SI{720}{px} \times \SI{540}{px}$, 時間分解能$\SI{527.1}{fps}$)にレンズ(HOZAN L-600-12, 焦点距離$\SI{12}{mm}$)を取り付け$\SI{10}{fps}$で撮影した。
\end{enumerate}

\subsection{解析方法}
\subsubsection{データの二値化処理}
取得したデータは以下の手順で二値化した。
\begin{enumerate}
  \item 動画データをImageJ (Fiji)のMake Substack機能を用いて$\SI{2}{s}$ (20 フレーム)毎に取り出し,動画化した。
  \item 全フレームに対して,初期フレーム($\SI{0.10}{s}$)を Difference機能で引き算し,背景画像を消去した。
  \item 金属樹の部分のみをCrop機能で切り出し,金属樹以外の影はClear 機能で消去した。
  \item 編集した動画を$\SI{30}{s}$ (15 フレーム)毎に取り出し,輝度値の閾値を30として二値化した。
  \item 二値化画像で値が0でないピクセルが初めて現れたフレーム時刻$t$を$t=0$とした。そのフレームで値を持つピクセルの重心を求め,その座標を中心(陰極線の位置)とした。
\end{enumerate}
以降の金属樹の解析ではこの二値化データを用いた。
\subsubsection{金属樹の外観・フラクタル次元}
界面活性剤濃度によるパターン変化を定量化するために,最終フレームの画像に対してボックスカウンティング法を行い,金属樹のフラクタル次元を求めた。
\subsubsection{枝の本数・太さのトラッキング}
金属樹の形態を特徴づけるために,枝やその太さがどのように変化するかを解析した。枝は陰極である\ce{Zn}線から等方的に広がっていると仮定し,最終フレームの画像について,図\ref{fig:circle}のように,中心からの距離$r$の円と交差する枝の本数やその太さを求めた。陰極線の位置を極座標の原点とし,適当な半径$r$に対して角度方向に走査した。二値化画像のピクセルの値が$0\rightarrow1\rightarrow0$となった時の1の部分を一本の枝とし,1の個数を枝の太さとした。

\begin{figure}[htbp]
  \begin{minipage}
    {0.5\textwidth}
    \subcaption{}
    \centering
    \includegraphics[width=0.9\textwidth]{../../figure/part2(exp_deposition)/0.00_circle.png}
    \label{fig:0.00_circle}
  \end{minipage}
  \begin{minipage}
    {0.5\textwidth}
    \subcaption{}
    \centering
    \includegraphics[width=0.9\textwidth]{../../figure/part2(exp_deposition)/0.05_circle.png}
    \label{fig:0.05_circle}
  \end{minipage}
  \caption{計測に用いた半径$r$の円(図中赤線)。交差する枝の本数と太さを計測した。\subref{fig:0.00_circle}界面活性剤濃度 $\SI{0}{\mathrm{vol}\%}$, $\SI{3787}{s}$の実験画像。\subref{fig:0.05_circle}界面活性剤濃度$\SI{0.05}{\mathrm{vol}\%}$, $\SI{821}{s}$の実験画像。}
  \label{fig:circle}
\end{figure}

\subsubsection{枝の長さ・分岐角度の計測}
全体形状の特徴はフラクタル次元で定量化できる。しかし全体形状以下の構造の特徴は隠れてしまう。そこで,より細かい金属樹の構造を解析するために,最終フレームの画像について枝の長さや分岐角度を解析し,界面活性剤濃度による形態変化を定量化した。解析手順は以下の通りである。
\begin{enumerate}
  \item 陰極線の位置を極座標の原点とし,適当な半径$r$に対して角度方向に走査した。二値化画像のピクセルの値が$0\rightarrow1\rightarrow0$となった時の1の部分を一本の枝とし,値が1のピクセルの重心の極座標半径$r$と角度$\theta$を計測した。一本の枝が終わったら,角度方向に走査を繰り返し,これを$r=\SI{2}{px}$から,データ内のピクセルで原点から最も遠い位置にあるピクセルの半径$r_{\mathrm{max}}$まで走査を行った。
  \item $r$が大きい側から最近接の重心位置を結び,複数の点と最近接となる点を分岐した点とみなした。図\ref{fig:branch_def_input}\subref{fig:branch_def}で定義される角度$\theta_{\mathrm{in}}$,$\theta_{\mathrm{out}}$と,分岐点と分岐点や分岐点と端点をつなぐ枝(ピンク色の枝)の長さ,$\theta_{\mathrm{out}}\geq 0.9\pi$となる枝どうしをつないだ枝(緑色の枝)の長さを計測し,分岐角度を求めた。
\end{enumerate}

\begin{figure}
  \begin{minipage}
    {0.32\textwidth}
    \subcaption{}
    \centering
    \includegraphics[width=0.9\textwidth]{../../figure/part2(exp_deposition)/branch_def.png}
    \label{fig:branch_def}
  \end{minipage}
  \begin{minipage}
    {0.32\textwidth}
    \subcaption{}
    \centering
    \includegraphics[width=0.9\textwidth]{../../figure/part2(exp_deposition)/den_input.png}
    \label{fig:den_input}
  \end{minipage}
  \begin{minipage}
    {0.32\textwidth}
    \subcaption{}
    \centering
    \includegraphics[width=0.9\textwidth]{../../figure/part2(exp_deposition)/den_analisys.png}
    \label{fig:den_analisys}
  \end{minipage}
  \caption{枝の長さと角度の解析方法。\subref{fig:branch_def}枝の長さと角度の定義。\subref{fig:den_input}入力画像。\subref{fig:den_analisys}解析画像。オレンジ色の点が枝の分岐点,青色の点が端点になっている。}
  \label{fig:branch_def_input}
\end{figure}

入力画像\ref{fig:branch_def_input}\subref{fig:den_input}に対して,処理結果は図\ref{fig:branch_def_input}\subref{fig:den_analisys}のように得られる。

$\theta_{\mathrm{out}}$の定義は,分岐点から生えた枝のうち,前の枝と成す角度が最大となるものとした。$\theta_{\mathrm{out}}$が$0.9\pi$以上になっている枝どうしを一本の枝とみなす定義は,一本の幹のような枝から複数の細かい枝が生える場合の特徴づけ,計測を目的としている。
\subsubsection{フィッティングによる分布の計測}
取得した枝の長さのデータから,その分布を計測した。枝の長さについて,金属樹がフラクタル構造を持つことより,冪分布$f(x)=ax^{-b}$になっていると仮定して\textbf{相補累積分布関数}を用いて図\ref{fig:fitting_exp}のようにフィッティングを行った。

確率密度関数$f(x)$が存在するとき,確率変数$X$がある値$x$以上になる確率を表す関数$\bar{F}(x)$を\textbf{相補累積分布関数}と呼び,
\begin{equation}
  \bar{F}(x) = P(X \geq x) = \int_{x}^{\infty} f(x')dx'
  \label{eq:complementary_cdf}
\end{equation}
と定義される。

フィッティング関数は式\eqref{eq:complementary_cdf}より以下のようになる。
\begin{equation}
  \begin{split}
    \bar{F}(x) & = \int_{x}^{\infty} ax'^{-b}dx' = \left[ \frac{a}{1-b}x'^{1-b} \right]_{x}^{\infty} \\
               & = \frac{a}{b-1}x^{1-b}
  \end{split}
\end{equation}

\begin{figure}[htbp]
  \centering
  \includegraphics[width=0.5\textwidth]{../../figure/part2(exp_deposition)/fitting_exp.png}
  \caption{冪分布のフィッティングの例。$\SI{0.05}{\mathrm{vol}\%}$の時の,図\ref{fig:branch_def_input}\subref{fig:branch_def}で定義される緑とピンク色の枝の分布を表している。矢印で表した範囲($\SI{10}{px}\approx\SI{1.6}{mm}$以上)でフィッティングを行った。}
  \label{fig:fitting_exp}
\end{figure}


枝の太さの結果\ref{fig:branch}\subref{fig:branch_number}より,枝の太さは太くても,おおむね$\SI{10}{px}\approx\SI{1.6}{mm}$であることがわかる。そのため,枝の長さが$\SI{1.6}{mm}$に満たないものはノイズ(例えるならば,幹に生えたコブ)によるものとして除外した。
% (*説明として適切か?先の結果を手順の段階で参照してもいいのか?)
残った枝について冪分布の指数$b$を測定し,界面活性剤濃度による分布の変化を調べた。今回の解析では,冪分布の変化を見たいという点と,指数$a$の値はブレが大きいと考えられる点より,指数$b$のみを用いて解析を行った。

\ifdraft{
  \bibliographystyle{../../Preamble/Physics.bst}
  \bibliography{../../Preamble/reference.bib}
}{}

\end{document}