\documentclass[autodetect-engine,dvi=dvipdfmx,a4paper,ja=standard,oneside,openany,11pt,draft]{bxjsbook}
\usepackage{../../Preamble/mypackage}

\begin{document}

\begin{titlepage}
  \newgeometry{top=30mm}
  \fontsize{20pt}{25pt}\selectfont{\textbf{概要}}
  \vspace{10mm}\\
  \fontsize{11pt}{15pt}\selectfont{

    非平衡状態では平衡系には見られない特徴的なパターンが現れることが知られている。樹枝状パターンは血管や河川など幅広い長さスケールでみられる非平衡パターンで,自己相似構造といった物理的に興味深い特徴を持つ。樹枝状パターンの枝は,成長している枝の先端が分岐することで形成される。そのため,先端のミクロな幾何学的形状は枝の分岐に大きな影響を与えると予想される。

    しかし,既存の研究ではパターンのフラクタル次元といったマクロな特徴量に着目したものが多く,ミクロな先端形状とマクロな樹枝状パターンとの関係は十分に議論されていない。


    本研究では,亜鉛の電解析出で生じる樹枝状の金属結晶である金属樹を用いて両スケールの対応を検討した。界面活性剤\Red{などの有機物質(*追加)}を溶液に添加すると成長途中の界面のミクロな幾何学的形状($\sim\mathcal{O}(\SI{e-4}{m})$)が変化することが報告されている。そこで,溶液の界面活性剤濃度を変化させ,\Red{金属樹を生成し,そのマクロな構造を,フラクタル次元に加え,分岐角度や枝の長さ( $\sim\mathcal{O}(\SI{e-2}{m})$)などを測定することで議論した。また,金属樹の生成過程の時間発展を拡大観察した所,界面活性剤を添加することで成長界面の粗さの成長が遅くなることが確認された。(*順番を章構成と同じにした。)}

    その結果,界面活性剤の添加により分岐角度の分布はあまり変化しない一方,枝長の分布は長い側に広がることが明らかになった。

    また,実験結果の再現を目的に樹枝状パターンを再現するモデルである拡散律速凝集(Diffusion Limited Aggregation : DLA)モデルをベースに数値計算を行った。

    その結果,イオンの運動性やDLAクラスターへの固着確率といったパラメータは,樹枝状パターンのフラクタル次元を増加させることが分かった。実験ではフラクタル次元が低下していたことを考えると,拡散よりも界面での反応が重要で,周辺の形状に依存したより複雑な選択則がモデルには必要であることが示唆された。

    本研究は、成長界面のミクロな幾何学的形状がマクロなパターン形成に及ぼす影響を解明し、金属樹に限らず、広く枝分かれ構造を持つパターンの予測に役立つ知見を与えるものである。}
\end{titlepage}

\ifdraft{
  \bibliographystyle{../Preamble/Physics.bst}
  \bibliography{../Preamble/reference.bib}
}{}
\end{document}