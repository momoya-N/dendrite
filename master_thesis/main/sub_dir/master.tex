\documentclass[autodetect-engine,dvi=dvipdfmx,a4paper,ja=standard]{bxjsbook}
\usepackage{docmute}
\usepackage{../../Preamble/mypackage}

\begin{document}

\documentclass[autodetect-engine,dvi=dvipdfmx,a4paper,ja=standard,11pt,titelpage,draft]{bxjsbook}
\usepackage{../../Preamble/mypackage}

\begin{document}
\begin{titlepage}
  % \newgeometry{top=30truemm,bottom=30truemm,left=20truemm,right=20truemm}
  \newgeometry{top=30mm}
  \begin{center}
    \fontsize{16pt}{25pt}\selectfont{千葉大学大学院融合理工学府\\修士論文\\}
    \vspace{20mm}
    \fontsize{25pt}{30pt}\selectfont{界面活性剤の添加による\\金属樹のマクロなパターン変化\\}
    \vspace{10mm}
    \fontsize{20pt}{25pt}\selectfont{令和7年3月 提出\\}
    \vspace{50mm}
    \fontsize{12pt}{16pt}\selectfont{先進理化学専攻 物理学コース\\}
    \vspace{10mm}
    \fontsize{12pt}{18pt}\selectfont{中村 友哉\\}
  \end{center}
\end{titlepage}
% \title{千葉大学大学院融合理工学府\\
%   修士論文\\
%   \vspace{20mm}
%   {\huge 界面活性剤の添加による金属樹の\\マクロなパターン変化}
% }
% \date{\today}
% \author{中村友哉}
% \thanks{
%   \begin{center}
%     \vspace{50mm}
%     \fontsize{20pt}{25pt}\selectfont{千葉大学融合理工学府 \\先進理化学専攻 物理学コース\\}
%     \vspace{10mm}
%     \fontsize{20pt}{25pt}\selectfont{23WM2103\\}
%     \vspace{3mm}
%     \fontsize{25pt}{45pt}\selectfont{中村 友哉\\}
%     \vspace{10mm}
%     \fontsize{20pt}{25pt}\selectfont{\today}
%   \end{center}
% }


% \maketitle
\end{document}

\thispagestyle{empty}

\restoregeometry
\pagenumbering{roman}
\documentclass[autodetect-engine,dvi=dvipdfmx,a4paper,ja=standard,oneside,openany,11pt,draft]{bxjsbook}
\usepackage{../../Preamble/mypackage}

\begin{document}

\begin{titlepage}
  \newgeometry{top=30mm}
  \fontsize{20pt}{25pt}\selectfont{\textbf{概要}}
  \vspace{10mm}\\
  \fontsize{11pt}{15pt}\selectfont{

    非平衡状態では平衡系には見られない特徴的なパターンが現れることが知られている。樹枝状パターンは血管から河川まで幅広い長さスケールでみられる非平衡パターンで,自己相似構造といった物理的に興味深い特徴を持つ。樹枝状パターンの枝は成長している枝の先端が分岐することで形成される。そのため,先端のミクロな幾何学的形状は枝の分岐に大きな影響を与えると予想される。

    しかし,既存の研究ではパターンのフラクタル次元といったマクロな特徴量に着目したものが多く,ミクロな先端形状とマクロな樹枝状パターンとの関係は十分に議論されていない。


    本研究では,亜鉛の電解析出で生じる樹枝状の金属結晶である金属樹を用いて両スケールの対応を検討した。電解析出において,界面活性剤などの添加物を電解質溶液に加えると,成長途中の界面のミクロな幾何学的形状($\SI{e-1}{mm}$程度まで)が変化することが報告されている。そこで,電解質溶液の界面活性剤濃度を変化させて金属樹を生成し,そのマクロな構造を解析した。解析はフラクタル次元に加え,分岐角度や枝の長さ( $\SI{10}{mm}$程度まで)を測定することで議論した。また,金属樹の生成過程の時間発展を拡大観察した所,界面活性剤を添加することで成長界面の粗さの時間発展が遅くなることが確認された。

    その結果,界面活性剤の添加により分岐角度の分布はあまり変化しない一方,枝長の分布は長い側に広がることが明らかになった。

    また,実験結果の再現を目的に樹枝状パターンを再現するモデルである拡散律速凝集(Diffusion Limited Aggregation : DLA)モデルをベースに数値計算を行った。

    その結果,イオンの運動性や樹枝状パターンへの固着確率といったパラメータが樹枝状パターンのフラクタル次元を増加させていた。実験ではフラクタル次元が低下していたことを考えると,拡散よりも界面での反応が重要で,周辺の形状に依存したより複雑な選択則がモデルには必要であることが示唆された。

    本研究は、成長界面のミクロな幾何学的形状がマクロなパターン形成に及ぼす影響を解明し、金属樹に限らず、広く枝分かれ構造を持つパターンの予測に役立つ知見を与えるものである。}
\end{titlepage}

\ifdraft{
  \bibliographystyle{../Preamble/Physics.bst}
  \bibliography{../Preamble/reference.bib}
}{}
\end{document}
\tableofcontents
\pagenumbering{arabic}

% Part1 General Introduction
\documentclass[autodetect-engine,dvi=dvipdfmx,a4paper,ja=standard,oneside,openany,11pt]{bxjsbook}
\usepackage{../../Preamble/mypackage}

\begin{document}
\chapter{序論}
\section{パターン形成の物理学}
自然界では絶えずエネルギーの注入や散逸,物質の輸送が起こっている。このような系を平衡系と区別して\textbf{非平衡系}と呼ぶ。その中でも,非平衡過程により生じる様々な秩序構造の形成メカニズムや統計的性質に着目する分野を\textbf{パターン形成の物理学}と呼ぶ。パターンの形成メカニズムは様々なものがあり,図\ref{fig:pattern_formation}\subref{fig:BZ}のように連続した酸化還元反応によるもの,図\ref{fig:pattern_formation}\subref{fig:reaction_diffusion_angelfish}のように生体組織内の物質の拡散によるもの,図\ref{fig:pattern_formation}\subref{fig:Benard_cell}のように熱対流によるものなどが挙げられる。

\begin{figure}[htbp]
  \centering
  \begin{minipage}
    {0.32\textwidth}
    \subcaption{}
    \centering
    \includegraphics[width=0.9\textwidth]{../../figure/part1/BZ_reaction.png}
    \label{fig:BZ}
  \end{minipage}
  \begin{minipage}
    {0.32\textwidth}
    \subcaption{}
    \centering
    \includegraphics[width=0.9\textwidth]{../../figure/part1/reaction_diffusion_angelfish.png}
    \label{fig:reaction_diffusion_angelfish}
  \end{minipage}
  \begin{minipage}
    {0.32\textwidth}
    \subcaption{}
    \centering
    \includegraphics[width=0.9\textwidth]{../../figure/part1/Benard_cell.png}
    \label{fig:Benard_cell}
  \end{minipage}
  \caption{様々な非平衡系でのパターン形成。\subref{fig:BZ}Belousov-Zhabotinsky反応によるスパイラルパターン\cite{BZ_reaction}。円形容器の直径はおおよそ$\SI{50}{mm}$。\subref{fig:reaction_diffusion_angelfish}サザナミヤッコ(キンチャクダイの一種)の体表のTuringパターン。写真は横幅がおおよそ$\SI{50}{mm}$\cite{kondo1995reaction}。\subref{fig:Benard_cell}熱対流によるB{\'e}nard セル。円形容器の直径は$\SI{120}{mm}${\cite{eckert1998square}}。}
  \label{fig:pattern_formation}
\end{figure}

\section{枝分かれと樹枝状パターン}
非平衡系のパターンの中でも枝分かれをくりかえして形成される\textbf{樹枝状パターン}は生物組織(図\ref{fig:pattern_formation_dendrite}\subref{fig:blood_vessel_nerve})や電解析出(図\ref{fig:pattern_formation_dendrite}\subref{fig:electro_deposition}),落雷のような絶縁破壊(図\ref{fig:pattern_formation_dendrite}\subref{fig:thunder}),フィヨルド(図\ref{fig:pattern_formation_dendrite}\subref{fig:fjord})のように自然界で広くみられるパターンである。$\SI{e-4}{m}$から$\SI{e5}{m}$ほどの幅広いスケールで見られ,スケールも形成メカニズムも異なるが,普遍的に見られるパターンである。このようなパターンは\textbf{フラクタル(自己相似)構造}をもち,パターンの一部を相似拡大・縮小したものの統計的な性質が元のパターンと一致する性質を持っている。

\begin{figure}[htbp]
  \begin{tabular}{cc}
    \begin{minipage}[t]{0.45\textwidth}
      \subcaption{}
      \centering
      \includegraphics[keepaspectratio, scale=0.8]{../../figure/part1/blood_vessel_nerve.png}
      \label{fig:blood_vessel_nerve}
    \end{minipage} &
    \begin{minipage}[t]{0.45\textwidth}
      \subcaption{}
      \centering
      \includegraphics[keepaspectratio, scale=0.8]{../../figure/part1/electro_deposition.png}
      \label{fig:electro_deposition}
    \end{minipage} \\

    \begin{minipage}[t]{0.45\textwidth}
      \subcaption{}
      \centering
      \includegraphics[keepaspectratio, scale=0.8]{../../figure/part1/thunder.jpg}
      \label{fig:thunder}
    \end{minipage}            &
    \begin{minipage}[t]{0.45\textwidth}
      \subcaption{}
      \centering
      \includegraphics[keepaspectratio, scale=0.8]{../../figure/part1/fjord.jpg}
      \label{fig:fjord}
    \end{minipage}
  \end{tabular}
  \caption{自然界に見られる様々な樹枝状パターン。\subref{fig:blood_vessel_nerve}血管(左)と神経(右)。スケールバー:$\SI{100}{\mu m}=\SI{e-4}{m}$ \cite{mukouyama2002sensory}。\subref{fig:electro_deposition}電解析出による亜鉛の金属樹。中村撮影。パターンの直径はおおよそ$\SI{e-2}{m}$。\subref{fig:thunder}落雷(Wikipedia)。長さスケールはおおよそ$\SI{e3}{m}$。\subref{fig:fjord}フィヨルド(Wikipedia)。河口からの距離はおおよそ$\SI{e5}{m}$。}
  \label{fig:pattern_formation_dendrite}
\end{figure}

\section{フラクタル次元}
\label{sec:fractal_dimension}
\subsection{フラクタル次元の定義}
フラクタル構造を特徴づける量として\textbf{フラクタル次元}が挙げられる。フラクタル次元の定義は様々であるが,ここでは\textbf{相似次元}を用いる。相似次元は次のように定義される。
\begin{equation}
  D_f=-\odv{\log N(\varepsilon)}{\log \varepsilon}
  \label{eq:fractal_dimension_def}
\end{equation}

ここで$N(\varepsilon)$は与えられたパターンを$\varepsilon$のスケールで埋めつくすのに必要な個数(長さ$\varepsilon^{\mathrm{空間次元}}$の物差しで何個になるか)である。フラクタル次元は,図形の複雑さを表す指標であり,整数である場合はユークリッド空間における次元を表す。例えば,ユークリッド空間中(図\ref{fig:fractal_stracture}\subref{fig:相似次元の考え方})では,スケールは$\varepsilon=1/l$で与えられる。$\varepsilon=1/1,1/2,1/3$と減少していくとパターンの数$N(\varepsilon)$は増加する。式\eqref{eq:fractal_dimension_def}を用いて計算すると,空間次元$D=1,2,3$に一致する。図\ref{fig:fractal_stracture}\subref{fig:シェルピンスキーのギャスケット}はSierpinskiのギャスケットと呼ばれるパターンである。このパターンは三角形のスケール(三角形の一辺の長さ)を半分にして,中央の三角形を取り去ることで構成される。スケール$\varepsilon$が$\varepsilon=1/2$になると,一辺の長さ$\varepsilon$の三角形の個数$N(\varepsilon)$が3つになる。よって式\eqref{eq:fractal_dimension_def}より$D_f=\log 3/\log 2=1.5849\cdots$となる。フラクタル構造を持つ系はスケールフリー性を持ち,系の大きさ(スケール)によらず同じ構造(パターン配置)を持つ。このような\textbf{スケール不変性}を持つ系では,あるスケールで計測できる量(パターンの数,密度等)にスケールの長さに対する冪関数則が成り立つことが知られている。

\begin{figure}[htbp]
  \begin{minipage}{0.45\textwidth}
    \centering
    \subcaption{}
    \includegraphics[scale=0.5]{../../figure/part1/Fractaldimensionexample.png}
    \label{fig:相似次元の考え方}
  \end{minipage}
  \begin{minipage}{0.45\textwidth}
    \centering
    \subcaption{}
    \includegraphics[scale=0.5]{../../figure/part1/Sierpinski_triangle.png}
    \label{fig:シェルピンスキーのギャスケット}
  \end{minipage}
  \caption{相似次元の考え方。\subref{fig:相似次元の考え方}各スケール毎のパターンの個数の変化(Wikipedia)。\subref{fig:シェルピンスキーのギャスケット} Sierpinskiのギャスケット(Wikipedia)。相似次元(フラクタル次元)は$D_f=\log 3/\log 2=1.5849\cdots$。}
  \label{fig:fractal_stracture}
\end{figure}

\subsection{フラクタル次元の計測法}
フラクタル次元の計測法にはいくつかの方法があるが,この節では本研究の解析の理解に必要なものに絞って記述する。計測法については文献\cite{フラクタルの物理Ⅰ}を参考にした。
\subsubsection{ボックスカウンティング法}
\begin{figure}[htbp]
  \centering
  \includegraphics[width=0.6\textwidth]{../../figure/part2(exp_deposition)/boxcounting.png}
  \caption{ボックスカウンティング法の模式図\cite{表面粗さ曲線のフラクタル解析}。}
  \label{fig:box_counting}
\end{figure}

ボックスカウンティング法とは,例えば図\ref{fig:box_counting}のように2次元空間内で与えられたパターンのフラクタル次元を求める場合,パターンを様々なスケール(大きさ)の正方形(2次元の場合)で覆い,その正方形の中にパターンが含まれるか否かを数えることでフラクタル次元を求める方法である。

スケールを$\varepsilon$,パターンが少しでも含まれるボックスの個数を$N(\varepsilon)$とすると,フラクタル次元$D_f$とボックスの数の関係はおおよそ$N(\varepsilon)\propto\varepsilon^{D_f}$となる。ボックスの大きさ$\varepsilon$を変え,$N(\varepsilon)$を求め,両対数プロットすることでその傾きからフラクタル次元を求められる。ボックスの大きさを変えて個数を計測するだけのため,コンピュータによる実装が簡単な反面,他の計測法に比べて精度が低い。
\subsubsection{回転半径法}
図\ref{fig:DLA_with_R_g}のようにランダムな要素を含むフラクタルパターンの大きさの目安を与えるのが\textbf{回転半径}である。回転半径$R_g$は
\begin{align}
  R_g & =\sqrt{\frac{1}{N}\sum_{i=1}^{N}(\bm{r}_i-\bm{r}_{\mathrm{c}})^2}
  \label{eq:gyration_radius}
\end{align}
と与えられる。ただし,$\bm{r}_c$はパターンの重心で,
\begin{equation}
  \bm{r}_{\mathrm{c}}=\frac{1}{N}\sum_{i=1}^{N}\bm{r}_i
\end{equation}
である。

\begin{figure}[htbp]
  \centering
  \includegraphics[width=0.4\textwidth]{../../figure/part3/DLA_with_R_g.png}
  \caption{樹枝状パターンの回転半径。おおよそ$R_g=\SI{108}{px}$。$R_g$がおおよそのパターンの大きさを与える。}
  \label{fig:DLA_with_R_g}
\end{figure}

パターンに含まれる粒子数$N$はユークリッド空間の面積や体積と同様に,回転半径$R_g$との間に,
\begin{equation}
  N\propto R_g^{D_f}
  \label{eq:gyration_radius_fractal}
\end{equation}
のような関係が成り立つ。$D_f$はフラクタル次元である。パターンの成長過程で,一定ステップごとに粒子数$N$と回転半径$R_g$を計測し,$R_g$と$N$の関係を両対数プロットすることで,その傾きからフラクタル次元$D_f$を求めることができる。この手法を\textbf{回転半径法}といい,平均操作が入っているため精度よく$D_f$を求めることができる。

\subsubsection{密度相関関数法}
\label{sec:density_correlation}
密度相関関数法は,パターンの密度分布を用いて,フラクタル次元を求める方法である。パターンの密度分布を$\rho(\bm{r})$とする。密度分布$\rho(\bm{r})$は図\ref{fig:density_func_dif}のように,
\begin{equation}
  \rho(\bm{r})=
  \begin{cases}
    1 & (\mathrm{位置}\bm{r}\mathrm{のピクセルがパターン内}) \\
    0 & (\mathrm{位置}\bm{r}\mathrm{のピクセルがパターン外})
  \end{cases}
  \label{eq:density_distribution}
\end{equation}
となる量である。
\begin{figure}[htbp]
  \centering
  \includegraphics[width=0.4\textwidth]{../../figure/part3/densitiy_func_dim.png}
  \caption{パターンの密度分布。赤色がパターンに含まれるピクセル。赤色のピクセルでは$\rho(\bm{r})=1$, 灰色のピクセルでは$\rho(\bm{r})=0$となる。}
  \label{fig:density_func_dif}
\end{figure}

パターンの密度相関関数$C(\bm{r})$は
\begin{equation}
  C(r)=\frac{1}{\Omega_t}\int \d\Omega \frac{1}{N}\sum_{{\bm{r'}}}\rho(\bm{r'+ r})\rho({\bm{r'}})
  \label{eq:density_correlation}
\end{equation}
と定義される。$\Omega$ は$\bm{r}$のなす立体角,$\Omega_t$ は全立体角,$N$ は粒子数である。空間次元$d$に対して,
\begin{equation}
  \begin{aligned}
    \Omega   & =\prod^{d-1}_{i=1} (\sin\theta_i)^{d-i-1} \d{\theta_1}...\d{\theta_{d-1}} \qquad(0\leqq \theta_i\leqq \pi\,,\,0\leqq\theta_{d-1}<2\pi) \\
    \Omega_t & =2^{d-1}\pi
  \end{aligned}
  \label{eq:omega}
\end{equation}
で与えられる。興味があるのは$N$の$r$に対するスケーリング則のため,定数分は除いて考える。$C(r)$はスケールフリー性を持つので,$C(r)\propto r^{-a}$と表せる(スケールフリー性と冪関数については付録\ref{sec:scale_free}参照)。$C(r)$はある粒子から距離$r$離れた際に粒子が存在する確率を表しているので,微小範囲の粒子数は$\d N=C(r)\d \bm{r}\propto C(r)r^{d-1}\d r\propto r^{d-a-1} \d r $とスケーリングできる。全範囲にわたって積分すればそのパターンの大きさ,したがって粒子数$N$が導出できる。回転半径程度がパターンのおおよその大きさで,回転半径以上では$C(r)$はほぼ0とみなせるため,粒子数$N$は,
\begin{equation}
  N\propto \int_{0}^{R_g} r^{d-a-1} \d r\propto r^{d-a}
  \label{eq:density_correlation_fractal}
\end{equation}
と表される。

本節の議論より,密度相関関数法の両対数プロットより$a$の値を求められれば,パターンのフラクタル次元$D_f$を求めることができる。式\ref{eq:gyration_radius_fractal}と式\ref{eq:density_correlation_fractal}より,$D_f=d-a$となる。

% \section{研究目的}
% 枝分かれパターンの形成メカニズムは現象によって異なる。\textcolor{red}{しかし,枝の分岐(Tip Splitting)は成長過程での界面の粗さが影響を与える\cite{}ことが知られており,}界面の粗さの不安定性が最終的な枝分かれ形状に影響を与えると思われる。しかし,成長界面の粗さの不安定性と最終的な枝分かれ形状の関係は十分に理解されていない。本研究では,成長界面の粗さの不安定性と枝分かれ形状の関係を明らかにすることを目的とする。モデル系として,亜鉛イオン水溶液の電界析出で生じる亜鉛の金属樹を用いる。亜鉛の金属樹はフラクタル性を示すことが知られており\cite{matsushita1984fractal},様々なパラメータでの形態変化が調べられている\cite{suda2003temperature}。本研究では電解質溶液に界面活性剤を加え,最終的に生じる樹枝状パターンがどのように変化するかを議論するとともに,パターン変化の原因であると思われる,電界析出する金属結晶の界面の粗さを平滑化させるleveling現象を検証した。また,枝分かれパターンのフラクタル性を再現するモデルである拡散律速凝集(Diffusion Limited Aggregation: DLA)モデル\cite{witten1981diffusion}をベースに,実験を再現するモデルの構築を行った。

\ifdraft{
  \bibliographystyle{../../Preamble/Physics.bst}
  \bibliography{../../Preamble/reference.bib}
}{}

\end{document}
% \documentclass[autodetect-engine,dvi=dvipdfmx,a4paper,ja=standard,oneside,openany,11pt,draft]{bxjsbook}
\documentclass[autodetect-engine,dvi=dvipdfmx,a4paper,ja=standard,oneside,openany,11pt,draft]{bxjsarticle}
\usepackage{../../Preamble/mypackage}

\begin{document}
\part{実験}
\section{金属樹}
\section{電界析出}
\section{界面成長}

\ifdraft{
  \bibliographystyle{../Preamble/Physics.bst}
  \bibliography{../Preamble/reference.bib}
}{}
\end{document}
% \documentclass[autodetect-engine,dvi=dvipdfmx,a4paper,ja=standard,oneside,openany,11pt,draft]{bxjsbook}
\documentclass[autodetect-engine,dvi=dvipdfmx,a4paper,ja=standard,oneside,openany,11pt,draft]{bxjsarticle}
\usepackage{../../Preamble/mypackage}

\begin{document}
\part{実験}
\section{金属樹}
\section{電界析出}
\section{界面成長}

\ifdraft{
  \bibliographystyle{../Preamble/Physics.bst}
  \bibliography{../Preamble/reference.bib}
}{}
\end{document}
% \documentclass[autodetect-engine,dvi=dvipdfmx,a4paper,ja=standard,oneside,openany,11pt,draft]{bxjsbook}
\documentclass[autodetect-engine,dvi=dvipdfmx,a4paper,ja=standard,oneside,openany,11pt,draft]{bxjsarticle}
\usepackage{../../Preamble/mypackage}

\begin{document}

\chapter{数値計算}
\section{ランダムウォーク(RW)と拡散方程式}
\subsection{ランダムウォーク}
溶液中で粒子が拡散する際,その運動は\textbf{ランダムウォーク(Random Walk:RW)}でモデル化される。例えば空間次元$d=2$の場合,

\section{Langevin方程式}
\section{拡散律速凝集(Diffusion-limited aggregation, DLA)モデル}

\ifdraft{
  \bibliographystyle{../../Preamble/Physics.bst}
  \bibliography{../../Preamble/reference.bib}
}{}
\end{document}
%研究目的,後で分割すること
\section{研究目的}
枝分かれパターンの形成メカニズムは現象によって異なる。しかし,枝の分岐(Tip Splitting)は成長過程で界面が不安定化し粗くなることで分岐していくため,界面の粗さの不安定性が最終的な枝分かれ形状に影響を与えると思われる。しかし,成長界面の粗さの不安定性と最終的な枝分かれ形状の関係は十分に理解されていない。本研究では,成長界面の粗さの不安定性と枝分かれ形状の関係を明らかにすることを目的とする。モデル系として,亜鉛イオン水溶液の電界析出で生じる亜鉛の金属樹を用いる。亜鉛の金属樹はフラクタル性を示すことが知られており\cite{matsushita1984fractal},様々なパラメータでの形態変化が調べられている\cite{suda2003temperature}。本研究では電解質溶液に界面活性剤を加え,最終的に生じる樹枝状パターンがどのように変化するかを議論するとともに,パターン変化の原因であると思われる,電界析出する金属結晶の界面の粗さを平滑化させるleveling現象を検証した。また,枝分かれパターンのフラクタル性を再現するモデルである拡散律速凝集(Diffusion Limited Aggregation: DLA)モデル\cite{witten1981diffusion}をベースに,実験を再現するモデルの構築を行った。

% Part2-1 Experiment(dendrite)
% \documentclass[autodetect-engine,dvi=dvipdfmx,a4paper,ja=standard,oneside,openany,11pt,draft]{bxjsbook}
\documentclass[autodetect-engine,dvi=dvipdfmx,a4paper,ja=standard,oneside,openany,11pt,draft]{bxjsarticle}
\usepackage{../../Preamble/mypackage}

\begin{document}
\sectrion{金属樹}
\subsection{実験系}
\subsection{実験方法}
\subsection{解析方法}
\subsubsection{画像解析}
\subsubsection{フラクタル次元解析}
金属樹の枝分かれ構造を特徴づけるためにフラクタル次元の計測を行った。フラクタル次元の



\ifdraft{
  \bibliographystyle{../../Preamble/Physics.bst}
  \bibliography{../../Preamble/reference.bib}
}{}
\end{document}
% \documentclass[autodetect-engine,dvi=dvipdfmx,a4paper,ja=standard,oneside,openany,11pt,draft]{bxjsbook}
\documentclass[autodetect-engine,dvi=dvipdfmx,a4paper,ja=standard,oneside,openany,11pt,draft]{bxjsarticle}
\usepackage{../../Preamble/mypackage}

\begin{document}
\subsection{実験結果}
\subsubsection{フラクタル次元}

\subsubsection{枝の本数・太さ}
\begin{figure}[H]
  \begin{minipage}
    {0.5\textwidth}
    \centering
    \includegraphics[width=0.9\textwidth]{../../figure/part2(exp_deposition)/branch_num.png}
    \subcaption{枝の本数}
    \label{fig:branch_number}
  \end{minipage}
  \begin{minipage}
    {0.5\textwidth}
    \centering
    \includegraphics[width=0.9\textwidth]{../../figure/part2(exp_deposition)/branch_thickness_mean.png}
    \subcaption{枝の太さ}
    \label{fig:branch_thickness}
  \end{minipage}
  \caption{中心からの距離$r$の円と交わる枝の本数・太さ}
\end{figure}
\ref{fig:branch_number}は中心からの距離$r$の円と交わる枝の本数,\ref{fig:branch_thickness}は枝の太さを示している。図中黒破線(20本の線)をおおよそ境にして,
\begin{enumerate}
  \item 界面活性剤濃度0.03\%以上:枝の本数は20本を超えない。
  \item 界面活性剤濃度0.03\%未満:枝の本数は20本を超え,外側に連れて増加していく。
\end{enumerate}
という特徴が見られる。また,\ref{fig:branch_thickness}は枝の太さを示している。図中黒破線は太さ約$\SI{0.06}{cm}$を表しており,この線をおおよそ境にして,
\begin{enumerate}
  \item 界面活性剤濃度0.03\%以上:枝の太さは緩やかに大きくなり,ばらつきも大きくなる。
  \item 界面活性剤濃度0.03\%未満:枝の太さはおおむね$\SI{0.06}{cm}$以下であり,太さのばらつきも小さい。
\end{enumerate}
\subsubsection{枝の長さ,分岐角度}

\ifdraft{
  \bibliographystyle{../../Preamble/Physics.bst}
  \bibliography{../../Preamble/reference.bib}
}{}
\end{document}
\documentclass[autodetect-engine,dvi=dvipdfmx,a4paper,ja=standard,oneside,openany,11pt]{bxjsbook}
\usepackage{../../Preamble/mypackage}

\begin{document}
\section{議論・考察}
界面活性剤の濃度を増加させて金属樹を生成すると,濃度の増加と共に,傾向としてフラクタル次元が減少する結果が得られた(図\ref{fig:fractal_dim})。この結果は金属樹の枝分かれが少なくなり(図\ref{fig:branch}\subref{fig:branch_number}),パターンが疎になることで一次元形状(直線の枝)の割合が増えたためだと考えられる。

また,図\ref{fig:angle}, \ref{fig:branch_length}, \ref{fig:branch_length_edited}より,分岐角度は界面活性剤の濃度に依存しない一方で,枝の長さは濃度が上がるほど,長い枝が出現しやすくなることが示唆された。界面活性剤の添加によって,\ref{sec:tip_splitting}節でも言及したような,析出界面における界面張力の増加や,イオン流束の減少による過飽和度の減少などにより,MS不安定性における波長や析出速度が変化したためと考えられる。

このことは図\ref{fig:branch_length_exp}において,界面活性剤濃度が上昇すると確率密度関数$f(x)=ax^{-b}$の指数$b$が傾向として減少することからも裏付けられる。指数$b$の減少は,図\ref{fig:pow_b_func}のように確率密度関数の値が全体的に上昇し,ある範囲の値(図\ref{fig:pow_b_func}では$2<x<3$の網掛け部の面積)が出る確率が上昇することを示している。したがって,界面活性剤の添加によって,長い枝がより出現しやすくなる傾向がある。ただし,図\ref{fig:branch_length_exp}\subref{fig:exp_b_branch_len}において,$\SI{0.005}{\mathrm{vol}\%}$は傾向から外れた低い値を取っていたが,これは図\ref{fig:branch_length},\ref{fig:branch_length_edited}で言及した,低濃度側で見られる極端に長い枝の影響である。また,図\ref{fig:branch_length_exp}\subref{fig:exp_b_branch_edited_len}において$\SI{0.005}{\mathrm{vol}\%}$の値が普通の枝だけの時に比べて相対的に大きい値となっているのは,$\SI{0.03}{\mathrm{vol}\%}$以上の濃度での幹の割合が増えたためである。

\begin{figure}[htbp]
  \centering
  \includegraphics[width=0.75\textwidth]{../../figure/part2(exp_deposition)/pow_b_func.png}
  \caption{確率密度関数(冪関数)と指数の大小の関係。}
  \label{fig:pow_b_func}
\end{figure}

\ifdraft{
  \bibliographystyle{../../Preamble/Physics.bst}
  \bibliography{../../Preamble/reference.bib}
}{}
\end{document}

% Part2-2 Experiment(surface)
% \documentclass[autodetect-engine,dvi=dvipdfmx,a4paper,ja=standard,oneside,openany,11pt,draft]{bxjsbook}
\documentclass[autodetect-engine,dvi=dvipdfmx,a4paper,ja=standard,oneside,openany,11pt,draft]{bxjsarticle}
\usepackage{../../Preamble/mypackage}

\begin{document}
\sectrion{金属樹}
\subsection{実験系}
\subsection{実験方法}
\subsection{解析方法}
\subsubsection{画像解析}
\subsubsection{フラクタル次元解析}
金属樹の枝分かれ構造を特徴づけるためにフラクタル次元の計測を行った。フラクタル次元の



\ifdraft{
  \bibliographystyle{../../Preamble/Physics.bst}
  \bibliography{../../Preamble/reference.bib}
}{}
\end{document}
% \documentclass[autodetect-engine,dvi=dvipdfmx,a4paper,ja=standard,oneside,openany,11pt,draft]{bxjsbook}
\documentclass[autodetect-engine,dvi=dvipdfmx,a4paper,ja=standard,oneside,openany,11pt,draft]{bxjsarticle}
\usepackage{../../Preamble/mypackage}

\begin{document}
\subsection{実験結果}
\subsubsection{フラクタル次元}

\subsubsection{枝の本数・太さ}
\begin{figure}[H]
  \begin{minipage}
    {0.5\textwidth}
    \centering
    \includegraphics[width=0.9\textwidth]{../../figure/part2(exp_deposition)/branch_num.png}
    \subcaption{枝の本数}
    \label{fig:branch_number}
  \end{minipage}
  \begin{minipage}
    {0.5\textwidth}
    \centering
    \includegraphics[width=0.9\textwidth]{../../figure/part2(exp_deposition)/branch_thickness_mean.png}
    \subcaption{枝の太さ}
    \label{fig:branch_thickness}
  \end{minipage}
  \caption{中心からの距離$r$の円と交わる枝の本数・太さ}
\end{figure}
\ref{fig:branch_number}は中心からの距離$r$の円と交わる枝の本数,\ref{fig:branch_thickness}は枝の太さを示している。図中黒破線(20本の線)をおおよそ境にして,
\begin{enumerate}
  \item 界面活性剤濃度0.03\%以上:枝の本数は20本を超えない。
  \item 界面活性剤濃度0.03\%未満:枝の本数は20本を超え,外側に連れて増加していく。
\end{enumerate}
という特徴が見られる。また,\ref{fig:branch_thickness}は枝の太さを示している。図中黒破線は太さ約$\SI{0.06}{cm}$を表しており,この線をおおよそ境にして,
\begin{enumerate}
  \item 界面活性剤濃度0.03\%以上:枝の太さは緩やかに大きくなり,ばらつきも大きくなる。
  \item 界面活性剤濃度0.03\%未満:枝の太さはおおむね$\SI{0.06}{cm}$以下であり,太さのばらつきも小さい。
\end{enumerate}
\subsubsection{枝の長さ,分岐角度}

\ifdraft{
  \bibliographystyle{../../Preamble/Physics.bst}
  \bibliography{../../Preamble/reference.bib}
}{}
\end{document}
\documentclass[autodetect-engine,dvi=dvipdfmx,a4paper,ja=standard,oneside,openany,11pt]{bxjsbook}
\usepackage{../../Preamble/mypackage}

\begin{document}
\section{議論・考察}
界面活性剤の濃度を増加させて金属樹を生成すると,濃度の増加と共に,傾向としてフラクタル次元が減少する結果が得られた(図\ref{fig:fractal_dim})。この結果は金属樹の枝分かれが少なくなり(図\ref{fig:branch}\subref{fig:branch_number}),パターンが疎になることで一次元形状(直線の枝)の割合が増えたためだと考えられる。

また,図\ref{fig:angle}, \ref{fig:branch_length}, \ref{fig:branch_length_edited}より,分岐角度は界面活性剤の濃度に依存しない一方で,枝の長さは濃度が上がるほど,長い枝が出現しやすくなることが示唆された。界面活性剤の添加によって,\ref{sec:tip_splitting}節でも言及したような,析出界面における界面張力の増加や,イオン流束の減少による過飽和度の減少などにより,MS不安定性における波長や析出速度が変化したためと考えられる。

このことは図\ref{fig:branch_length_exp}において,界面活性剤濃度が上昇すると確率密度関数$f(x)=ax^{-b}$の指数$b$が傾向として減少することからも裏付けられる。指数$b$の減少は,図\ref{fig:pow_b_func}のように確率密度関数の値が全体的に上昇し,ある範囲の値(図\ref{fig:pow_b_func}では$2<x<3$の網掛け部の面積)が出る確率が上昇することを示している。したがって,界面活性剤の添加によって,長い枝がより出現しやすくなる傾向がある。ただし,図\ref{fig:branch_length_exp}\subref{fig:exp_b_branch_len}において,$\SI{0.005}{\mathrm{vol}\%}$は傾向から外れた低い値を取っていたが,これは図\ref{fig:branch_length},\ref{fig:branch_length_edited}で言及した,低濃度側で見られる極端に長い枝の影響である。また,図\ref{fig:branch_length_exp}\subref{fig:exp_b_branch_edited_len}において$\SI{0.005}{\mathrm{vol}\%}$の値が普通の枝だけの時に比べて相対的に大きい値となっているのは,$\SI{0.03}{\mathrm{vol}\%}$以上の濃度での幹の割合が増えたためである。

\begin{figure}[htbp]
  \centering
  \includegraphics[width=0.75\textwidth]{../../figure/part2(exp_deposition)/pow_b_func.png}
  \caption{確率密度関数(冪関数)と指数の大小の関係。}
  \label{fig:pow_b_func}
\end{figure}

\ifdraft{
  \bibliographystyle{../../Preamble/Physics.bst}
  \bibliography{../../Preamble/reference.bib}
}{}
\end{document}

% Part3 Simulation
\documentclass[autodetect-engine,dvi=dvipdfmx,a4paper,ja=standard,oneside,openany,11pt,draft]{bxjsbook}
% \documentclass[autodetect-engine,dvi=dvipdfmx,a4paper,ja=standard,oneside,openany,11pt,draft]{bxjsarticle}
\usepackage{../../Preamble/mypackage}

\begin{document}
\section{数値計算のモデル}
\subsection{RWの方向の決定方法}
\begin{figure}[htbp]
  \centering
  \includegraphics[width=0.5\textwidth]{../../figure/part3/RW_2dim.png}
  \caption{各方向へ移動する確率$p_i,q_i$と動かない確率$r_i$,電場$E_i$の模式図}
  \label{fig:RW_2dim}
\end{figure}

数値計算は二次元DLA($d=2$)をベースに,電場によるドリフトと界面活性剤による界面への固着の影響を取り込んだモデルを作成した。\ref{sec:RW}と\ref{sec:Langevin}の内容をもとに,RWとLangevin方程式の対応を考える。\ref{sec:RW}より,外力(電場)について,
\begin{equation}
  \bar{F}_i=\mu q E_i=(p_i-q_i)\frac{a}{\Delta t}
  \label{eq:force}
\end{equation}が成り立つので,$p_i-q_i=\mu q E_i\Delta t/a$となる。また,拡散係数$D=a^2/2d\Delta t$より,$a^2=2dD\Delta t$となる。次に,式\ref{eq:Langevin_overdamped_average},\ref{eq:Langevin_overdamped_variance}を離散化する。格子間隔$a$,$\Delta t$を用いると,以下のように離散化できる。
\begin{equation}
  \begin{split}
    \bm{x} & =a\bm{X}, \qquad t=n\Delta t, \qquad (n\in\mathbb{N},\bm{X}\in\mathbb{Z}^d,a,\Delta t \in \mathbb{R}) \\
    \label{eq:discretization}
  \end{split}
\end{equation}
1ステップの時間発展を考えると,式\ref{eq:discretization}で$n=1,X_i=\pm1$とすればよい。$\langle\cdot\rangle$を確率分布に関する平均とすれば,$\langle X_i\rangle=(+1)p_i+(-1)q_i=\mu q E_i\Delta t/a$,$\langle X_i^2\rangle=(+1)^2p_i+(-1)^2q_i=1/d$より以下が成り立つ。ただし,電場の平均操作については,1ステップの時間・空間スケールではほぼ一定であるとして,動く前での電場を用いている。
\begin{equation}
  \begin{aligned}
    \langle\bm{x}\rangle & =a\langle\bm{X}\rangle                    & \langle(\bm{x}(t)-\langle\bm{x}(t)\rangle)^2\rangle & =a^2\langle(\bm{X}-\langle\bm{X}\rangle)^2\rangle                        \\
                         & =a\sum_{i=1}^{d}\bm{e}_i(p_i-q_i)         &                                                     & =a^2\{\langle\bm{X}^2\rangle-\langle\bm{X}\rangle^2\}                    \\
                         & =\mu q \Delta t\sum_{i=1}{d} \bm{e}_i E_i &                                                     & =a^2\sum_{i=1}^{d}\ab\{\frac{1}{d}-\frac{(\mu q \Delta t)^2}{a^2}E_i^2\} \\
                         & =\mu q \Delta t\bm{E}                     &                                                     & =a^2\ab\{1-\frac{(\mu q \Delta t)^2}{a^2}\bm{E}^2\}                      \\
  \end{aligned}
  \label{eq:discrete_average_variance}
\end{equation}
式\ref{eq:discrete_average_variance}は素朴にRWとLangevin方程式を対応させたものである。しかし,分散の方は電場の強さに依存しており,\ref{sec:Langevin}で求めた,分散が外力に依存しないという結果と異なる。このままだと,場所によって分散が異なる,つまり場所によって温度が異なるという物理的に不可解な状況になってしまう。そこで,RWにおいて図\ref{fig:RW_2dim}のように\textbf{その場にとどまる確率 $r$}を導入し,電場の影響を打ち消すように調整した。物理的には電場と運動が拮抗し,動けない場合を想定している。まず,確率の保存則より,$r+\sum_{i=1}^{d}(p_i+q_i) =1$が成り立つ。そのため,第$i$成分に対して$p_i+q_i=(1-r)/d$となる。以降,$\bm{X}$空間での平均と,分散を,$\langle X_i\rangle=2\alpha E_i$,$\langle\bm{X}^2\rangle-\langle\bm{X}\rangle^2=2dC$と置く。ここで,$\alpha=\mu q\Delta t/2a$であり電場に対する応答の大きさを表す量である。式\ref{eq:discrete_average_variance}より,分散は以下のように修正される。
\begin{equation}
  \begin{split}
    \langle(\bm{x}(t)-\langle\bm{x}(t)\rangle)^2\rangle & =a^2\ab\{(1-r)-\frac{(\mu q \Delta t)^2}{a^2}\bm{E}^2\} \\
                                                        & =a^2\ab\{(1-r)-4\alpha^2\bm{E}^2\}                      \\
                                                        & =a^2 2dC                                                \\
                                                        & =2dD\Delta t
  \end{split}
  \label{eq:discrete_variance}
\end{equation}
式\ref{eq:discrete_variance}より,$C=D\Delta t/a^2$,$r=1-2dC-4\alpha^2\bm{E}^2$となる。ここで注意しなければならないのは,ここまでの議論が成り立つためには,少なくとも$r>0$とならなければならない点である。$a^2=2dD\Delta t$のままであれば,$C=1/2d$より,$r=-4\alpha^2\bm{E}^2<0$となってしまう。そのため,格子間隔$a$を固定すれば,時間間隔$\Delta t$を変化させることで$r>0$を確保することができる。以上の議論をまとめると,RWにおいて電場の影響を打ち消す確率$r$を導入することで,Langevin方程式の分散が外力に依存しないという結果を再現することができる。しかし,$r>0$を担保するため,時間間隔$\Delta t$を変化させなければならない。これはパラメータである$\alpha=\mu q\Delta t/2a,C=D\Delta t/a^2$を変えることに他ならない。今回の数値計算では分散$C$を固定して,$\alpha$を変化させて形状変化を調べた。$C$の値は以下のように決めた。まず,各$i$成分の確率は以下の式を満たす。

\begin{equation}
  \left\{
  \begin{aligned}
    p_i+q_i & =2C+\frac{4\alpha^2}{d}\bm{E}^2 \\
    p_i-q_i & =2\alpha E_i
  \end{aligned}
  \right.
  \label{eq:prob}
\end{equation}
これより,各確率$p_i,q_i,r$は,
\begin{equation}
  \left\{
  \begin{aligned}
    p_i & =C+\alpha E_i+\frac{2\alpha^2}{d}\bm{E}^2 \\
    q_i & =C-\alpha E_i+\frac{2\alpha^2}{d}\bm{E}^2 \\
    r   & =1-2dC-4\alpha^2\bm{E}^2
  \end{aligned}
  \right.
  \label{eq:prob2}
\end{equation}
\textcolor{red}{$\bm{E}^2/d$を平均的な電場の強さとして,各方向に分割すると,}
\begin{equation}
  \left\{
  \begin{aligned}
    p_i & =C+\alpha E_i+2\alpha^2 E_i^2 \\
    q_i & =C-\alpha E_i+2\alpha^2 E_i^2 \\
    r   & =1-2dC-4\alpha^2\bm{E}^2
  \end{aligned}
  \right.
  \label{eq:prob3}
\end{equation}
$p_i,q_i$を$\alpha E\i$について平方完成すると,$\alpha E_i=\mp1/4$で最小値$C-1/8$を取る。また,$r$は$\alpha E_i=0$で最大値$1-2dC$を取る。これら二つの値が0と1の間に入るという条件は$d=2$の時,
\begin{align}
  0<C-\frac{1}{8}<1 &  & \Leftrightarrow &  & \frac{1}{8}<C<\frac{9}{8} \\
  0<1-4C<1          &  & \Leftrightarrow &  & 0<C<\frac{1}{4}
  \label{eq:condition}
\end{align}
以上より,$C$の許される範囲は
\begin{equation}
  \frac{1}{8}<C<\frac{1}{4}
  \label{eq:condition2}
\end{equation}
となる。これより,上下端の中央の値である$C=3/16$を数値計算に用いる値とした。
\subsection{固着確率$P$の定義}
実験における,界面でのlevelerによるイオンの析出阻害を再現するため,ブラウン運動してきた粒子を\textbf{クラスターに取り込む確率$P$}を導入した。本来のDLAでは,粒子がすでにある隣のクラスターに来た場合,そのまま新たなクラスターとして取り込むが,今回の数値計算では確率$P$でクラスターとして取り込んだ。取り込まれなかった場合は再びRWを行う。
\subsection{数値計算方法}
今回の数値計算では,パターンにより生じる電場と組み合わせてブラウン運動する粒子の移動確率を計算した。まず,計算する系の条件を以下のように設定する。
\begin{table}[htbp]
  \centering
  \caption{数値計算の系の条件 }
  \begin{tabular}{|c||c|}
    \hline
    系の条件                      & 設定値                                                  \\ \hline\hline
    系のサイズ $L$                 & $\SI{512}{pix}\times\SI{512}{pix}$                   \\ \hline
    粒子数 $N$                   & 15000                                                \\ \hline
    中心座標 $\bm{x_c}=(x_c,y_c)$ & $(x_c,y_c)=(L/2,L/2)=(256,256)$                      \\ \hline
    境界条件                      & $||\bm{r}-\bm{x_c}||>L/2$で電位$V=1.0$,パターンが存在する点で$V=0$ \\ \hline
    SOR法による電場の収束条件            & 前回との誤差$10^{-5}$未満                                    \\ \hline
    初期条件                      & $\bm{x_c}$に粒子を一つ置く                                   \\ \hline
    電場の計算タイミング                & 150粒子毎                                               \\ \hline
  \end{tabular}
\end{table}

具体的な計算方法は以下のとおりである。ただし,電場の計算においては,金属樹の電解質溶液の電気的中性が期待されるため,ラプラス方程式$\nabla^2\phi=0$を解いた。
\begin{enumerate}
  \item 現在のパターンの内,中心からの距離が最も遠い点を探索する
  \item その点から30 pix離れた円上にランダムに粒子を一つ置く
  \item 式\ref{eq:prob3}に従って粒子の移動方向を決定する。電位が1.0の領域(中心からの距離が$L/2$の円よりも外側)に来たらその粒子を棄却し,新たな粒子を2.に従って置く。
  \item クラスターの隣の位置に来たら,与えた固着確率$P$でクラスターとして取り込むか判定する
  \item $150$粒子取り込むたびに,SOR法\textcolor{red}{付録にて}を用いて電位のラプラス方程式を解き,電場を計算した。
  \item 以上を15000粒子分繰り返す
\end{enumerate}
この計算を,$\alpha=0.0$から$3.0$まで($0.0~1.0$まで$0.1$刻み,$1.0~3.0$までは0.2刻み),$P=0.1,0.4,0.7,1.0$の4つの固着確率について,各条件について21個のデータを取り,フラクタル次元を計測した。$\alpha=2.6$では計算が収束しなかったため,$\alpha\leq2.4$までのデータを用いた。原因については不明だが,収束する場合としない場合があったことより,電場の影響が大きくなりすぎ,ランダムな運動よりも電場に従った弾道的な運動になったことで,たまたま中心付近を通るものしかクラスターに取り込まれなくなり,パターンの成長速度が極端に遅くなったためと考えられる。
\begin{figure}[htbp]
  \begin{minipage}{0.32\hsize}
    \subcaption{}
    \centering
    \includegraphics[width=0.8\textwidth]{../../figure/part3/DLA_alpha=0_P=1.png}
    \label{fig:DLA_alpha_0_P_1}
  \end{minipage}
  \begin{minipage}{0.32\hsize}
    \subcaption{}
    \centering
    \includegraphics[width=0.8\textwidth]{../../figure/part3/DLA_phi_alpha=0_P=1.png}
    \label{fig:DLA_phi_alpha_0_P_1}
  \end{minipage}
  \begin{minipage}{0.32\hsize}
    \subcaption{}
    \centering
    \includegraphics[width=0.8\textwidth]{../../figure/part3/DLA_phi_alpha=0_P=1.png}
    \label{fig:DLA_E_alpha_0_P_1}
  \end{minipage}
  \caption{$C=3/16$,パラメータ$\alpha=0.0$固着確率$P=1.0$(通常のDLA)の例。\subref{fig:DLA_alpha_0_P_1}はクラスターの形状,\subref{fig:DLA_phi_alpha_0_P_1}は電位分布,\subref{fig:DLA_E_alpha_0_P_1}は電場の強度分布を示す。}
  \label{fig:DLA_ex}
\end{figure}
\subsection{解析方法}
数値計算は$\alpha$は0.0~3.0(,$P$は0.1,0.4.,0.7,1.0の4つについて計算を行各条件に付き21個のデータについて解析は各固着確率,$\alpha$に対して,フラクタル次元を計測することで行った。フラクタル次元は密度相関関数法で計測した。密度相関関数を計算し,粒子間距離が$2~R_g$以内($R_g$は回転半径)の範囲でフィッティングを行った。

\ifdraft{
  \bibliographystyle{../../Preamble/Physics.bst}
  \bibliography{../../Preamble/reference.bib}
}{}
\end{document}
\documentclass[autodetect-engine,dvi=dvipdfmx,a4paper,ja=standard,oneside,openany]{bxjsbook}
\usepackage{../../Preamble/mypackage}

\begin{document}
\section{数値計算結果}
\subsection{パターンの変化}

\begin{figure}[htbp]
  \centering
  \includegraphics[width=0.8\textwidth]{../../figure/part3/sim_result.png}
  \caption{数値計算によるパターンの結果。典型的なものを示す。横軸は応答の大きさ$\alpha$,縦軸は固着確率$P$である。}
  \label{fig:sim_result}
\end{figure}

図\ref{fig:sim_result}は数値計算の結果を示している。横軸は応答の大きさ$\alpha$,縦軸は固着確率$P$である。$\alpha$が増加してもはっきりとした違いは見られなかった。また,固着確率$P$の減少にともなって,$\alpha$の値によらずパターンの枝が太くなった。
\subsection{パターンの回転半径,フラクタル次元$D_f$の変化}

\begin{figure}[htbp]
  \centering
  \includegraphics[width=0.8\textwidth]{../../figure/part3/R_g_result.jpg}
  \caption{数値計算によるパターンの回転半径$R_g$の変化。横軸は応答の大きさ$\alpha$,縦軸は回転半径$R_g$である。}
  \label{fig:R_g_result}
\end{figure}

\begin{figure}
  \centering
  \includegraphics[width=0.6\textwidth]{../../figure/part3/fractal_dim_result.jpg}
  \caption{数値計算によるパターンのフラクタル次元$D_f$の変化。横軸は応答の大きさ$\alpha$,縦軸はフラクタル次元$D_f$である。}
  \label{fig:fractal_dim_result}
\end{figure}

図\ref{fig:R_g_result}は電場への応答$\alpha$と固着確率$P$を変化させたときの,回転半径の結果を示している。横軸は応答の大きさ$\alpha$,縦軸は回転半径$R_g$である。図\ref{fig:R_g_result}より,回転半径は$P>0.10$では$\alpha$の増加に対してほぼ単調に減少していた。逆に$P=0.10$ではほぼ一定値を取っていた。また,固着確率が減少するほど,回転半径の大きさも減少し,$\alpha$に対する減少割合は緩やかになった。特に$\alpha\lessapprox1.0$で顕著であった。また,固着確率が線形に減少(0.3ずつ)しているにもかかわらず,回転半径は線形に減少しておらず,$P=0.10$と$P=0.40$の間に大きなとびがあった。

図\ref{fig:fractal_dim_result}は電場への応答$\alpha$と固着確率$P$を変化させたときの,フラクタル次元の結果を示している。横軸は応答の大きさ$\alpha$,縦軸はフラクタル次元$D_f$である。フラクタル次元は,$P>0.10$では$\alpha$の増加に対してほぼ単調に増加しており,パターンが密になっていった。$P=0.10$ではほぼ一定値を取っていた。また,固着確率が減少するほど,フラクタル次元は2に近づき,$\alpha$に対する増加割合は,回転半径の場合と同様に緩やかになった。特に$\alpha\lessapprox1.0$で顕著であった。また,固着確率が線形に減少(0.3ずつ)しているにもかかわらず,フラクタル次元$D_f$は線形に増加しておらず,回転半径と同様に$P=0.10$と$P=0.40$の間に大きなとびがあった。
\ifdraft{
  \bibliographystyle{../../Preamble/Physics.bst}
  \bibliography{../../Preamble/reference.bib}
}{}
\end{document}
\documentclass[autodetect-engine,dvi=dvipdfmx,a4paper,ja=standard,oneside,openany,11pt,draft]{bxjsbook}
\usepackage{../../Preamble/mypackage}

\begin{document}
\section{議論・考察}
結果をまとめると,次のようになる。
\begin{itemize}
  \item パターンの見た目は電場への応答の大きさ$\alpha$を大きくしてもさほど変化は見られないが,固着確率$P$を増加させるとパターンの枝が太くなることが確認された。
  \item 回転半径について,$\alpha$を増加させるとほぼ単調減少し,固着確率の減少により回転半径も減少していた。また,変化の割合も緩やかになっていた。
  \item フラクタル次元については$\alpha$を増加させるとほぼ単調増加し,固着確率の減少によりフラクタル次元は2に近づいていた。
\end{itemize}

まず,回転半径の変化とフラクタル次元の変化が逆の傾向を示している。回転半径がパターンの平均的な大きさを示していることより,粒子数が一定ならば,回転半径の減少でパターンが密になることは自明であり,その結果としてフラクタル次元が増加することも自明である。そのため,回転半径とフラクタル次元は逆の傾向を示す。

また,電場への応答$\alpha$の増加により,粒子がより電場の強い点,つまり,より細かい構造が多い部分に集まりやすくなるため,細かい構造が優先的に粒子で埋められることで密なパターンになったと考えられる。クラスター形成の初期段階から細かい構造が埋められることで枝の広がりが抑制され,枝の遮蔽効果が弱くなり,内側まで粒子が入り込みやすくなった影響も考えられる。

今回の数値計算からは,$\alpha$の影響に比べて固着確率$P$の影響が顕著であり,見た目の変化がはっきりしていた点も含めて,パターン形成において固着確率の影響が大きいことが示唆された。また,固着確率の低下は拡散律速凝集から界面での反応が律速段階になる凝集過程である反応律速凝集(Reaction Limited Aggregation:RLA)への遷移を意味するが,$P=0.10$と$P>0.10$との間にとびがあることから,この遷移は連続的でないことが示唆された。

また,実験結果と異なり,フラクタル次元の低下は再現できなかった。電場への応答に比べて固着確率の影響が大きかったことより,固着確率の計算方法を工夫することが必要である。あるいは,粒子の数を増やして,より大きなパターンにすることで,太い枝を持つ,フラクタル次元の低い樹枝状パターンになる可能性もある。

今回の数値計算はRWを用いた離散的なモデルだが,添加物の拡散や粒子の拡散を考慮した連続体モデルに拡張することが考えられる。計算の複雑さや計算量が増えてしまうが,界面形状を連続体として考えることで,実験により近い結果が得られる可能性が高い。

\ifdraft{
  \bibliographystyle{../../Preamble/Physics.bst}
  \bibliography{../../Preamble/reference.bib}
}{}
\end{document}

% Part4 Conclusion to Appendix
% \documentclass[autodetect-engine,dvi=dvipdfmx,a4paper,ja=standard,oneside,openany,11pt,draft]{bxjsbook}
\documentclass[autodetect-engine,dvi=dvipdfmx,a4paper,ja=standard,oneside,openany,11pt,draft,textwidth=50zw]{bxjsbook}
\usepackage{../../Preamble/mypackage}

\begin{document}
\chapter{結論}
\ifdraft{
  \bibliographystyle{../../Preamble/Physics.bst}
  \bibliography{../../Preamble/reference.bib}
}{}
\end{document}
% \documentclass[autodetect-engine,dvi=dvipdfmx,a4paper,ja=standard,oneside,openany,11pt,draft]{bxjsbook}
\documentclass[autodetect-engine,dvi=dvipdfmx,a4paper,ja=standard,oneside,openany,11pt,draft]{bxjsarticle}
\usepackage{../../Preamble/mypackage}

\begin{document}
\lipsum[1-5]

\ifdraft{
  \bibliographystyle{../Preamble/Physics.bst}
  \bibliography{../Preamble/reference.bib}
}{}
\end{document}
\documentclass[autodetect-engine,dvi=dvipdfmx,a4paper,ja=standard,oneside,openany,11pt,draft]{bxjsbook}
\usepackage{../../Preamble/mypackage}


\begin{document}
\appendix
\chapter{}
\section{スケールフリー性を持つ関数がべき関数になることの証明}
\label{sec:scale_free}
スケールフリー関数はスケール$a$のみに依存する関数$g(a)$を用いて以下のように表される。
\begin{equation}
  \frac{f(ax)}{f(x)}=g(a) \quad (a>0)
\end{equation}
これを両辺$a$で微分し,変形していく。
\begin{equation}
  \begin{split}
    \frac{\d (ax)}{\d a}\frac{\d f(ax)}{\d (ax)}          & =\frac{\d g(a)}{\d a}f(x)                                                       \\
    \lim_{a\to 1}x\frac{\d f(ax)}{\d (ax)}                & =D_f f(x) \qquad\left(\left.\frac{\d f(ax)}{\d (ax)}\right|_{a\to1}:=D_f\right) \\
    \frac{1}{f(x)}\frac{\d f(x)}{\d x}                    & =\frac{D_f}{x}                                                                  \\
    \int_{x_0}^{x} \frac{1}{f(x)}\frac{\d f(x)}{\d x}\d x & =\int_{x_0}^{x} \frac{D_f}{x}\d x                                               \\
    \log f(x)-\log f(x_0)                                 & =D_f\log x-D_f\log x_0                                                          \\
    \log \frac{f(x)}{f(x_0)}                              & =D_f\log \frac{x}{x_0}                                                          \\
    f(x)                                                  & =f(x_0)\left(\frac{x}{x_0}\right)^{D_f}
  \end{split}
\end{equation}
以上よりスケールフリー性を持つ関数はべき関数になることが示された。
\chapter{}
\section{MS不安定性の詳細な計算}
\chapter{}
\section{ランダムウォーク(RW)からの拡散方程式の導出}
1ステップ後の粒子の濃度についての式
\begin{equation}
  c(\bm{x},t+\Delta t)=\sum_{i=1}^{d}\left[p_i c(\bm{x}-a\bm{e}_i,t)+q_i c(\bm{x}+a\bm{e}_i,t)\right]
  \label{eq:RW}
\end{equation}
を左辺は一次,右辺は二次までTaylor展開する。
\begin{equation}
  \begin{split}
    \mathrm{(左辺)} & =c(\bm{x},t)+\pdv{c(\bm{x},t)}{t}\Delta t+\mathcal{O}((\Delta t)^2)                                                                                                                 \\
    \mathrm{(右辺)} & =\sum_{i=1}^{d}\left[\underset{=1/d}{\uwave{(p_i+q_i)}}c(\bm{x},t)-(p_i-q_i)\pdv{c(\bm{x},t)}{x_i}a+\underset{=1/d}{\uwave{(p_i+q_i)}}\pdv[2]{c(\bm{x},t)}{x_i}\frac{a^2}{2}\right] \\
                  & =c(\bm{x},t)+\sum_{i=1}^{d}\left[-(p_i-q_i)\pdv{c(\bm{x},t)}{x_i}a+\pdv[2]{c(\bm{x},t)}{x_i}\frac{a^2}{2d}\right]+\mathcal{O}(a^3)
  \end{split}
  \label{eq:RW_taylor}
\end{equation}
以上の結果から,以下の式が成り立つ。
\begin{equation}
  \pdv{c(\bm{x},t)}{t}=-\sum_{i=1}^{d}(p_i-q_i)\frac{a}{\Delta t}\pdv{c(\bm{x},t)}{x_i}+\frac{a^2}{2d\Delta t}\sum_{i=1}^{d}\pdv[2]{c(\bm{x},t)}{x_i}
  \label{eq:RW_diffusion}
\end{equation}

\section{過減衰Langenvin方程式の分散の詳細な計算}
運動方程式
\begin{equation}
  \odv{\bm{x}}{t}=\mu q\bm{E}+\mu\bm{\xi}(t)
  \label{eq:Langevin_overdamped}
\end{equation}
より,時刻$t$での位置$\bm{x}(t)$は0から$t$までの積分を行い,以下のように表される。ただしランダム力は以下の関係式を満たす。
\begin{equation}
  \begin{split}
    \langle\bm{\xi}(t)\rangle                  & =0                                                   \\
    \langle\bm{\xi}(t)\cdot\bm{\xi}(t')\rangle & =2d\gamma k_B T\delta(t-t') \qquad (d:\mathrm{空間次元})
  \end{split}
  \label{eq:random_force}
\end{equation}
よって,時刻$t$での位置$\bm{x}(t)$は以下のように表される。
\begin{equation}
  \bm{x}(t)=\bm{x}(0)+\mu q\bm{E}t+\mu\int_0^t\bm{\xi}(t')\d t'
  \label{eq:Langevin_overdamped_integrated}
\end{equation}
式\ref{eq:Langevin_overdamped_integrated}の両辺の二乗を取り,平均を取ると,以下のようになる。ただし,初期位置$\bm{x}(0)=0$とした。
\begin{equation}
  \begin{split}
    \langle\bm{x}(t)^2\rangle & =(\mu q \bm{E} t)^2+2\mu^2 q t\ab\langle\int_{0}^{t}\bm{E}(\bm{x})\cdot\bm{\xi}(t') \d t'\rangle+\mu^2\ab\langle\int_{0}^{t}\int_{0}^{t}\bm{\xi}(t')\cdot
    \bm{\xi}(t'') \d t' \d t''\rangle                                                                                                                                                                                                                            \\
                              & =(\mu q \bm{E} t)^2+2\mu^2 q t\ab\int_{0}^{t}\bm{E}(\bm{x})\cdot\underset{=0}{\uwave{\langle\bm{\xi}(t')\rangle}} \d t'+\mu^2\ab\int_{0}^{t}\int_{0}^{t}\underset{=2d\gamma k_B T\delta(t'-t'')}{\uwave{\langle\bm{\xi}(t')\cdot
    \bm{\xi}(t'')\rangle}} \d t' \d t''                                                                                                                                                                                                                          \\
                              & =(\mu q \bm{E} t)^2+2d\mu k_B T t                                                                                                                                                                                                \\
  \end{split}
  \label{eq:Langevin_overdamped_average}
\end{equation}

\section{逐次加速緩和法(Successive Over Relaxation: SOR)の概要}
\label{sec:SOR}
\textbf{SOR法}はPoisson方程式の数値解法である\textbf{Gauss-Seidel法}を改良したものである。まず,Poisson方程式
\begin{equation}
  \nabla^2\phi(\bm{x})=-\rho(\bm{x})
  \label{eq:poisson}
\end{equation}
を数値的に解くために,離散化した方程式を考える。ここで,$\bm{x}=(x,y)$は二次元空間内の位置ベクトル,$\phi(\bm{x})$はポテンシャル,$\rho(\bm{x})$は適当な関数である。二次元空間を格子点で離散化し,格子点$(i,j)$におけるLaplasianは以下のように導かれる。格子間隔$\Delta x=\Delta y=h$とすると,
\begin{equation}
  \left\{
  \begin{aligned}
    \phi_{i\pm1,j} & =\phi_{i,j}\pm \pdv{\phi}{x}h+\pdv[2]{\phi}{x}\frac{h^2}{2}+\mathcal{O}(h^3) \\
    \phi_{i,j\pm1} & =\phi_{i,j}\pm \pdv{\phi}{y}h+\pdv[2]{\phi}{y}\frac{h^2}{2}+\mathcal{O}(h^3)
  \end{aligned}
  \right.
  \label{eq:discrete_laplasian}
\end{equation}
これより,Laplasianは以下のように離散化される。
\begin{equation}
  \begin{split}
    (\nabla^2\phi)_{i,j} & =\left\{\left(\pdv*[2]{}{x}+\pdv*[2]{}{y}\right)\phi\right\}_{i,j}                                            \\
                         & =\frac{1}{h^2}\left\{\phi_{i+1,j}+\phi_{i-1,j}+\phi_{i,j+1}+\phi_{i,j-1}-4\phi_{i,j}\right\}+\mathcal{O}(h^3) \\
                         & =-\rho_{i,j}
  \end{split}
  \label{eq:discrete_poisson}
\end{equation}
この式を以下のように更新して,前回ステップとの差が規定値以下になるまで計算していく。ここで$n$は計算ステップである。
\begin{equation}
  \phi_{i,j}^{n+1}=\frac{1}{4}\left(\phi_{i+1,j}^{n}+\phi_{i-1,j}^{n+1}+\phi_{i,j+1}^{n}+\phi_{i,j-1}^{n+1}+h^2\rho_{i,j}\right)
  \label{eq:Gauss-Seidel}
\end{equation}
式\ref{eq:Gauss-Seidel}のようなラプラス方程式の計算方法を\textbf{Gauss-Seidel法}という。ここに加速パラメータ$r$を導入し,以下のように更新式を変更する。
\begin{equation}
  \phi_{i,j}^{n+1}=(1-r)\phi_{i,j}^{n}+\frac{r}{4}\left(\phi_{i+1,j}^{n}+\phi_{i-1,j}^{n+1}+\phi_{i,j+1}^{n}+\phi_{i,j-1}^{n+1}+h^2\rho_{i,j}\right)
  \label{eq:SOR}
\end{equation}
式\ref{eq:SOR}の更新方法を\textbf{SOR法}という。SOR法はGauss-Seidel法よりも収束が速いことが知られている。実際,電場の計算にかかった実時間は,Gaiss-Seidel法に対しておおよそ1/10程度になっていた。

加速パラメータ$r$の値によって,効率が変化する。
\begin{itemize}
  \item $0<r<1$のとき,\textbf{under-relax}と呼ばれ,収束が速くなる。
  \item $r=1$のとき,Gauss-Seidel法と等価である。
  \item $1<r<2$のとき,\textbf{over-relax}と呼ばれ,収束が早くなる。
  \item $2<r$のとき,収束しない。
\end{itemize}
十分大きな系のサイズ$N$に対して,最適な$r=r_{\mathrm{optimaze}}$は,
\begin{equation}
  r_{\mathrm{optimaze}}=\frac{2}{1+\pi/N}
  \label{eq:optimaze_r}
\end{equation}
で与えられる。$r=r_{\mathrm{optimaze}}$の時,計算量はGauss-Seidel法が$\mathcal{O}(N^2)$なのに対して, SOR法は(厳密に)$2N$である。Gauss-Seidel法およびSOR法については文献\cite{hinch2020numerical}を参考にした。

\ifdraft{
  \bibliographystyle{../../Preamble/Physics.bst}
  \bibliography{../../Preamble/reference.bib}
}{}
\end{document}

\bibliographystyle{../../Preamble/Physics.bst}
\bibliography{../../Preamble/reference.bib}

\end{document}