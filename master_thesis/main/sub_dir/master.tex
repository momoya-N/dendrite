\documentclass[autodetect-engine,dvi=dvipdfmx,a4paper,ja=standard]{bxjsbook}
\usepackage{docmute}
\usepackage{../../Preamble/mypackage}

\begin{document}

\documentclass[autodetect-engine,dvi=dvipdfmx,a4paper,ja=standard,oneside,openany,draft]{bxjsarticle}
% \documentclass[autodetect-engine,dvi=dvipdfmx,a4paper,ja=standard,oneside,openany,draft]{bxjsbook}
\usepackage{../Preamble/mypackage}
\begin{document}
\newgeometry{top=80truemm,bottom=30truemm,left=20truemm,right=20truemm}
\newgeometry{top=80mm}
\begin{center}
  \fontsize{30pt}{50pt}\selectfont{界面活性剤の添加による金属樹のマクロなパターン変化\\}
  \vspace{50mm}
  \fontsize{20pt}{25pt}\selectfont{千葉大学融合理工学府 \\先進理化学専攻 物理学コース\\}
  \vspace{10mm}
  \fontsize{20pt}{25pt}\selectfont{23WM2103\\}
  \vspace{3mm}
  \fontsize{25pt}{45pt}\selectfont{中村 友哉\\}
  \vspace{10mm}
  \fontsize{20pt}{25pt}\selectfont{\today}
\end{center}
\end{document}

\thispagestyle{empty}

\restoregeometry
\pagenumbering{roman}
% \documentclass[autodetect-engine,dvi=dvipdfmx,a4paper,ja=standard,oneside,openany,11pt,draft]{bxjsbook}
\documentclass[autodetect-engine,dvi=dvipdfmx,a4paper,ja=standard,oneside,openany,11pt,draft]{bxjsarticle}
\usepackage{../../Preamble/mypackage}

\begin{document}
\section{Abstract}
非平衡状態では平衡系には見られない特徴的なパターンが現れることが知られている。樹枝状パターンは血管や河川など幅広い長さスケールでみられる非平衡パターンで,自己相似構造といった物理的に興味深い特徴を持つ。樹枝状パターンの枝は,成長している枝の先端が分岐することで形成される。そのため,先端のミクロな幾何学的形状は枝の分岐に大きな影響を与えると予想される。しかし,既存の研究ではパターンのフラクタル次元といったマクロな特徴量に着目したものが多く,ミクロな先端形状とマクロな樹枝状パターンとの関係は十分に議論されていない。本研究では,亜鉛の電解析出で生じる樹枝状の金属結晶である金属樹を用いて両スケールの対応を検討した。界面活性剤を溶液に添加すると成長途中の界面のミクロな幾何学的形状($\sim\mathcal{O}(\SI{e-4}{m})$)が変化することが報告されている。そこで,溶液の界面活性剤濃度を変化させ,金属樹の生成過程の時間発展を拡大観察した所,界面活性剤を添加することで成長界面の粗さの成長が遅くなることが確認された。さらに,生成した金属樹のマクロな構造を,フラクタル次元に加え,分岐角度や枝の長さ( $\sim\mathcal{O}(\SI{e-2}{m})$)などを測定することで議論した。その結果,界面活性剤の添加により分岐角度の分布はあまり変化しない一方,枝長の分布は長い側に広がることが明らかになった。本研究は、成長界面のミクロな幾何学的形状がマクロなパターン形成に及ぼす影響を解明し、金属樹に限らず、広く枝分かれ構造を持つパターンの予測に役立つ知見を与えるものである

\ifdraft{
  \bibliographystyle{../Preamble/Physics.bst}
  \bibliography{../Preamble/reference.bib}
}{}
\end{document}
\tableofcontents
\pagenumbering{arabic}

% Part1 General Introduction
% \documentclass[autodetect-engine,dvi=dvipdfmx,a4paper,ja=standard,oneside,openany,11pt,draft]{bxjsbook}
\documentclass[autodetect-engine,dvi=dvipdfmx,a4paper,ja=standard,oneside,openany,11pt,draft]{bxjsarticle}
\usepackage{../../Preamble/mypackage}

\begin{document}


\ifdraft{
  \bibliographystyle{../Preamble/Physics.bst}
  \bibliography{../Preamble/reference.bib}
}{}
\end{document}
\documentclass[autodetect-engine,dvi=dvipdfmx,a4paper,ja=standard,oneside,openany,11pt]{bxjsbook}
\usepackage{../../Preamble/mypackage}

\begin{document}
% \chapter{実験:界面成長}
\section{界面成長}
\subsection{界面成長}
\textbf{界面成長}とは,異なる相の境界面が時間発展により成長していく現象である。たとえば,金属樹は液相内に含まれる金属イオン$\ce{M^{z+}}$が基盤の金属$\ce{M}$の表面に付着して固相に取り込まれて成長していく。

界面成長は,成長界面のランダムなゆらぎによりその形成過程が支配される。例えば,基盤(初期時刻の界面)から成長した高さ$h$の時間$t$に対する成長過程は,ランダムなゆらぎの項を含む次のような方程式で記述される\cite{kardar1987scaling}。
\begin{equation}
  \pdv{h}{t}=\nu\nabla^2h+\lambda(\nabla h)^2+\eta(\bm{x},t)
  \label{eq:KPZ}
\end{equation}
ここで,$\nu$は界面張力,$\lambda$は最低次の非線形項の係数,$\eta(\bm{x},t)$はランダムなゆらぎを表すホワイトノイズである。この方程式は\textbf{Kardar-Parisi-Zhang (KPZ)方程式}と呼ばれる。KPZ方程式は,非線形項の係数$\lambda$が正のとき,成長界面がより粗い界面になり,負のときは滑らかな界面になることが知られている。

また,式\eqref{eq:KPZ}に,より高次の微分項を加えたもの\cite{wolf1990growth}
\begin{equation}
  \pdv{h}{t}=\nu\nabla^2h-K\nabla^4h+\eta(\bm{x},t)
  \label{eq:KPZ_higher}
\end{equation}
なども提案されている。$K$は4次の勾配項の係数である。式\eqref{eq:KPZ_higher}の右辺一項目は界面への吸着・脱着プロセスによる界面緩和の効果を,二項目の線形項は界面拡散のように,化学ポテンシャルの勾配に従う流束による緩和の効果を表しており,電解析出による界面形状を再現することが知られている。

\subsection{leveling効果}
工学分野や電気化学分野において,電解析出(メッキ)を行う際に重要なのが``いかに平坦な析出界面を形成するか''である。表面の粗さを抑えることで,金属結晶の成長過程での格子欠陥を減少させることができる。格子欠陥の少ない金属結晶は,高品質な製品を製造するうえで不可欠なため,平坦に析出させる技術は重要である。そのため,成長過程での界面粗さを抑制するために電解質溶液に微量の添加物を加えて,析出速度や界面の微小な凹凸に対するイオンの析出位置を制御している。電解質溶液に添加物を加えることで析出界面の粗さの成長を抑制できる効果を\textbf{leveling効果}と呼ぶ。

leveling効果を起こす添加物は様々なものが知られており,\textbf{leveler(leveling剤)}と呼ばれる。例えば亜鉛の電解析出においては第4級アンモニウム塩,ポリエチレングリコールなどの高分子,界面活性剤,イオン液体塩,有機酸\cite{sorour2017review}などが知られている。図\ref{fig:leveling}はチオ尿素によるleveling効果の実験結果である\cite{schilardi1998evolution}。

\begin{figure}[htbp]
  \begin{minipage}
    {0.5\textwidth}
    \subcaption{}
    \centering
    \includegraphics[width=0.9\textwidth]{../../figure/part2(exp_surface)/el_dep_surface_no_TU_expfig.png}
    \label{fig:no_leveling_effect}
  \end{minipage}
  \begin{minipage}
    {0.5\textwidth}
    \subcaption{}
    \centering
    \includegraphics[width=0.9\textwidth]{../../figure/part2(exp_surface)/el_dep_surface_0.025M_TU_expfig.png}
    \label{fig:leveling_effect}
  \end{minipage}
  \caption{チオ尿素によるleveling効果に関する過去の報告(実験結果)\cite{schilardi1998evolution}。\subref{fig:no_leveling_effect}チオ尿素を加えない($\SI{0}{M}$)場合の実験結果。\subref{fig:leveling_effect}はチオ尿素を$\SI{0.025}{M}$加えた場合の実験結果。}
  \label{fig:leveling}
\end{figure}

図\ref{fig:leveling}\subref{fig:no_leveling_effect}では早い段階($<\SI{2400}{s}$)で析出界面が粗くなっているのに対し,\ref{fig:leveling}\subref{fig:leveling_effect}では長時間($>\SI{6000}{s}$)経過しても析出界面が平坦になっていることがわかる。また,図\ref{fig:leveling_result}は図\ref{fig:leveling}の成長界面の高さの平均値$\langle h\rangle$を右軸にその標準偏差$W_L$を左軸に表し,その時間変化をプロットしたものである。

\begin{figure}[htbp]
  \begin{minipage}
    {0.5\textwidth}
    \subcaption{}
    \centering
    \includegraphics[width=0.9\textwidth]{../../figure/part2(exp_surface)/el_dep_surface_no_TU_resultfig.png}
    \label{fig:no_leveling_effect_result}
  \end{minipage}
  \begin{minipage}
    {0.5\textwidth}
    \subcaption{}
    \centering
    \includegraphics[width=0.9\textwidth]{../../figure/part2(exp_surface)/el_dep_surface_0.025M_TU_resultfig.png}
    \label{fig:leveling_effect_result}
  \end{minipage}
  \caption{チオ尿素によるleveling効果に関する過去の報告(解析結果)\cite{schilardi1998evolution}\subref{fig:no_leveling_effect_result}チオ尿素を加えない($\SI{0}{M}$)場合の解析結果。\subref{fig:leveling_effect_result}チオ尿素を$\SI{0.025}{M}$加えた場合の解析結果。}
  \label{fig:leveling_result}
\end{figure}

図\ref{fig:leveling_result}からわかるように,チオ尿素を加えることで析出界面の高さの平均値$\langle h\rangle$の成長速度が遅くなり,標準偏差$W_L$の時間発展も遅くなっていることがわかる。これは,チオ尿素によるleveling効果によって析出界面の粗さが抑制されていることを示している。

leveling効果のメカニズムは物質によって様々なものがある\cite{めっき添加剤の作用機構と表面形状制御}が,一般的には添加物の界面への吸着・拡散による界面成長速度の抑制や,金属イオンの析出阻害などが原因とされ\cite{oniciu1991some},界面付近の過飽和度の減少や界面張力の増加によってMS不安定性を抑制することで起こるとされている。

\ifdraft{
  \bibliographystyle{../../Preamble/Physics.bst}
  \bibliography{../../Preamble/reference.bib}
}{}

\end{document}
\documentclass[autodetect-engine,dvi=dvipdfmx,a4paper,ja=standard,oneside,openany,11pt]{bxjsbook}
\usepackage{../../Preamble/mypackage}

\begin{document}
% \chapter{実験:界面成長}
\section{界面成長}
\subsection{界面成長}
\textbf{界面成長}とは,異なる相の境界面が時間発展により成長していく現象である。たとえば,金属樹は液相内に含まれる金属イオン$\ce{M^{z+}}$が基盤の金属$\ce{M}$の表面に付着して固相に取り込まれて成長していく。

界面成長は,成長界面のランダムなゆらぎによりその形成過程が支配される。例えば,基盤(初期時刻の界面)から成長した高さ$h$の時間$t$に対する成長過程は,ランダムなゆらぎの項を含む次のような方程式で記述される\cite{kardar1987scaling}。
\begin{equation}
  \pdv{h}{t}=\nu\nabla^2h+\lambda(\nabla h)^2+\eta(\bm{x},t)
  \label{eq:KPZ}
\end{equation}
ここで,$\nu$は界面張力,$\lambda$は最低次の非線形項の係数,$\eta(\bm{x},t)$はランダムなゆらぎを表すホワイトノイズである。この方程式は\textbf{Kardar-Parisi-Zhang (KPZ)方程式}と呼ばれる。KPZ方程式は,非線形項の係数$\lambda$が正のとき,成長界面がより粗い界面になり,負のときは滑らかな界面になることが知られている。

また,式\eqref{eq:KPZ}に,より高次の微分項を加えたもの\cite{wolf1990growth}
\begin{equation}
  \pdv{h}{t}=\nu\nabla^2h-K\nabla^4h+\eta(\bm{x},t)
  \label{eq:KPZ_higher}
\end{equation}
なども提案されている。$K$は4次の勾配項の係数である。式\eqref{eq:KPZ_higher}の右辺一項目は界面への吸着・脱着プロセスによる界面緩和の効果を,二項目の線形項は界面拡散のように,化学ポテンシャルの勾配に従う流束による緩和の効果を表しており,電解析出による界面形状を再現することが知られている。

\subsection{leveling効果}
工学分野や電気化学分野において,電解析出(メッキ)を行う際に重要なのが``いかに平坦な析出界面を形成するか''である。表面の粗さを抑えることで,金属結晶の成長過程での格子欠陥を減少させることができる。格子欠陥の少ない金属結晶は,高品質な製品を製造するうえで不可欠なため,平坦に析出させる技術は重要である。そのため,成長過程での界面粗さを抑制するために電解質溶液に微量の添加物を加えて,析出速度や界面の微小な凹凸に対するイオンの析出位置を制御している。電解質溶液に添加物を加えることで析出界面の粗さの成長を抑制できる効果を\textbf{leveling効果}と呼ぶ。

leveling効果を起こす添加物は様々なものが知られており,\textbf{leveler(leveling剤)}と呼ばれる。例えば亜鉛の電解析出においては第4級アンモニウム塩,ポリエチレングリコールなどの高分子,界面活性剤,イオン液体塩,有機酸\cite{sorour2017review}などが知られている。図\ref{fig:leveling}はチオ尿素によるleveling効果の実験結果である\cite{schilardi1998evolution}。

\begin{figure}[htbp]
  \begin{minipage}
    {0.5\textwidth}
    \subcaption{}
    \centering
    \includegraphics[width=0.9\textwidth]{../../figure/part2(exp_surface)/el_dep_surface_no_TU_expfig.png}
    \label{fig:no_leveling_effect}
  \end{minipage}
  \begin{minipage}
    {0.5\textwidth}
    \subcaption{}
    \centering
    \includegraphics[width=0.9\textwidth]{../../figure/part2(exp_surface)/el_dep_surface_0.025M_TU_expfig.png}
    \label{fig:leveling_effect}
  \end{minipage}
  \caption{チオ尿素によるleveling効果に関する過去の報告(実験結果)\cite{schilardi1998evolution}。\subref{fig:no_leveling_effect}チオ尿素を加えない($\SI{0}{M}$)場合の実験結果。\subref{fig:leveling_effect}はチオ尿素を$\SI{0.025}{M}$加えた場合の実験結果。}
  \label{fig:leveling}
\end{figure}

図\ref{fig:leveling}\subref{fig:no_leveling_effect}では早い段階($<\SI{2400}{s}$)で析出界面が粗くなっているのに対し,\ref{fig:leveling}\subref{fig:leveling_effect}では長時間($>\SI{6000}{s}$)経過しても析出界面が平坦になっていることがわかる。また,図\ref{fig:leveling_result}は図\ref{fig:leveling}の成長界面の高さの平均値$\langle h\rangle$を右軸にその標準偏差$W_L$を左軸に表し,その時間変化をプロットしたものである。

\begin{figure}[htbp]
  \begin{minipage}
    {0.5\textwidth}
    \subcaption{}
    \centering
    \includegraphics[width=0.9\textwidth]{../../figure/part2(exp_surface)/el_dep_surface_no_TU_resultfig.png}
    \label{fig:no_leveling_effect_result}
  \end{minipage}
  \begin{minipage}
    {0.5\textwidth}
    \subcaption{}
    \centering
    \includegraphics[width=0.9\textwidth]{../../figure/part2(exp_surface)/el_dep_surface_0.025M_TU_resultfig.png}
    \label{fig:leveling_effect_result}
  \end{minipage}
  \caption{チオ尿素によるleveling効果に関する過去の報告(解析結果)\cite{schilardi1998evolution}\subref{fig:no_leveling_effect_result}チオ尿素を加えない($\SI{0}{M}$)場合の解析結果。\subref{fig:leveling_effect_result}チオ尿素を$\SI{0.025}{M}$加えた場合の解析結果。}
  \label{fig:leveling_result}
\end{figure}

図\ref{fig:leveling_result}からわかるように,チオ尿素を加えることで析出界面の高さの平均値$\langle h\rangle$の成長速度が遅くなり,標準偏差$W_L$の時間発展も遅くなっていることがわかる。これは,チオ尿素によるleveling効果によって析出界面の粗さが抑制されていることを示している。

leveling効果のメカニズムは物質によって様々なものがある\cite{めっき添加剤の作用機構と表面形状制御}が,一般的には添加物の界面への吸着・拡散による界面成長速度の抑制や,金属イオンの析出阻害などが原因とされ\cite{oniciu1991some},界面付近の過飽和度の減少や界面張力の増加によってMS不安定性を抑制することで起こるとされている。

\ifdraft{
  \bibliographystyle{../../Preamble/Physics.bst}
  \bibliography{../../Preamble/reference.bib}
}{}

\end{document}
\documentclass[autodetect-engine,dvi=dvipdfmx,a4paper,ja=standard,oneside,openany,11pt,draft,textwidth=50zw]{bxjsbook}
% \documentclass[autodetect-engine,dvi=dvipdfmx,a4paper,ja=standard,oneside,openany,11pt,draft]{bxjsarticle}
\usepackage{../../Preamble/mypackage}

\begin{document}

\chapter{数値計算}
\section{概要}
\subsection{ランダムウォーク(RW)と拡散方程式}
\label{sec:RW}
\begin{wrapfigure}{r}{0.4\textwidth}
  \centering
  \includegraphics[width=0.4\textwidth]{../../figure/part3/RW_d_dim.png}
  \caption{ランダムウォークの模式図}
  \label{fig:random_walk}
\end{wrapfigure}
溶液中で粒子が拡散する際,その運動は\textbf{ランダムウォーク(Random Walk:RW)}でモデル化される。RWは粒子がある確率に従ってランダムに移動する運動である。$d$次元空間内でRWする場合を考える。
図\ref{fig:random_walk}のように,1ステップ$\Delta t$毎に格子間隔$a$で正方向に確率$p_i$,負方向に確率$q_i$で推移する粒子の運動を考える。各軸が選ばれる確率は等しいとすると,各軸に対して,$p_i+q_i=1/d$が成り立つ。この条件のもと,ある時刻$t$,位置$\bm{x}=\{x_1,x_2,\cdots,x_d\}$での粒子の濃度$c(\bm{x},t)$の時間発展を考える。$\Delta t$秒後の濃度は,以下のようにあらわされる。
\begin{equation}
  c(\bm{x},t+\Delta t)=\sum_{i=1}^{d}\left[p_i c(\bm{x}-a\bm{e}_i,t)+q_i c(\bm{x}+a\bm{e}_i,t)\right]
  \label{eq:RW}
\end{equation}
これを左辺は一次,右辺は二次までTaylor展開すると式\ref{eq:RW_taylor}より,以下のようになる。
\begin{equation}
  \pdv{c(\bm{x},t)}{t}=-\sum_{i=1}^{d}(p_i-q_i)\frac{a}{\Delta t}\pdv{c(\bm{x},t)}{x_i}+\frac{a^2}{2d\Delta t}\sum_{i=1}^{d}\pdv[2]{c(\bm{x},t)}{x_i}
  \label{eq:RW_diffusion}
\end{equation}
$d$次元のナブラ演算子$\nabla=\sum_{1}^{d}\bm{e}_i\pdv*{}{x_i}$とする。また,粒子に働く平均的な駆動力$\bar{F}_i=(p_i-q_i)\frac{a}{\Delta t}$,拡散係数$D=\frac{a^2}{2d\Delta t}$とすると,以下のように表される。
\begin{equation}
  \pdv{c(\bm{x},t)}{t}=-\bm{\bar{F}}\cdot\nabla c(\bm{x},t)+D\nabla^2 c(\bm{x},t)
  \label{eq:RW_diffusion_result}
\end{equation}
ここで,ベクトル演算の恒等式$\nabla\cdot(\psi\bm{a})=\bm{a}\cdot\nabla\psi+\psi\nabla\cdot\bm{a}$より,駆動力$\bar{F}$が,$\nabla\cdot\bar{F}=0$を満たすならば,式\ref{eq:RW_diffusion_result}は,
\begin{equation}
  \pdv{c(\bm{x},t)}{t}=-\nabla\cdot\ab\{\bm{\bar{F}}c(\bm{x},t)-D\nabla c(\bm{x},t)\}
  \label{eq:RW_diffusion_result_divergence}
\end{equation}
と表される。これは流れを$\bm{J}=\bm{\bar{F}}c(\bm{x},t)-D\nabla c(\bm{x},t)$としたときの粒子の保存則に他ならない。
例として,電界析出における,電解質溶液中を拡散するイオンの運動を考える。外力が電場のみかかっている場合,電位$\phi$,イオンの電荷$q$,易動度$\mu$とすると,$\bm{\bar{F}}=\mu q\bm{E}=-\mu q \nabla\phi$となる。この時,電荷密度$\rho(\bm{x},t)$,誘電率$\varepsilon$とすれば,ガウスの法則より,$\nabla\cdot\bm{\bar{F}}=-\mu q \nabla^2\phi=-\mu q \rho(\bm{x,t})/\varepsilon$が成り立つ。電界析出において初期に急激な電流が流れた後は気体の発生が見られないことより,溶液中では電気的中性が満たされているとみなせる。よって$\rho(\bm{x,t})=0$となり,$\nabla\cdot\bar{F}=0$を満たす。そのため,電界析出におけるイオンの運動は流れを$\bm{J}=-\mu qc(\bm{x},t) \nabla\phi-D\nabla c(\bm{x},t)$としたときの\textcolor{red}{イオンの保存則(拡散方程式)(*等価な表現としてもいいのか?)}
\begin{equation}
  \pdv{c(\bm{x},t)}{t}=-\nabla\cdot\bm{J}
  \label{eq:RW_diffusion_result_divergence_ion}
\end{equation}
で表される。以上の内容は文献\cite{フラクタルの物理Ⅱ}\cite{フラクタル科学}を参考にした。
\subsection{Langevin方程式}
\label{sec:Langevin}
ブラウン運動する粒子の運動は,粘性抵抗などによる,速度に比例する減衰項と,外力項,ランダム力の項を含むLangevin方程式で表され,以下のように与えられる。
\begin{equation}
  m\odv[2]{\bm{x}}{t}=-\gamma\odv{\bm{x}}{t}+\bm{F}+\bm{\xi}(t)
  \label{eq:Langevin}
\end{equation}
ここで,$m$は粒子の質量,$\gamma$は粘性抵抗係数,$\bm{F}$は外力,$\bm{\xi}(t)$はランダム力である。ランダム力は以下の関係式を満たす。
\begin{equation}
  \begin{split}
    \langle\bm{\xi}(t)\rangle                  & =0                                                   \\
    \langle\bm{\xi}(t)\cdot\bm{\xi}(t')\rangle & =2d\gamma k_B T\delta(t-t') \qquad (d:\mathrm{空間次元})
  \end{split}
  \label{eq:random_force}
\end{equation}
ここで,$\langle\cdots\rangle$は確率分布に関する平均を表し,$k_B$はボルツマン定数,$T$は温度である。過減衰極限として,慣性項を無視し,外力が電場のみである場合を考えると,運動方程式は以下のようになる。
\begin{equation}
  \odv{\bm{x}}{t}=\mu q\bm{E}+\mu\bm{\xi}(t) \qquad (\mu=1/\gamma:\mathrm{易動度})
  \label{eq:Langevin_overdamped}
\end{equation}
ここで,$q$は粒子の電荷,$\bm{E}$は電場である。粒子の拡散の時間スケールに対して,電場の時間変化のスケールが十分遅いと仮定して,$\bm{E}=\bm{E}(\bm{x})$(位置のみに依存)とする。初期位置$\bm{x}(0)=0$とすると,その平均と二乗平均は\ref{eq:Langevin_overdamped_average}より,以下のようになる。

\begin{equation}
  \begin{split}
    \langle\bm{x}(t)\rangle   & =\mu q\langle\bm{E}\rangle t                    \\
    \langle\bm{x}(t)^2\rangle & =(\mu q \langle\bm{E}\rangle t)^2+2d\mu k_B T t
  \end{split}
  \label{eq:Langevin_overdamped_average}
\end{equation}
これより,粒子の位置の分散は以下のように与えられる。
\begin{equation}
  \langle(\bm{x}(t)-\langle\bm{x}(t)\rangle)^2\rangle=2d \frac{k_B T}{\gamma} t \quad (=2d D t, D=k_B T/\gamma:\mathrm{拡散係数})
  \label{eq:Langevin_overdamped_variance}
\end{equation}
これより,分散は外力(電場)によらない一定値になることがわかる。

\subsection{拡散律速凝集(Diffusion-limited aggregation, DLA)モデル}

\ifdraft{
  \bibliographystyle{../../Preamble/Physics.bst}
  \bibliography{../../Preamble/reference.bib}
}{}
\end{document}
%研究目的,後で分割すること
\section{研究目的}
枝分かれパターンの形成メカニズムは現象によって異なる。しかし,枝の分岐(Tip Splitting)は成長過程で界面が不安定化し粗くなることで分岐していくため,界面の粗さの不安定性が最終的な枝分かれ形状に影響を与えると思われる。しかし,成長界面の粗さの不安定性と最終的な枝分かれ形状の関係は十分に理解されていない。本研究では,成長界面の粗さの不安定性と枝分かれ形状の関係を明らかにすることを目的とする。モデル系として,亜鉛イオン水溶液の電界析出で生じる亜鉛の金属樹を用いる。亜鉛の金属樹はフラクタル性を示すことが知られており\cite{matsushita1984fractal},様々なパラメータでの形態変化が調べられている\cite{suda2003temperature}。本研究では電解質溶液に界面活性剤を加え,最終的に生じる樹枝状パターンがどのように変化するかを議論するとともに,パターン変化の原因であると思われる,電界析出する金属結晶の界面の粗さを平滑化させるleveling現象を検証した。また,枝分かれパターンのフラクタル性を再現するモデルである拡散律速凝集(Diffusion Limited Aggregation: DLA)モデル\cite{witten1981diffusion}をベースに,実験を再現するモデルの構築を行った。

% Part2-1 Experiment(dendrite)
% \documentclass[autodetect-engine,dvi=dvipdfmx,a4paper,ja=standard,oneside,openany,11pt,draft]{bxjsbook}
\documentclass[autodetect-engine,dvi=dvipdfmx,a4paper,ja=standard,oneside,openany,11pt,draft]{bxjsarticle}
\usepackage{../../Preamble/mypackage}

\begin{document}
\sectrion{金属樹}
\subsection{実験系}
\subsection{実験方法}
\subsection{解析方法}
\subsubsection{画像解析}
\subsubsection{フラクタル次元解析}
金属樹の枝分かれ構造を特徴づけるためにフラクタル次元の計測を行った。フラクタル次元の



\ifdraft{
  \bibliographystyle{../../Preamble/Physics.bst}
  \bibliography{../../Preamble/reference.bib}
}{}
\end{document}
\documentclass[autodetect-engine,dvi=dvipdfmx,a4paper,ja=standard,oneside,openany,11pt]{bxjsbook}
\usepackage{../../Preamble/mypackage}

\begin{document}
\section{実験結果}
\subsection{金属樹の概観・フラクタル次元}

\begin{figure}[htbp]
  \begin{minipage}
    {0.32\textwidth}
    \subcaption{}
    \centering
    \includegraphics[width=0.9\textwidth]{../../figure/part2(exp_deposition)/3620s_0.00sur.png}
    \label{fig:non_surfactant}
  \end{minipage}
  \begin{minipage}
    {0.32\textwidth}
    \subcaption{}
    \centering
    \includegraphics[width=0.9\textwidth]{../../figure/part2(exp_deposition)/3092s_0.005sur.png}
    \label{fig:0.005_surfactant}
  \end{minipage}
  \begin{minipage}
    {0.32\textwidth}
    \subcaption{}
    \centering
    \includegraphics[width=0.9\textwidth]{../../figure/part2(exp_deposition)/3548s_0.01sur.png}
    \label{fig:0.01_surfactant}
  \end{minipage}
  \\
  \begin{minipage}
    {0.32\textwidth}
    \subcaption{}
    \centering
    \includegraphics[width=0.9\textwidth]{../../figure/part2(exp_deposition)/3126s_0.03sur.png}
    \label{fig:0.03_surfactant}
  \end{minipage}
  \begin{minipage}
    {0.32\textwidth}
    \subcaption{}
    \centering
    \includegraphics[width=0.9\textwidth]{../../figure/part2(exp_deposition)/814s_0.05sur.png}
    \label{fig:0.05_surfactant}
  \end{minipage}
  \caption{界面活性剤濃度による金属樹の形態変化\subref{fig:non_surfactant}界面活性剤濃度$\SI{0}{\mathrm{vol}\%}$,$\SI{3620}{s}$,\subref{fig:0.005_surfactant}界面活性剤濃度$\SI{0.005}{\mathrm{vol}\%}$,$\SI{3092}{s}$\subref{fig:0.01_surfactant}界面活性剤濃度$\SI{0.01}{\mathrm{vol}\%}$,$\SI{3548}{s}$,\subref{fig:0.03_surfactant}界面活性剤濃度$\SI{0.03}{\mathrm{vol}\%}$,$\SI{3126}{s}$,\subref{fig:0.05_surfactant}界面活性剤濃度$\SI{0.05}{\mathrm{vol}\%}$,$\SI{814}{s}$。}
  \label{fig:surfactant}
\end{figure}

図\ref{fig:surfactant}は界面活性剤を加えた際の,各濃度ごとにおける金属樹の最終形状の結果である。界面活性剤濃度が高くなるほど定性的には枝分かれが減少していく。これらのフラクタル次元をボックスカウンティング法を用いて計測したものが図\ref{fig:fractal_dim}である。

\begin{figure}[htbp]
  \centering
  \includegraphics[width=0.5\textwidth]{../../figure/part2(exp_deposition)/fractal_dim_result.png}
  \caption{界面活性剤濃度によるフラクタル次元の変化}
  \label{fig:fractal_dim}
\end{figure}

図\ref{fig:fractal_dim}より,界面活性剤濃度が高くなるほどフラクタル次元が小さくなることがわかった。また,界面活性剤濃度が0.03\%以上の条件ではフラクタル次元が減少していることも分かった。また,界面活性剤濃度が高くなれほどばらつきが大きくなっていた。

\subsection{枝の本数・太さ}

\begin{figure}[htbp]
  \begin{minipage}
    {0.45\textwidth}
    \subcaption{}
    \centering
    \includegraphics[width=0.9\textwidth]{../../figure/part2(exp_deposition)/branch_num.png}
    \label{fig:branch_number}
  \end{minipage}
  \begin{minipage}
    {0.45\textwidth}
    \subcaption{}
    \centering
    \includegraphics[width=0.9\textwidth]{../../figure/part2(exp_deposition)/branch_thickness_mean.png}
    \label{fig:branch_thickness}
  \end{minipage}
  \caption{中心からの距離$r$の円と交わる枝の本数・太さの結果。\subref{fig:branch_number}枝の本数。\subref{fig:branch_thickness}枝の太さ。}
  \label{fig:branch}
\end{figure}

図\ref{fig:branch}\subref{fig:branch_number}は中心からの距離$r$の円と交わる枝の本数,図\ref{fig:branch}\subref{fig:branch_thickness}は枝の太さを示している。枝の太さは半径$r$の円と重なった枝の太さの平均値を表している。まず,枝の本数について,界面活性剤濃度$\SI{0.03}{\mathrm{vol}\%}$未満では$r$の増加と共に本数も増加していく。一方,$\SI{0.03}{\mathrm{vol}\%}$以上では枝の本数は$r$によらず同じ程度である。

図\ref{fig:branch}\subref{fig:branch_number}の黒破線は例として枝の本数20本を表している。低濃度側ではこの線を超えることが多い。一方で高濃度側ではこの線を超えないことが多いことからも,高濃度側においては,枝の本数は$r$が増加してもあまり増加しないことが分かる。

また,図\ref{fig:branch}\subref{fig:branch_thickness}は枝の太さを示しており,界面活性剤濃度$\SI{0.03}{\mathrm{vol}\%}$未満では,$r$によらず枝の太さは比較的小さく一定で,ばらつきも小さい。一方,界面活性剤濃度$\SI{0.03}{\mathrm{vol}\%}$以上では,枝の太さが$r$とともに緩やかに大きくなり,ばらつきも大きくなる。

図中黒破線は例として太さ約$\SI{0.6}{mm}$を表している。
高濃度側では$\SI{0.6}{mm}$を超えることが多い一方で,低濃度側では$\SI{0.6}{mm}$を超えないことが多いことからも,濃度による枝の太さの違いが見て取れる。

$r$の増加と時間経過はおおむね比例することより,界面活性剤濃度が$\SI{0.03}{\mathrm{vol}\%}$未満では金属樹は頻繁に細い枝が発生,枝分かれをする。一方で$\SI{0.03}{\mathrm{vol}\%}$以上では枝分かれが抑制され,太い枝として成長していくことが分かった。

\subsection{分岐角度}

\begin{figure}[htbp]
  \begin{minipage}
    {0.5\textwidth}
    \subcaption{}
    \centering
    \includegraphics[width=0.9\textwidth]{../../figure/part2(exp_deposition)/angle_in_result.png}
    \label{fig:angle_in}
  \end{minipage}
  \begin{minipage}
    {0.45\textwidth}
    \subcaption{}
    \centering
    \includegraphics[width=0.9\textwidth]{../../figure/part2(exp_deposition)/angle_out_result.png}
    \label{fig:angle_out}
  \end{minipage}
  \caption{分岐角度の界面活性剤濃度による分布の変化。ビン幅は$0.1\pi$。\subref{fig:angle_in}枝の内側の分岐角度$\theta_{\mathrm{in}}$の分布。\subref{fig:angle_out}枝の外側の分岐角度$\theta_{\mathrm{out}}$の分布。}
  \label{fig:angle}
\end{figure}

\begin{table}
  \centering
  \caption{界面活性剤濃度による$\theta_{\mathrm{in}}$の平均値}
  \begin{tabular}{|c|c|}
    \hline
    濃度      & $\theta_{\mathrm{in}}$の平均値 \\
    \hline\hline
    0.00\%  & $0.392\pi$                 \\ \hline
    0.005\% & $0.393\pi$                 \\ \hline
    0.10\%  & $0.389\pi$                 \\ \hline
    0.30\%  & $0.406\pi$                 \\ \hline
    0.50\%  & $0.422\pi$                 \\
    \hline
  \end{tabular}
  \label{tab:angle_average}
\end{table}

図\ref{fig:angle}は枝の分岐角度の分布である。図\ref{fig:angle}\subref{fig:angle_in}は枝の内側の分岐角度の分布で,表\ref{tab:angle_average}は各濃度における$\theta_{\mathrm{in}}$の平均値である。分岐角度の分布はフラクタル次元ほど明らかな変化は見られず,界面活性剤の濃度によらずほぼ同じような分布となっている。内側の分岐角度の平均値はおおよそ$2\pi/5$ radになっていた。

図\ref{fig:angle}\subref{fig:angle_out}は枝の外側の分岐角度の分布である。これも濃度によらずほぼ同じような分布となっている。$\theta_{\mathrm{out}}$の値は8割以上が$0.8\pi$ 以上であり,真横に曲がるなどの分岐は少ないことが分かった。
\subsection{枝の長さ}

\begin{figure}[htbp]
  \begin{minipage}
    {0.45\textwidth}
    \subcaption{}
    \centering
    \includegraphics[width=0.9\textwidth]{../../figure/part2(exp_deposition)/branch_hist_abs.png}
    \label{fig:branch_length_absolute}
  \end{minipage}
  \begin{minipage}
    {0.45\textwidth}
    \subcaption{}
    \centering
    \includegraphics[width=0.9\textwidth]{../../figure/part2(exp_deposition)/branch_hist_relativ.png}
    \label{fig:branch_length_relativ}
  \end{minipage}
  \caption{枝の長さ分布。\subref{fig:branch_length_absolute}絶対度数表示。\subref{fig:branch_length_relativ}相対度数表示。}
  \label{fig:branch_length}
\end{figure}

図\ref{fig:branch_length}は枝の分岐点から次の分岐点までの枝(図\ref{fig:branch_def_input}\subref{fig:branch_def}のピンクの枝)の長さ分布を,カメラの解像度($\SI{1}{px}=11/\SI{681}{cm}\approx \SI{0.16}{mm}$)未満を除いて示している。図\ref{fig:branch_length}\subref{fig:branch_length_relativ}は相対度数で表しており,図\ref{fig:branch_length}\subref{fig:branch_length_absolute}は絶対度数で表している。\ref{fig:branch_length}\subref{fig:branch_length_relativ}で示したように,界面活性剤濃度が高くなるほど長い枝の割合が増加していることが分かった(図中赤丸)。また,極端に長いものが低濃度($\SI{0}{\mathrm{vol}\%}$と$\SI{0.005}{\mathrm{vol}\%}$)でわずかな割合見られることが分かった(図中青丸)。これは図\ref{fig:branch_length}\subref{fig:branch_length_absolute}からわかるように,一本だけ存在する長い枝の影響である。

\begin{figure}[htbp]
  \begin{minipage}
    {0.45\textwidth}
    \subcaption{}
    \centering
    \includegraphics[width=0.9\textwidth]{../../figure/part2(exp_deposition)/branch_edited_hist_abs.png}
    \label{fig:branch_length_absolute_edited}
  \end{minipage}
  \begin{minipage}
    {0.45\textwidth}
    \subcaption{}
    \centering
    \includegraphics[width=0.9\textwidth]{../../figure/part2(exp_deposition)/branch_edited_hist_relativ.png}
    \label{fig:branch_length_relativ_edited}
  \end{minipage}
  \caption{再構成後の枝の長さの分布。\subref{fig:branch_length_absolute_edited}絶対度数表示。\subref{fig:branch_length_relativ_edited}相対度数表示。}
  \label{fig:branch_length_edited}
\end{figure}

図\ref{fig:branch_length_edited}は$\theta_{\mathrm{out}}$が$0.9\pi$ 以上になる枝の組み合わせを一本の枝としたものと,それ以外の組になっていない枝(図\ref{fig:branch_def_input}\subref{fig:branch_def}のピンクの枝と緑の枝)の長さの分布を解像度($\SI{1}{px}=11/681 \si{cm}\approx \SI{0.16}{mm}$)未満を除いて示している。図\ref{fig:branch_length_edited}\subref{fig:branch_length_absolute_edited}は絶対度数で表しており,図\ref{fig:branch_length_edited}\subref{fig:branch_length_relativ_edited}は相対度数で表している。図\ref{fig:branch_length}に比べて,高濃度における長い枝の割合が増加していた。また,極端に長いものも同様に低濃度側で見られた。これは図\ref{fig:branch_length}と同様に,一本だけ存在する長い枝の影響である。

\begin{figure}[htbp]
  \begin{minipage}
    {0.59\textwidth}
    \subcaption{}
    \centering
    \includegraphics[width=0.9\textwidth]{../../figure/part2(exp_deposition)/exp_b_branch_len.png}
    \label{fig:exp_b_branch_len}
  \end{minipage}
  \begin{minipage}
    {0.39\textwidth}
    \subcaption{}
    \centering
    \includegraphics[width=0.9\textwidth]{../../figure/part2(exp_deposition)/exp_b_branch_edited_len.png}
    \label{fig:exp_b_branch_edited_len}
  \end{minipage}
  \caption{枝の長さの冪分布の指数$b$の測定結果。\subref{fig:exp_b_branch_len}図\ref{fig:branch_def_input}\subref{fig:branch_def}のピンクの枝の指数$b$。\subref{fig:exp_b_branch_edited_len}図\ref{fig:branch_def_input}\subref{fig:branch_def}のピンクと緑の枝の指数$b$。}
  \label{fig:branch_length_exp}
\end{figure}

\begin{table}[htbp]
  \begin{minipage}{0.45\textwidth}
    \centering
    \caption{界面活性剤濃度による図\ref{fig:branch_def_input}\subref{fig:branch_def}のピンク色の枝の分布の指数$b$の平均値}
    \begin{tabular}{|c|c|}
      \hline
      濃度      & 指数$b$の平均値 \\ \hline\hline
      0.00\%  & $4.29$    \\ \hline
      0.005\% & $3.88$    \\ \hline
      0.10\%  & $4.41$    \\ \hline
      0.30\%  & $3.89$    \\ \hline
      0.50\%  & $3.76$    \\
      \hline
    \end{tabular}
    \label{tab:brnch_len_exp}
  \end{minipage}
  \hfill
  \begin{minipage}{0.45\textwidth}
    \centering
    \caption{界面活性剤濃度による図\ref{fig:branch_def_input}\subref{fig:branch_def}のピンクと緑色の枝の分布の指数$b$の平均値}
    \begin{tabular}{|c|c|}
      \hline
      濃度      & 指数$b$の平均値 \\ \hline\hline
      0.00\%  & $3.33$    \\ \hline
      0.005\% & $3.29$    \\ \hline
      0.10\%  & $3.46$    \\ \hline
      0.30\%  & $3.24$    \\ \hline
      0.50\%  & $2.92$    \\
      \hline
    \end{tabular}
    \label{tab:branch_len_exp_edited}
  \end{minipage}
\end{table}

図\ref{fig:branch_length_exp}は界面活性剤ごとの枝の長さの分布の指数$b$を示している。図\ref{fig:branch_length_exp}\subref{fig:exp_b_branch_len}は図\ref{fig:branch_def_input}\subref{fig:branch_def}で定義されるピンクの枝の分布の指数$b$であり,図\ref{fig:branch_length_exp}\subref{fig:exp_b_branch_edited_len}は図\ref{fig:branch_def_input}\subref{fig:branch_def}で定義されるピンクと緑の枝の分布の指数$b$である。表\ref{tab:brnch_len_exp}はピンクの枝の分布の指数$b$の平均値であり,表\ref{tab:branch_len_exp_edited}はピンクと緑の枝の分布の指数$b$の平均値である。

図\ref{fig:branch_length_exp}\subref{fig:exp_b_branch_len}より,ピンクの枝の指数$b$の値は,
\begin{itemize}
  \item $\SI{0}{\mathrm{vol}\%}$,$\SI{0.01}{\mathrm{vol}\%}$では相対的に大きな値($b>4.00$)
  \item $\SI{0.03}{\mathrm{vol}\%}$以上では相対的に小さな値($b<4.00$)
  \item $\SI{0.005}{\mathrm{vol}\%}$は傾向から外れている
\end{itemize}
となることがわかった。$\SI{0.005}{\mathrm{vol}\%}$が低い値を取っているのは,図\ref{fig:branch_length},\ref{fig:branch_length_edited}で言及した,低濃度側で見られる極端に長い枝の影響だと思われる。

図\ref{fig:branch_length_exp}\subref{fig:exp_b_branch_edited_len}より,ピンクと緑の枝の分布は,
\begin{itemize}
  \item 0.03\%以下では相対的に大きな値($b>3.20$)
  \item 0.05\%では相対的に小さな値($b<3.20$)
\end{itemize}
となることが分かった。$\SI{0.005}{\mathrm{vol}\%}$の値がピンクの枝だけの時に比べて相対的に大きい値となっているのは,$\SI{0.03}{\mathrm{vol}\%}$以上の濃度での緑の枝(一本の枝とみなしたもの)の割合が増えたためだと思われる。
\ifdraft{
  \bibliographystyle{../../Preamble/Physics.bst}
  \bibliography{../../Preamble/reference.bib}
}{}
\end{document}
% \documentclass[autodetect-engine,dvi=dvipdfmx,a4paper,ja=standard,oneside,openany,11pt,draft]{bxjsbook}
\documentclass[autodetect-engine,dvi=dvipdfmx,a4paper,ja=standard,oneside,openany,11pt,draft]{bxjsarticle}
\usepackage{../../Preamble/mypackage}

\begin{document}


\ifdraft{
  \bibliographystyle{../Preamble/Physics.bst}
  \bibliography{../Preamble/reference.bib}
}{}
\end{document}

% Part2-2 Experiment(surface)
% \documentclass[autodetect-engine,dvi=dvipdfmx,a4paper,ja=standard,oneside,openany,11pt,draft]{bxjsbook}
\documentclass[autodetect-engine,dvi=dvipdfmx,a4paper,ja=standard,oneside,openany,11pt,draft]{bxjsarticle}
\usepackage{../../Preamble/mypackage}

\begin{document}
\sectrion{金属樹}
\subsection{実験系}
\subsection{実験方法}
\subsection{解析方法}
\subsubsection{画像解析}
\subsubsection{フラクタル次元解析}
金属樹の枝分かれ構造を特徴づけるためにフラクタル次元の計測を行った。フラクタル次元の



\ifdraft{
  \bibliographystyle{../../Preamble/Physics.bst}
  \bibliography{../../Preamble/reference.bib}
}{}
\end{document}
\documentclass[autodetect-engine,dvi=dvipdfmx,a4paper,ja=standard,oneside,openany,11pt]{bxjsbook}
\usepackage{../../Preamble/mypackage}

\begin{document}
\section{実験結果}
\subsection{金属樹の概観・フラクタル次元}

\begin{figure}[htbp]
  \begin{minipage}
    {0.32\textwidth}
    \subcaption{}
    \centering
    \includegraphics[width=0.9\textwidth]{../../figure/part2(exp_deposition)/3620s_0.00sur.png}
    \label{fig:non_surfactant}
  \end{minipage}
  \begin{minipage}
    {0.32\textwidth}
    \subcaption{}
    \centering
    \includegraphics[width=0.9\textwidth]{../../figure/part2(exp_deposition)/3092s_0.005sur.png}
    \label{fig:0.005_surfactant}
  \end{minipage}
  \begin{minipage}
    {0.32\textwidth}
    \subcaption{}
    \centering
    \includegraphics[width=0.9\textwidth]{../../figure/part2(exp_deposition)/3548s_0.01sur.png}
    \label{fig:0.01_surfactant}
  \end{minipage}
  \\
  \begin{minipage}
    {0.32\textwidth}
    \subcaption{}
    \centering
    \includegraphics[width=0.9\textwidth]{../../figure/part2(exp_deposition)/3126s_0.03sur.png}
    \label{fig:0.03_surfactant}
  \end{minipage}
  \begin{minipage}
    {0.32\textwidth}
    \subcaption{}
    \centering
    \includegraphics[width=0.9\textwidth]{../../figure/part2(exp_deposition)/814s_0.05sur.png}
    \label{fig:0.05_surfactant}
  \end{minipage}
  \caption{界面活性剤濃度による金属樹の形態変化\subref{fig:non_surfactant}界面活性剤濃度$\SI{0}{\mathrm{vol}\%}$,$\SI{3620}{s}$,\subref{fig:0.005_surfactant}界面活性剤濃度$\SI{0.005}{\mathrm{vol}\%}$,$\SI{3092}{s}$\subref{fig:0.01_surfactant}界面活性剤濃度$\SI{0.01}{\mathrm{vol}\%}$,$\SI{3548}{s}$,\subref{fig:0.03_surfactant}界面活性剤濃度$\SI{0.03}{\mathrm{vol}\%}$,$\SI{3126}{s}$,\subref{fig:0.05_surfactant}界面活性剤濃度$\SI{0.05}{\mathrm{vol}\%}$,$\SI{814}{s}$。}
  \label{fig:surfactant}
\end{figure}

図\ref{fig:surfactant}は界面活性剤を加えた際の,各濃度ごとにおける金属樹の最終形状の結果である。界面活性剤濃度が高くなるほど定性的には枝分かれが減少していく。これらのフラクタル次元をボックスカウンティング法を用いて計測したものが図\ref{fig:fractal_dim}である。

\begin{figure}[htbp]
  \centering
  \includegraphics[width=0.5\textwidth]{../../figure/part2(exp_deposition)/fractal_dim_result.png}
  \caption{界面活性剤濃度によるフラクタル次元の変化}
  \label{fig:fractal_dim}
\end{figure}

図\ref{fig:fractal_dim}より,界面活性剤濃度が高くなるほどフラクタル次元が小さくなることがわかった。また,界面活性剤濃度が0.03\%以上の条件ではフラクタル次元が減少していることも分かった。また,界面活性剤濃度が高くなれほどばらつきが大きくなっていた。

\subsection{枝の本数・太さ}

\begin{figure}[htbp]
  \begin{minipage}
    {0.45\textwidth}
    \subcaption{}
    \centering
    \includegraphics[width=0.9\textwidth]{../../figure/part2(exp_deposition)/branch_num.png}
    \label{fig:branch_number}
  \end{minipage}
  \begin{minipage}
    {0.45\textwidth}
    \subcaption{}
    \centering
    \includegraphics[width=0.9\textwidth]{../../figure/part2(exp_deposition)/branch_thickness_mean.png}
    \label{fig:branch_thickness}
  \end{minipage}
  \caption{中心からの距離$r$の円と交わる枝の本数・太さの結果。\subref{fig:branch_number}枝の本数。\subref{fig:branch_thickness}枝の太さ。}
  \label{fig:branch}
\end{figure}

図\ref{fig:branch}\subref{fig:branch_number}は中心からの距離$r$の円と交わる枝の本数,図\ref{fig:branch}\subref{fig:branch_thickness}は枝の太さを示している。枝の太さは半径$r$の円と重なった枝の太さの平均値を表している。まず,枝の本数について,界面活性剤濃度$\SI{0.03}{\mathrm{vol}\%}$未満では$r$の増加と共に本数も増加していく。一方,$\SI{0.03}{\mathrm{vol}\%}$以上では枝の本数は$r$によらず同じ程度である。

図\ref{fig:branch}\subref{fig:branch_number}の黒破線は例として枝の本数20本を表している。低濃度側ではこの線を超えることが多い。一方で高濃度側ではこの線を超えないことが多いことからも,高濃度側においては,枝の本数は$r$が増加してもあまり増加しないことが分かる。

また,図\ref{fig:branch}\subref{fig:branch_thickness}は枝の太さを示しており,界面活性剤濃度$\SI{0.03}{\mathrm{vol}\%}$未満では,$r$によらず枝の太さは比較的小さく一定で,ばらつきも小さい。一方,界面活性剤濃度$\SI{0.03}{\mathrm{vol}\%}$以上では,枝の太さが$r$とともに緩やかに大きくなり,ばらつきも大きくなる。

図中黒破線は例として太さ約$\SI{0.6}{mm}$を表している。
高濃度側では$\SI{0.6}{mm}$を超えることが多い一方で,低濃度側では$\SI{0.6}{mm}$を超えないことが多いことからも,濃度による枝の太さの違いが見て取れる。

$r$の増加と時間経過はおおむね比例することより,界面活性剤濃度が$\SI{0.03}{\mathrm{vol}\%}$未満では金属樹は頻繁に細い枝が発生,枝分かれをする。一方で$\SI{0.03}{\mathrm{vol}\%}$以上では枝分かれが抑制され,太い枝として成長していくことが分かった。

\subsection{分岐角度}

\begin{figure}[htbp]
  \begin{minipage}
    {0.5\textwidth}
    \subcaption{}
    \centering
    \includegraphics[width=0.9\textwidth]{../../figure/part2(exp_deposition)/angle_in_result.png}
    \label{fig:angle_in}
  \end{minipage}
  \begin{minipage}
    {0.45\textwidth}
    \subcaption{}
    \centering
    \includegraphics[width=0.9\textwidth]{../../figure/part2(exp_deposition)/angle_out_result.png}
    \label{fig:angle_out}
  \end{minipage}
  \caption{分岐角度の界面活性剤濃度による分布の変化。ビン幅は$0.1\pi$。\subref{fig:angle_in}枝の内側の分岐角度$\theta_{\mathrm{in}}$の分布。\subref{fig:angle_out}枝の外側の分岐角度$\theta_{\mathrm{out}}$の分布。}
  \label{fig:angle}
\end{figure}

\begin{table}
  \centering
  \caption{界面活性剤濃度による$\theta_{\mathrm{in}}$の平均値}
  \begin{tabular}{|c|c|}
    \hline
    濃度      & $\theta_{\mathrm{in}}$の平均値 \\
    \hline\hline
    0.00\%  & $0.392\pi$                 \\ \hline
    0.005\% & $0.393\pi$                 \\ \hline
    0.10\%  & $0.389\pi$                 \\ \hline
    0.30\%  & $0.406\pi$                 \\ \hline
    0.50\%  & $0.422\pi$                 \\
    \hline
  \end{tabular}
  \label{tab:angle_average}
\end{table}

図\ref{fig:angle}は枝の分岐角度の分布である。図\ref{fig:angle}\subref{fig:angle_in}は枝の内側の分岐角度の分布で,表\ref{tab:angle_average}は各濃度における$\theta_{\mathrm{in}}$の平均値である。分岐角度の分布はフラクタル次元ほど明らかな変化は見られず,界面活性剤の濃度によらずほぼ同じような分布となっている。内側の分岐角度の平均値はおおよそ$2\pi/5$ radになっていた。

図\ref{fig:angle}\subref{fig:angle_out}は枝の外側の分岐角度の分布である。これも濃度によらずほぼ同じような分布となっている。$\theta_{\mathrm{out}}$の値は8割以上が$0.8\pi$ 以上であり,真横に曲がるなどの分岐は少ないことが分かった。
\subsection{枝の長さ}

\begin{figure}[htbp]
  \begin{minipage}
    {0.45\textwidth}
    \subcaption{}
    \centering
    \includegraphics[width=0.9\textwidth]{../../figure/part2(exp_deposition)/branch_hist_abs.png}
    \label{fig:branch_length_absolute}
  \end{minipage}
  \begin{minipage}
    {0.45\textwidth}
    \subcaption{}
    \centering
    \includegraphics[width=0.9\textwidth]{../../figure/part2(exp_deposition)/branch_hist_relativ.png}
    \label{fig:branch_length_relativ}
  \end{minipage}
  \caption{枝の長さ分布。\subref{fig:branch_length_absolute}絶対度数表示。\subref{fig:branch_length_relativ}相対度数表示。}
  \label{fig:branch_length}
\end{figure}

図\ref{fig:branch_length}は枝の分岐点から次の分岐点までの枝(図\ref{fig:branch_def_input}\subref{fig:branch_def}のピンクの枝)の長さ分布を,カメラの解像度($\SI{1}{px}=11/\SI{681}{cm}\approx \SI{0.16}{mm}$)未満を除いて示している。図\ref{fig:branch_length}\subref{fig:branch_length_relativ}は相対度数で表しており,図\ref{fig:branch_length}\subref{fig:branch_length_absolute}は絶対度数で表している。\ref{fig:branch_length}\subref{fig:branch_length_relativ}で示したように,界面活性剤濃度が高くなるほど長い枝の割合が増加していることが分かった(図中赤丸)。また,極端に長いものが低濃度($\SI{0}{\mathrm{vol}\%}$と$\SI{0.005}{\mathrm{vol}\%}$)でわずかな割合見られることが分かった(図中青丸)。これは図\ref{fig:branch_length}\subref{fig:branch_length_absolute}からわかるように,一本だけ存在する長い枝の影響である。

\begin{figure}[htbp]
  \begin{minipage}
    {0.45\textwidth}
    \subcaption{}
    \centering
    \includegraphics[width=0.9\textwidth]{../../figure/part2(exp_deposition)/branch_edited_hist_abs.png}
    \label{fig:branch_length_absolute_edited}
  \end{minipage}
  \begin{minipage}
    {0.45\textwidth}
    \subcaption{}
    \centering
    \includegraphics[width=0.9\textwidth]{../../figure/part2(exp_deposition)/branch_edited_hist_relativ.png}
    \label{fig:branch_length_relativ_edited}
  \end{minipage}
  \caption{再構成後の枝の長さの分布。\subref{fig:branch_length_absolute_edited}絶対度数表示。\subref{fig:branch_length_relativ_edited}相対度数表示。}
  \label{fig:branch_length_edited}
\end{figure}

図\ref{fig:branch_length_edited}は$\theta_{\mathrm{out}}$が$0.9\pi$ 以上になる枝の組み合わせを一本の枝としたものと,それ以外の組になっていない枝(図\ref{fig:branch_def_input}\subref{fig:branch_def}のピンクの枝と緑の枝)の長さの分布を解像度($\SI{1}{px}=11/681 \si{cm}\approx \SI{0.16}{mm}$)未満を除いて示している。図\ref{fig:branch_length_edited}\subref{fig:branch_length_absolute_edited}は絶対度数で表しており,図\ref{fig:branch_length_edited}\subref{fig:branch_length_relativ_edited}は相対度数で表している。図\ref{fig:branch_length}に比べて,高濃度における長い枝の割合が増加していた。また,極端に長いものも同様に低濃度側で見られた。これは図\ref{fig:branch_length}と同様に,一本だけ存在する長い枝の影響である。

\begin{figure}[htbp]
  \begin{minipage}
    {0.59\textwidth}
    \subcaption{}
    \centering
    \includegraphics[width=0.9\textwidth]{../../figure/part2(exp_deposition)/exp_b_branch_len.png}
    \label{fig:exp_b_branch_len}
  \end{minipage}
  \begin{minipage}
    {0.39\textwidth}
    \subcaption{}
    \centering
    \includegraphics[width=0.9\textwidth]{../../figure/part2(exp_deposition)/exp_b_branch_edited_len.png}
    \label{fig:exp_b_branch_edited_len}
  \end{minipage}
  \caption{枝の長さの冪分布の指数$b$の測定結果。\subref{fig:exp_b_branch_len}図\ref{fig:branch_def_input}\subref{fig:branch_def}のピンクの枝の指数$b$。\subref{fig:exp_b_branch_edited_len}図\ref{fig:branch_def_input}\subref{fig:branch_def}のピンクと緑の枝の指数$b$。}
  \label{fig:branch_length_exp}
\end{figure}

\begin{table}[htbp]
  \begin{minipage}{0.45\textwidth}
    \centering
    \caption{界面活性剤濃度による図\ref{fig:branch_def_input}\subref{fig:branch_def}のピンク色の枝の分布の指数$b$の平均値}
    \begin{tabular}{|c|c|}
      \hline
      濃度      & 指数$b$の平均値 \\ \hline\hline
      0.00\%  & $4.29$    \\ \hline
      0.005\% & $3.88$    \\ \hline
      0.10\%  & $4.41$    \\ \hline
      0.30\%  & $3.89$    \\ \hline
      0.50\%  & $3.76$    \\
      \hline
    \end{tabular}
    \label{tab:brnch_len_exp}
  \end{minipage}
  \hfill
  \begin{minipage}{0.45\textwidth}
    \centering
    \caption{界面活性剤濃度による図\ref{fig:branch_def_input}\subref{fig:branch_def}のピンクと緑色の枝の分布の指数$b$の平均値}
    \begin{tabular}{|c|c|}
      \hline
      濃度      & 指数$b$の平均値 \\ \hline\hline
      0.00\%  & $3.33$    \\ \hline
      0.005\% & $3.29$    \\ \hline
      0.10\%  & $3.46$    \\ \hline
      0.30\%  & $3.24$    \\ \hline
      0.50\%  & $2.92$    \\
      \hline
    \end{tabular}
    \label{tab:branch_len_exp_edited}
  \end{minipage}
\end{table}

図\ref{fig:branch_length_exp}は界面活性剤ごとの枝の長さの分布の指数$b$を示している。図\ref{fig:branch_length_exp}\subref{fig:exp_b_branch_len}は図\ref{fig:branch_def_input}\subref{fig:branch_def}で定義されるピンクの枝の分布の指数$b$であり,図\ref{fig:branch_length_exp}\subref{fig:exp_b_branch_edited_len}は図\ref{fig:branch_def_input}\subref{fig:branch_def}で定義されるピンクと緑の枝の分布の指数$b$である。表\ref{tab:brnch_len_exp}はピンクの枝の分布の指数$b$の平均値であり,表\ref{tab:branch_len_exp_edited}はピンクと緑の枝の分布の指数$b$の平均値である。

図\ref{fig:branch_length_exp}\subref{fig:exp_b_branch_len}より,ピンクの枝の指数$b$の値は,
\begin{itemize}
  \item $\SI{0}{\mathrm{vol}\%}$,$\SI{0.01}{\mathrm{vol}\%}$では相対的に大きな値($b>4.00$)
  \item $\SI{0.03}{\mathrm{vol}\%}$以上では相対的に小さな値($b<4.00$)
  \item $\SI{0.005}{\mathrm{vol}\%}$は傾向から外れている
\end{itemize}
となることがわかった。$\SI{0.005}{\mathrm{vol}\%}$が低い値を取っているのは,図\ref{fig:branch_length},\ref{fig:branch_length_edited}で言及した,低濃度側で見られる極端に長い枝の影響だと思われる。

図\ref{fig:branch_length_exp}\subref{fig:exp_b_branch_edited_len}より,ピンクと緑の枝の分布は,
\begin{itemize}
  \item 0.03\%以下では相対的に大きな値($b>3.20$)
  \item 0.05\%では相対的に小さな値($b<3.20$)
\end{itemize}
となることが分かった。$\SI{0.005}{\mathrm{vol}\%}$の値がピンクの枝だけの時に比べて相対的に大きい値となっているのは,$\SI{0.03}{\mathrm{vol}\%}$以上の濃度での緑の枝(一本の枝とみなしたもの)の割合が増えたためだと思われる。
\ifdraft{
  \bibliographystyle{../../Preamble/Physics.bst}
  \bibliography{../../Preamble/reference.bib}
}{}
\end{document}
% \documentclass[autodetect-engine,dvi=dvipdfmx,a4paper,ja=standard,oneside,openany,11pt,draft]{bxjsbook}
\documentclass[autodetect-engine,dvi=dvipdfmx,a4paper,ja=standard,oneside,openany,11pt,draft]{bxjsarticle}
\usepackage{../../Preamble/mypackage}

\begin{document}


\ifdraft{
  \bibliographystyle{../Preamble/Physics.bst}
  \bibliography{../Preamble/reference.bib}
}{}
\end{document}

% Part3 Simulation
\documentclass[autodetect-engine,dvi=dvipdfmx,a4paper,ja=standard,oneside,openany,11pt]{bxjsbook}
\usepackage{../../Preamble/mypackage}

\begin{document}
\chapter{数値計算}
\section{数値計算のモデル}
\subsection{RWの方向の決定方法}
\begin{figure}[htbp]
  \centering
  \includegraphics[width=0.5\textwidth]{../../figure/part3/RW_2dim.png}
  \caption{各方向へ移動する確率$p_i,q_i$と動かない確率$r_i$,電場$E_i$の模式図。}
  \label{fig:RW_2dim}
\end{figure}

金属樹のパターン形成における界面の影響や電場の影響を評価し,実験結果を再現するため,数値計算を行った。数値計算は二次元DLA(空間次元$d=2$)をベースに,電場によるドリフトと界面活性剤による界面への固着の影響を取り込んだモデルを作成した。\ref{sec:RW}節と\ref{sec:Langevin}節の内容をもとに,RWとLangevin方程式の対応を考える。\ref{sec:RW}節より,外力(電場)について,
\begin{equation}
  \bar{F}_i=\mu q E_i=(p_i-q_i)\frac{a}{\Delta t}
  \label{eq:force}
\end{equation}が成り立つので,$p_i-q_i=\mu q E_i\Delta t/a$となる。また,拡散係数$D=a^2/(2d\Delta t)$より,$a^2=2dD\Delta t$となる。次に,式\eqref{eq:Langevin_overdamped_average}を離散化する。格子間隔$a$,$\Delta t$を用いると,以下のように離散化できる。
\begin{equation}
  \begin{split}
    \bm{x} & =a\bm{X}, \qquad t=n\Delta t, \qquad (n\in\mathbb{N},\bm{X}\in\mathbb{Z}^d,a,\Delta t \in \mathbb{R}) \\
    \label{eq:discretization}
  \end{split}
\end{equation}
1ステップの時間発展を考えると,式\eqref{eq:discretization}で$n=1,X_i=\pm1$とすればよい。$\langle\cdot\rangle$を確率分布に関する平均とすれば,$\langle X_i\rangle=(+1)p_i+(-1)q_i=\mu q E_i\Delta t/a$,$\langle X_i^2\rangle=(+1)^2p_i+(-1)^2q_i=1/d$より以下が成り立つ。ただし,電場の平均操作については,1ステップの時間・空間スケールではほぼ一定であるとして,動く前での電場を用いている。平均は
\begin{equation}
  \begin{split}
    \langle\bm{x}\rangle & =a\langle\bm{X}\rangle                     \\
                         & =a\sum_{i=1}^{d}\bm{e}_i(p_i-q_i)          \\
                         & =\mu q \Delta t\sum_{i=1}^{d} \bm{e}_i E_i \\
                         & =\mu q \Delta t\bm{E}
  \end{split}
  \label{eq:discrete_average}
\end{equation}
となる。次に,分散について考える。分散は
\begin{equation}
  \begin{aligned}
    \langle(\bm{x}(t)-\langle\bm{x}(t)\rangle)^2\rangle
     & =a^2\langle(\bm{X}-\langle\bm{X}\rangle)^2\rangle                        \\
     & =a^2\{\langle\bm{X}^2\rangle-\langle\bm{X}\rangle^2\}                    \\
     & =a^2\sum_{i=1}^{d}\ab\{\frac{1}{d}-\frac{(\mu q \Delta t)^2}{a^2}E_i^2\} \\
     & =a^2\ab\{1-\frac{(\mu q \Delta t)^2}{a^2}\bm{E}^2\}                      \\
  \end{aligned}
  \label{eq:discrete_variance_3_2}
\end{equation}
式\eqref{eq:discrete_average}と式\eqref{eq:discrete_variance_3_2}は素朴にRWとLangevin方程式を対応させたものである。しかし,分散の方は電場の強さに依存しており,\ref{sec:Langevin}節で求めた,分散が外力に依存しないという結果と異なる。このままだと,場所によって分散が異なる,つまり場所によって温度が異なるという物理的に不可解な状況になってしまう。

そこで,RWにおいて図\ref{fig:RW_2dim}のように\textbf{その場にとどまる確率 $r$}を導入し,電場の影響を打ち消すように調整した。物理的には電場と運動が拮抗し,動けない場合を想定している。まず,確率の保存則より,$r+\sum_{i=1}^{d}(p_i+q_i) =1$が成り立つ。そのため,第$i$成分に対して$p_i+q_i=(1-r)/d$となる。以降,$\bm{X}$空間での平均・分散を,$\langle X_i\rangle=2\alpha E_i$,$\langle\bm{X}^2\rangle-\langle\bm{X}\rangle^2=2dC$と置く。ここで,$\alpha=\mu q\Delta t/(2a)$であり,電場に対する応答の大きさを表す量である。式\ref{eq:discrete_average_variance}より,分散は以下のように修正される。
\begin{equation}
  \begin{split}
    \langle(\bm{x}(t)-\langle\bm{x}(t)\rangle)^2\rangle & =a^2\ab\{(1-r)-\frac{(\mu q \Delta t)^2}{a^2}\bm{E}^2\} \\
                                                        & =a^2\ab\{(1-r)-4\alpha^2\bm{E}^2\}                      \\
                                                        & =a^2 2dC                                                \\
                                                        & =2dD\Delta t
  \end{split}
  \label{eq:discrete_variance}
\end{equation}
式\eqref{eq:discrete_variance}より,$C=D\Delta t/a^2$,$r=1-2dC-4\alpha^2\bm{E}^2$となる。

ここで注意しなければならないのは,ここまでの議論が成り立つためには,少なくとも$r>0$とならなければならない点である。$a^2=2dD\Delta t$のままであれば,$C=1/(2d)$より,$r=-4\alpha^2\bm{E}^2<0$となってしまう。そのため,格子間隔$a$を固定すれば,時間間隔$\Delta t$を変化させることで,$a^2\neq2dD\Delta t$になり,$C\neq1/(2d)$なので$r>0$を確保することができる。

以上の議論をまとめると,RWにおいて電場の影響を打ち消す確率$r$を導入することで,Langevin方程式の分散が外力に依存しないという結果を再現することができる。しかし,$r>0$を担保するため,時間間隔$\Delta t$を変化させなければならない。これはパラメータである$\alpha=\mu q\Delta t/(2a),C=D\Delta t/a^2$を変えることに他ならない。今回の数値計算では分散$C$を固定して,$\alpha$を変化させて形状変化を調べた。

$C$の値は以下のように定めた。まず,各$i$成分の確率は

\begin{equation}
  \left\{
  \begin{aligned}
    p_i+q_i & =2C+4\alpha^2E_i^2 \\
    p_i-q_i & =2\alpha E_i
  \end{aligned}
  \right.
  \label{eq:prob}
\end{equation}
を満たす。これは,式\eqref{eq:discrete_average_variance}の分散に関して,総和を取る前の関係式を用いると$\langle X_i^2\rangle-\langle X_i\rangle^2=2C$となることより,
\begin{equation}
  \begin{split}
    \langle X_i^2\rangle-\langle X_i\rangle^2 & =\left\{\frac{1-r}{d}-\frac{(\mu q \Delta t)^2}{a^2}E_i^2\right\} \\
                                              & =(p_i+q_i)-4\alpha^2E_i^2                                         \\
                                              & =2C
  \end{split}
  \label{eq:prob_middle}
\end{equation}
となることを用いた。これより,各確率$p_i,q_i,r$は,
\begin{equation}
  \left\{
  \begin{aligned}
    p_i & =C+\alpha E_i+2\alpha^2 E_i^2 \\
    q_i & =C-\alpha E_i+2\alpha^2 E_i^2 \\
    r   & =1-2dC-4\alpha^2\bm{E}^2
  \end{aligned}
  \right.
  \label{eq:prob3}
\end{equation}
で与えられる。

$p_i,q_i$を$\alpha E_i$について平方完成すると,$\alpha E_i=\mp1/4$で最小値$C-1/8$を取る。また,$r$は$\alpha E_i=0$で最大値$1-2dC$を取る。これら二つの値が0と1の間に入るという条件は$d=2$の時,
\begin{align}
  0<C-\frac{1}{8}<1 &  & \Leftrightarrow &  & \frac{1}{8}<C<\frac{9}{8} \\
  0<1-4C<1          &  & \Leftrightarrow &  & 0<C<\frac{1}{4}
  \label{eq:condition}
\end{align}
以上より,$C$の許される範囲は
\begin{equation}
  \frac{1}{8}<C<\frac{1}{4}
  \label{eq:condition2}
\end{equation}
となる。これより,上下端の中央の値である$C=3/16$を数値計算に用いる値とした。
\subsection{固着確率$P$の定義}
実験における,界面での界面活性剤によるイオンの析出阻害を再現するため,Brown運動してきた粒子を,パターンを構成する\textbf{クラスターに取り込む確率$P$}を導入した。本来のDLAでは,粒子がクラスターの隣に来た場合,そのまま新たなクラスターとして取り込むが,今回の数値計算では確率$P$でクラスターとして取り込んだ。取り込まれなかった場合は再びRWを行った。
\subsection{数値計算方法}
今回の数値計算では,パターンにより生じる電場と組み合わせてBrown運動する粒子の遷移確率を計算した。まず,計算する系の条件を表\ref{tab:condition}ように設定する。
\begin{table}[htbp]
  \centering
  \caption{数値計算の系の条件 }
  \begin{tabular}{|c||c|}
    \hline
    系の条件                      & 設定値                              \\ \hline\hline
    系のサイズ $L$                 & $\SI{512}{px}\times\SI{512}{px}$ \\ \hline
    粒子数 $N$                   & 15000                            \\ \hline
    中心座標 $\bm{x_c}=(x_c,y_c)$ & $(x_c,y_c)=(L/2,L/2)=(256,256)$  \\ \hline
    境界条件                      &
    \begin{tabular}{c}
      位置$\bm{x}=(x,y)$に対して,$||\bm{r}-\bm{x_c}||>L/2$で \\
      電位$V=1.0$,パターンが存在する点で$V=0$
    \end{tabular}
    \\ \hline
    SOR法による電場の収束条件            & 前回ループとの誤差$10^{-5}$未満             \\ \hline
    初期条件                      & $\bm{x_c}$に粒子を一つ置く               \\ \hline
    電場の計算タイミング                & 150粒子毎                           \\ \hline
  \end{tabular}
  \label{tab:condition}
\end{table}

具体的な計算方法は以下のとおりである。ただし,電場の計算においては,金属樹の電解質溶液の電気的中性が期待されるため,Laplace方程式$\nabla^2\phi=0$を解いた。
\begin{samepage}
  \begin{enumerate}
    \item 現在のパターンの内,中心からの距離が最も遠い点を探索する
    \item その点から30 px離れた円上にランダムに粒子を一つ置く
    \item 式\eqref{eq:prob3}に従って粒子の移動方向を決定する。電位が1.0の領域(中心からの距離が$L/2$の円よりも外側)に来たらその粒子を棄却し,新たな粒子を2.に従って置く。
    \item クラスターの隣の位置に来たら,与えた固着確率$P$でクラスターとして取り込むか判定する
    \item $150$粒子取り込むたびに,SOR法(詳しくは付録\ref{sec:SOR}参照)を用いて電位のLaplace方程式を解き,電場を計算した(図\ref{fig:DLA_ex})。
    \item 以上を15000粒子分繰り返す
  \end{enumerate}
\end{samepage}


この計算を,$\alpha=0.0$から$3.0$まで(0.0から1.0まで0.1刻み,1.0から3.0までは0.2刻み),$P=0.1,0.4,0.7,1.0$の4つの固着確率について,各条件について21回行った。

\begin{figure}[htbp]
  \begin{minipage}{0.32\hsize}
    \subcaption{}
    \centering
    \includegraphics[width=0.8\textwidth]{../../figure/part3/DLA_alpha=0_P=1.png}
    \label{fig:DLA_alpha_0_P_1}
  \end{minipage}
  \begin{minipage}{0.32\hsize}
    \subcaption{}
    \centering
    \includegraphics[width=0.8\textwidth]{../../figure/part3/DLA_phi_alpha=0_P=1.png}
    \label{fig:DLA_phi_alpha_0_P_1}
  \end{minipage}
  \begin{minipage}{0.32\hsize}
    \subcaption{}
    \centering
    \includegraphics[width=0.8\textwidth]{../../figure/part3/DLA_E^2_alpha=0_P=1.png}
    \label{fig:DLA_E_alpha_0_P_1}
  \end{minipage}
  \caption{$C=3/16$,パラメータ$\alpha=0.0$固着確率$P=1.0$(通常のDLA)の例。\subref{fig:DLA_alpha_0_P_1}DLAパターンの形状。\subref{fig:DLA_phi_alpha_0_P_1}電位分布。\subref{fig:DLA_E_alpha_0_P_1}電場の強度分布。}
  \label{fig:DLA_ex}
\end{figure}

\subsection{解析方法}
各条件につき21個のデータについて最終形状の回転半径$R_g$とフラクタル次元を計測した。フラクタル次元の計測には密度相関関数法を用いた。まず密度相関関数$C(r)$を計算し,粒子間距離$2\leq r\leq R_g$以内の範囲でフィッティングを行い,\ref{sec:density_correlation}節に従いフラクタル次元を計測した。
$\alpha=2.6$では計算が収束しなかったため,解析には$\alpha\leq2.4$までのデータを用いた。

原因については,計算が収束する場合としない場合があったことより,電場の影響が大きくなりすぎ,ランダムな運動よりも電場に従った弾道的な運動になったことで,たまたま中心付近を通るものしかクラスターに取り込まれなくなり,パターンの成長速度が極端に遅くなったためと考えられる。

\ifdraft{
  \bibliographystyle{../../Preamble/Physics.bst}
  \bibliography{../../Preamble/reference.bib}
}{}
\end{document}
\documentclass[autodetect-engine,dvi=dvipdfmx,a4paper,ja=standard,oneside,openany,11pt,draft]{bxjsbook}
\usepackage{../../Preamble/mypackage}

\begin{document}
\section{数値計算結果}
\subsection{パターンの変化}

\begin{figure}[htbp]
  \centering
  \includegraphics[width=0.8\textwidth]{../../figure/part3/sim_result.png}
  \caption{数値計算によるパターンの結果。典型的なものを示す。横軸は応答の大きさ$\alpha$,縦軸は固着確率$P$である。}
  \label{fig:sim_result}
\end{figure}

図\ref{fig:sim_result}は数値計算の結果を示している。横軸は応答の大きさ$\alpha$,縦軸は固着確率$P$である。$\alpha$が増加しても見た目には変化はわかりづらい。また,固着確率$P$が増加すると$\alpha$の値によらずパターンの枝が太くなっていくことがわかる。
\subsection{パターンの回転半径,フラクタル次元$D_f$の変化}

\begin{figure}[htbp]
  \centering
  \includegraphics[width=0.8\textwidth]{../../figure/part3/R_g_result.png}
  \caption{数値計算によるパターンの回転半径$R_g$の変化。横軸は応答の大きさ$\alpha$,縦軸は回転半径$R_g$である。}
  \label{fig:R_g_result}
\end{figure}

\begin{figure}
  \centering
  \includegraphics[width=0.6\textwidth]{../../figure/part3/fractal_dim_result.png}
  \caption{数値計算によるパターンのフラクタル次元$D_f$の変化。横軸は応答の大きさ$\alpha$,縦軸はフラクタル次元$D_f$である。}
  \label{fig:fractal_dim_result}
\end{figure}

図\ref{fig:R_g_result}は電場への応答$\alpha$と固着確率$P$を変化させたときの,回転半径の結果を示している。横軸は応答の大きさ$\alpha$,縦軸は回転半径$R_g$である。図\ref{fig:R_g_result}より,回転半径は$P>0.10$では$\alpha$の増加に対してほぼ単調に減少していることがわかる。逆に$P=0.10$ではほぼ一定値を取っている。また,固着確率が減少するほど,回転半径の大きさも減少し,減少割合は緩やかになる。特に$\alpha\lessapprox1.0$で顕著である。また,固着確率が線形に減少(0.3ずつ)しているにもかかわらず,回転半径は線形に減少しておらず,$P=0.10$と$P=0.40$の間に大きなとびがある。

図\ref{fig:fractal_dim_result}は電場への応答$\alpha$と固着確率$P$を変化させたときの,フラクタル次元の結果を示している。横軸は応答の大きさ$\alpha$,縦軸はフラクタル次元$D_f$である。フラクタル次元は,$P>0.10$では$\alpha$の増加に対してほぼ単調に増加しており,パターンが密になっていくことがわかる。$P=0.10$ではほぼ一定値を取っている。また,固着確率が減少するほど,フラクタル次元は2に近づき,増加割合は,回転半径の場合とは逆に急になっている。特に$\alpha\lessapprox1.0$で顕著である。また,固着確率が線形に減少(0.3ずつ)しているにもかかわらず,フラクタル次元$D_f$は線形に増加しておらず,回転半径と同様に$P=0.10$と$P=0.40$の間に大きなとびがある。
\ifdraft{
  \bibliographystyle{../../Preamble/Physics.bst}
  \bibliography{../../Preamble/reference.bib}
}{}
\end{document}
\documentclass[autodetect-engine,dvi=dvipdfmx,a4paper,ja=standard,oneside,openany,11pt,draft]{bxjsbook}
\usepackage{../../Preamble/mypackage}

\begin{document}
\section{議論・考察}
結果をまとめると,次のようになる。
\begin{itemize}
  \item パターンの見た目は電場への応答の大きさ$\alpha$を大きくしてもさほど変化は見られないが,固着確率$P$を増加させるとパターンの枝が太くなることが確認された。
  \item 回転半径について,$\alpha$を増加させるとほぼ単調減少し,固着確率の減少により回転半径も減少していた。また,変化の割合も緩やかになっていた。
  \item フラクタル次元については$\alpha$を増加させるとほぼ単調増加し,固着確率の減少によりフラクタル次元は2に近づいていた。
\end{itemize}

まず,回転半径の変化とフラクタル次元の変化が逆の傾向を示している。回転半径がパターンの平均的な大きさを示していることより,粒子数が一定ならば,回転半径の減少でパターンが密になることは自明であり,その結果としてフラクタル次元が増加することも自明である。そのため,回転半径とフラクタル次元は逆の傾向を示す。

また,電場への応答$\alpha$の増加により,粒子がより電場の強い点,つまり,より細かい構造が多い部分に集まりやすくなるため,細かい構造が優先的に粒子で埋められることで密なパターンになったと考えられる。クラスター形成の初期段階から細かい構造が埋められることで枝の広がりが抑制され,枝の遮蔽効果が弱くなり,内側まで粒子が入り込みやすくなった影響も考えられる。

今回の数値計算からは,$\alpha$の影響に比べて固着確率$P$の影響が顕著であり,見た目の変化がはっきりしていた点も含めて,パターン形成において固着確率の影響が大きいことが示唆された。また,固着確率の低下は拡散律速凝集から界面での反応が律速段階になる凝集過程である反応律速凝集(Reaction Limited Aggregation:RLA)への遷移を意味するが,$P=0.10$と$P>0.10$との間にとびがあることから,\Red{この遷移は急激な変化である可能性が示唆された。}

また,実験結果と異なり,フラクタル次元の低下は再現できなかった。電場への応答に比べて固着確率の影響が大きかったことより,固着確率の計算方法を工夫することが必要である。あるいは,粒子の数を増やして,より大きなパターンにすることで,太い枝を持つ,フラクタル次元の低い樹枝状パターンになる可能性もある。

今回の数値計算はRWを用いた離散的なモデルだが,添加物の拡散や粒子の拡散を考慮した連続体モデルに拡張することが考えられる。計算の複雑さや計算量が増えてしまうが,界面形状を連続体として考えることで,実験により近い結果が得られる可能性が高い。

\ifdraft{
  \bibliographystyle{../../Preamble/Physics.bst}
  \bibliography{../../Preamble/reference.bib}
}{}
\end{document}

% Part4 Conclusion to Appendix
% \documentclass[autodetect-engine,dvi=dvipdfmx,a4paper,ja=standard,oneside,openany,11pt,draft]{bxjsbook}
\documentclass[autodetect-engine,dvi=dvipdfmx,a4paper,ja=standard,oneside,openany,11pt,draft,textwidth=50zw]{bxjsbook}
\usepackage{../../Preamble/mypackage}

\begin{document}
\chapter{結論}
\ifdraft{
  \bibliographystyle{../../Preamble/Physics.bst}
  \bibliography{../../Preamble/reference.bib}
}{}
\end{document}
\documentclass[autodetect-engine,dvi=dvipdfmx,a4paper,ja=standard,oneside,openany,11pt,draft]{bxjsbook}
\usepackage{../../Preamble/mypackage}

\begin{document}
\chapter*{謝辞}
本研究を進めるうえで指導教員である北畑裕之先生と伊藤弘明先生には,研究の進め方,実験・解析方法,学会での発表方法,理科系の文章の書き方などの実務的なことから,物理学をする上での考え方や思考法,研究に対する心構えなど,研究者としての姿勢や哲学を学ぶことができました。

自分の知識不足や誤解など,至らない点が多く,考えていることをうまく言葉に出来ないことが多々ありましたが,その都度助け舟を出していただいて根気強く丁寧に議論していただきました。また,思い込みや固定観念,不勉強などで,研究や議論が進まないときなどは,適切なアドバイスで研究の方向性を示していただくとともに,認知の歪みの矯正や新たな知識の獲得に努めるように指導していただきました。締め切り直前になっても何度も原稿の添削をしていただいたことなども含めて,この紙面の余白にも書ききれないほどの感謝の念に堪えません。

また,研究室の先輩,同期,後輩の方々にも,自分の興味とは異なる分野にもかかわらず積極的に質問やアドバイスをいただきました。ゼミだけでなく普段の何気ない雑談や冗談を通して,研究のアイデアや知識,プログラミング方法のみならず,研究への思いや物理学に対する考え方など,多くのことを学びました。また,1を聞いたら10を教えてくれるような同期や後輩にも恵まれ,大変有意義な研究室生活を送ることができました。

学部生時代からの友人,特に電磁気や流体力学,熱力学の輪講をしている他研究室の友人や,修士課程まで進んでいる高校時代の友人など,互いに励ましあえる仲間の存在は大変心強い存在でした。

最後に,家族には,生活費などの経済的な支援のみならず,大学院への進学,特に博士課程への進学の応援など,精神的な支えをいただきました。学部入学から現在まで,家族の支えがなければここまで研究を続けることはできなかったと思います。

研究室に所属してからの3年間,多くの方と関わり,助けていただき,学びを得ることができました。この場を借りて感謝の意を表すとともに,本修士論文の締めとさせていただきます。


\ifdraft{
  \bibliographystyle{../../Preamble/Physics.bst}
  \bibliography{../../Preamble/reference.bib}
}{}
\end{document}
% \documentclass[autodetect-engine,dvi=dvipdfmx,a4paper,ja=standard,oneside,openany,11pt,draft]{bxjsbook}
\documentclass[autodetect-engine,dvi=dvipdfmx,a4paper,ja=standard,oneside,openany,11pt,draft]{bxjsarticle}
\usepackage{../../Preamble/mypackage}

\begin{document}


\ifdraft{
  \bibliographystyle{../Preamble/Physics.bst}
  \bibliography{../Preamble/reference.bib}
}{}
\end{document}

\bibliographystyle{../../Preamble/Physics.bst}
\bibliography{../../Preamble/reference.bib}

\end{document}