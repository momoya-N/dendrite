\documentclass[autodetect-engine,dvi=dvipdfmx,a4paper,ja=standard,oneside,openany,11pt]{bxjsbook}
\usepackage{../../Preamble/mypackage}

\begin{document}
\chapter{序論}
\section{パターン形成の物理学}
自然界では絶えずエネルギーの注入や散逸,物質の輸送が起こっている。このような系を平衡系と区別して\textbf{非平衡系}と呼ぶ。その中でも,非平衡過程により生じる様々な秩序構造の形成メカニズムや統計的性質に着目する分野を\textbf{パターン形成の物理学}と呼ぶ。パターンの形成メカニズムは様々なものがあり,図\ref{fig:pattern_formation}\subref{fig:BZ}のように連続した酸化還元反応によるもの,図\ref{fig:pattern_formation}\subref{fig:reaction_diffusion_angelfish}のように生体組織内の物質の拡散によるもの,図\ref{fig:pattern_formation}\subref{fig:Benard_cell}のように熱対流によるものなどが挙げられる。

\begin{figure}[htbp]
  \centering
  \begin{minipage}
    {0.32\textwidth}
    \subcaption{}
    \centering
    \includegraphics[width=0.9\textwidth]{../../figure/part1/BZ_reaction.png}
    \label{fig:BZ}
  \end{minipage}
  \begin{minipage}
    {0.32\textwidth}
    \subcaption{}
    \centering
    \includegraphics[width=0.9\textwidth]{../../figure/part1/reaction_diffusion_angelfish.png}
    \label{fig:reaction_diffusion_angelfish}
  \end{minipage}
  \begin{minipage}
    {0.32\textwidth}
    \subcaption{}
    \centering
    \includegraphics[width=0.9\textwidth]{../../figure/part1/Benard_cell.png}
    \label{fig:Benard_cell}
  \end{minipage}
  \caption{様々な非平衡系でのパターン形成。\subref{fig:BZ}Belousov-Zhabotinsky反応によるスパイラルパターン\cite{BZ_reaction}。円形容器の直径はおおよそ$\SI{50}{mm}$。\subref{fig:reaction_diffusion_angelfish}サザナミヤッコ(キンチャクダイの一種)の体表のTuringパターン。写真は横幅がおおよそ$\SI{50}{mm}$\cite{kondo1995reaction}。\subref{fig:Benard_cell}熱対流によるB{\'e}nard セル。円形容器の直径は$\SI{120}{mm}${\cite{eckert1998square}}。}
  \label{fig:pattern_formation}
\end{figure}

\section{枝分かれと樹枝状パターン}
非平衡系のパターンの中でも枝分かれをくりかえして形成される\textbf{樹枝状パターン}は生物組織(図\ref{fig:pattern_formation_dendrite}\subref{fig:blood_vessel_nerve})や電解析出(図\ref{fig:pattern_formation_dendrite}\subref{fig:electro_deposition}),落雷のような絶縁破壊(図\ref{fig:pattern_formation_dendrite}\subref{fig:thunder}),フィヨルド(図\ref{fig:pattern_formation_dendrite}\subref{fig:fjord})のように自然界で広くみられるパターンである。$\SI{e-4}{m}$から$\SI{e5}{m}$ほどの幅広いスケールで見られ,スケールも形成メカニズムも異なるが,普遍的に見られるパターンである。このようなパターンは\textbf{フラクタル(自己相似)構造}をもち,パターンの一部を相似拡大・縮小したものの統計的な性質が元のパターンと一致する性質を持っている。

\begin{figure}[htbp]
  \begin{tabular}{cc}
    \begin{minipage}[t]{0.45\textwidth}
      \subcaption{}
      \centering
      \includegraphics[keepaspectratio, scale=0.8]{../../figure/part1/blood_vessel_nerve.png}
      \label{fig:blood_vessel_nerve}
    \end{minipage} &
    \begin{minipage}[t]{0.45\textwidth}
      \subcaption{}
      \centering
      \includegraphics[keepaspectratio, scale=0.8]{../../figure/part1/electro_deposition.png}
      \label{fig:electro_deposition}
    \end{minipage} \\

    \begin{minipage}[t]{0.45\textwidth}
      \subcaption{}
      \centering
      \includegraphics[keepaspectratio, scale=0.8]{../../figure/part1/thunder.jpg}
      \label{fig:thunder}
    \end{minipage}            &
    \begin{minipage}[t]{0.45\textwidth}
      \subcaption{}
      \centering
      \includegraphics[keepaspectratio, scale=0.8]{../../figure/part1/fjord.jpg}
      \label{fig:fjord}
    \end{minipage}
  \end{tabular}
  \caption{自然界に見られる様々な樹枝状パターン。\subref{fig:blood_vessel_nerve}血管(左)と神経(右)。スケールバー:$\SI{100}{\mu m}=\SI{e-4}{m}$ \cite{mukouyama2002sensory}。\subref{fig:electro_deposition}電解析出による亜鉛の金属樹。中村撮影。パターンの直径はおおよそ$\SI{e-2}{m}$。\subref{fig:thunder}落雷(Wikipedia)。長さスケールはおおよそ$\SI{e3}{m}$。\subref{fig:fjord}フィヨルド(Wikipedia)。河口からの距離はおおよそ$\SI{e5}{m}$。}
  \label{fig:pattern_formation_dendrite}
\end{figure}

\section{フラクタル次元}
\label{sec:fractal_dimension}
\subsection{フラクタル次元の定義}
フラクタル構造を特徴づける量として\textbf{フラクタル次元}が挙げられる。フラクタル次元の定義は様々であるが,ここでは\textbf{相似次元}を用いる。相似次元は次のように定義される。
\begin{equation}
  D_f=-\odv{\log N(\varepsilon)}{\log \varepsilon}
  \label{eq:fractal_dimension_def}
\end{equation}

ここで$N(\varepsilon)$は与えられたパターンを$\varepsilon$のスケールで埋めつくすのに必要な個数(長さ$\varepsilon^{\mathrm{空間次元}}$の物差しで何個になるか)である。フラクタル次元は,図形の複雑さを表す指標であり,整数である場合はユークリッド空間における次元を表す。例えば,ユークリッド空間中(図\ref{fig:fractal_stracture}\subref{fig:相似次元の考え方})では,スケールは$\varepsilon=1/l$で与えられる。$\varepsilon=1/1,1/2,1/3$と減少していくとパターンの数$N(\varepsilon)$は増加する。式\eqref{eq:fractal_dimension_def}を用いて計算すると,空間次元$D=1,2,3$に一致する。図\ref{fig:fractal_stracture}\subref{fig:シェルピンスキーのギャスケット}はSierpinskiのギャスケットと呼ばれるパターンである。このパターンは三角形のスケール(三角形の一辺の長さ)を半分にして,中央の三角形を取り去ることで構成される。スケール$\varepsilon$が$\varepsilon=1/2$になると,一辺の長さ$\varepsilon$の三角形の個数$N(\varepsilon)$が3つになる。よって式\eqref{eq:fractal_dimension_def}より$D_f=\log 3/\log 2=1.5849\cdots$となる。フラクタル構造を持つ系はスケールフリー性を持ち,系の大きさ(スケール)によらず同じ構造(パターン配置)を持つ。このような\textbf{スケール不変性}を持つ系では,あるスケールで計測できる量(パターンの数,密度等)にスケールの長さに対する冪関数則が成り立つことが知られている。

\begin{figure}[htbp]
  \begin{minipage}{0.45\textwidth}
    \centering
    \subcaption{}
    \includegraphics[scale=0.5]{../../figure/part1/Fractaldimensionexample.png}
    \label{fig:相似次元の考え方}
  \end{minipage}
  \begin{minipage}{0.45\textwidth}
    \centering
    \subcaption{}
    \includegraphics[scale=0.5]{../../figure/part1/Sierpinski_triangle.png}
    \label{fig:シェルピンスキーのギャスケット}
  \end{minipage}
  \caption{相似次元の考え方。\subref{fig:相似次元の考え方}各スケール毎のパターンの個数の変化(Wikipedia)。\subref{fig:シェルピンスキーのギャスケット} Sierpinskiのギャスケット(Wikipedia)。相似次元(フラクタル次元)は$D_f=\log 3/\log 2=1.5849\cdots$。}
  \label{fig:fractal_stracture}
\end{figure}

\subsection{フラクタル次元の計測法}
フラクタル次元の計測法にはいくつかの方法があるが,この節では本研究の解析の理解に必要なものに絞って記述する。計測法については文献\cite{フラクタルの物理Ⅰ}を参考にした。
\subsubsection{ボックスカウンティング法}
\begin{figure}[htbp]
  \centering
  \includegraphics[width=0.6\textwidth]{../../figure/part2(exp_deposition)/boxcounting.png}
  \caption{ボックスカウンティング法の模式図\cite{表面粗さ曲線のフラクタル解析}。}
  \label{fig:box_counting}
\end{figure}

ボックスカウンティング法とは,例えば図\ref{fig:box_counting}のように2次元空間内で与えられたパターンのフラクタル次元を求める場合,パターンを様々なスケール(大きさ)の正方形(2次元の場合)で覆い,その正方形の中にパターンが含まれるか否かを数えることでフラクタル次元を求める方法である。

スケールを$\varepsilon$,パターンが少しでも含まれるボックスの個数を$N(\varepsilon)$とすると,フラクタル次元$D_f$とボックスの数の関係はおおよそ$N(\varepsilon)\propto\varepsilon^{D_f}$となる。ボックスの大きさ$\varepsilon$を変え,$N(\varepsilon)$を求め,両対数プロットすることでその傾きからフラクタル次元を求められる。ボックスの大きさを変えて個数を計測するだけのため,コンピュータによる実装が簡単な反面,他の計測法に比べて精度が低い。
\subsubsection{回転半径法}
図\ref{fig:DLA_with_R_g}のようにランダムな要素を含むフラクタルパターンの大きさの目安を与えるのが\textbf{回転半径}である。回転半径$R_g$は
\begin{align}
  R_g & =\sqrt{\frac{1}{N}\sum_{i=1}^{N}(\bm{r}_i-\bm{r}_{\mathrm{c}})^2}
  \label{eq:gyration_radius}
\end{align}
と与えられる。ただし,$\bm{r}_c$はパターンの重心で,
\begin{equation}
  \bm{r}_{\mathrm{c}}=\frac{1}{N}\sum_{i=1}^{N}\bm{r}_i
\end{equation}
である。

\begin{figure}[htbp]
  \centering
  \includegraphics[width=0.4\textwidth]{../../figure/part3/DLA_with_R_g.png}
  \caption{樹枝状パターンの回転半径。おおよそ$R_g=\SI{108}{px}$。$R_g$がおおよそのパターンの大きさを与える。}
  \label{fig:DLA_with_R_g}
\end{figure}

パターンに含まれる粒子数$N$はユークリッド空間の面積や体積と同様に,回転半径$R_g$との間に,
\begin{equation}
  N\propto R_g^{D_f}
  \label{eq:gyration_radius_fractal}
\end{equation}
のような関係が成り立つ。$D_f$はフラクタル次元である。パターンの成長過程で,一定ステップごとに粒子数$N$と回転半径$R_g$を計測し,$R_g$と$N$の関係を両対数プロットすることで,その傾きからフラクタル次元$D_f$を求めることができる。この手法を\textbf{回転半径法}といい,平均操作が入っているため精度よく$D_f$を求めることができる。

\subsubsection{密度相関関数法}
\label{sec:density_correlation}
密度相関関数法は,パターンの密度分布を用いて,フラクタル次元を求める方法である。パターンの密度分布を$\rho(\bm{r})$とする。密度分布$\rho(\bm{r})$は図\ref{fig:density_func_dif}のように,
\begin{equation}
  \rho(\bm{r})=
  \begin{cases}
    1 & (\mathrm{位置}\bm{r}\mathrm{のピクセルがパターン内}) \\
    0 & (\mathrm{位置}\bm{r}\mathrm{のピクセルがパターン外})
  \end{cases}
  \label{eq:density_distribution}
\end{equation}
となる量である。
\begin{figure}[htbp]
  \centering
  \includegraphics[width=0.4\textwidth]{../../figure/part3/densitiy_func_dim.png}
  \caption{パターンの密度分布。赤色がパターンに含まれるピクセル。赤色のピクセルでは$\rho(\bm{r})=1$, 灰色のピクセルでは$\rho(\bm{r})=0$となる。}
  \label{fig:density_func_dif}
\end{figure}

パターンの密度相関関数$C(\bm{r})$は
\begin{equation}
  C(r)=\frac{1}{\Omega_t}\int \d\Omega \frac{1}{N}\sum_{{\bm{r'}}}\rho(\bm{r'+ r})\rho({\bm{r'}})
  \label{eq:density_correlation}
\end{equation}
と定義される。$\Omega$ は$\bm{r}$のなす立体角,$\Omega_t$ は全立体角,$N$ は粒子数である。空間次元$d$に対して,
\begin{equation}
  \begin{aligned}
    \Omega   & =\prod^{d-1}_{i=1} (\sin\theta_i)^{d-i-1} \d{\theta_1}...\d{\theta_{d-1}} \qquad(0\leqq \theta_i\leqq \pi\,,\,0\leqq\theta_{d-1}<2\pi) \\
    \Omega_t & =2^{d-1}\pi
  \end{aligned}
  \label{eq:omega}
\end{equation}
で与えられる。興味があるのは$N$の$r$に対するスケーリング則のため,定数分は除いて考える。$C(r)$はスケールフリー性を持つので,$C(r)\propto r^{-a}$と表せる(スケールフリー性と冪関数については付録\ref{sec:scale_free}参照)。$C(r)$はある粒子から距離$r$離れた際に粒子が存在する確率を表しているので,微小範囲の粒子数は$\d N=C(r)\d \bm{r}\propto C(r)r^{d-1}\d r\propto r^{d-a-1} \d r $とスケーリングできる。全範囲にわたって積分すればそのパターンの大きさ,したがって粒子数$N$が導出できる。回転半径程度がパターンのおおよその大きさで,回転半径以上では$C(r)$はほぼ0とみなせるため,粒子数$N$は,
\begin{equation}
  N\propto \int_{0}^{R_g} r^{d-a-1} \d r\propto r^{d-a}
  \label{eq:density_correlation_fractal}
\end{equation}
と表される。

本節の議論より,密度相関関数法の両対数プロットより$a$の値を求められれば,パターンのフラクタル次元$D_f$を求めることができる。式\ref{eq:gyration_radius_fractal}と式\ref{eq:density_correlation_fractal}より,$D_f=d-a$となる。

% \section{研究目的}
% 枝分かれパターンの形成メカニズムは現象によって異なる。\textcolor{red}{しかし,枝の分岐(Tip Splitting)は成長過程での界面の粗さが影響を与える\cite{}ことが知られており,}界面の粗さの不安定性が最終的な枝分かれ形状に影響を与えると思われる。しかし,成長界面の粗さの不安定性と最終的な枝分かれ形状の関係は十分に理解されていない。本研究では,成長界面の粗さの不安定性と枝分かれ形状の関係を明らかにすることを目的とする。モデル系として,亜鉛イオン水溶液の電界析出で生じる亜鉛の金属樹を用いる。亜鉛の金属樹はフラクタル性を示すことが知られており\cite{matsushita1984fractal},様々なパラメータでの形態変化が調べられている\cite{suda2003temperature}。本研究では電解質溶液に界面活性剤を加え,最終的に生じる樹枝状パターンがどのように変化するかを議論するとともに,パターン変化の原因であると思われる,電界析出する金属結晶の界面の粗さを平滑化させるleveling現象を検証した。また,枝分かれパターンのフラクタル性を再現するモデルである拡散律速凝集(Diffusion Limited Aggregation: DLA)モデル\cite{witten1981diffusion}をベースに,実験を再現するモデルの構築を行った。

\ifdraft{
  \bibliographystyle{../../Preamble/Physics.bst}
  \bibliography{../../Preamble/reference.bib}
}{}

\end{document}