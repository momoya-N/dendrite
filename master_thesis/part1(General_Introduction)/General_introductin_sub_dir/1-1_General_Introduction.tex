\documentclass[autodetect-engine,dvi=dvipdfmx,a4paper,ja=standard,oneside,openany,11pt,draft]{bxjsbook}
\usepackage{../../Preamble/mypackage}

\begin{document}
\chapter{序論}
\section{パターン形成の物理学}
自然界では絶えずエネルギーの注入や散逸,物質の輸送などが起こっている。このような系を平衡系と区別して\textbf{非平衡系}と呼ぶ。その中でも,非平衡過程により生じる様々な秩序構造の形成メカニズムや統計的性質に着目する分野を\textbf{パターン形成の物理学}と呼ぶ。パターンの形成メカニズムは様々なものがあり,図\ref{fig:BZ}のように連続した酸化還元反応によるもの,図\ref{fig:reaction_diffusion_angelfish}のように生体組織内の物質の拡散によるもの,図\ref{fig:Be’nard_cell}のように熱対流によるものなどが挙げられる。
\begin{figure}[htbp]
  \centering
  \begin{minipage}
    {0.32\textwidth}
    \subcaption{}
    \centering
    \includegraphics[width=0.9\textwidth]{../../figure/part1/BZ_reaction.jpg}
    \label{fig:BZ}
  \end{minipage}
  \begin{minipage}
    {0.32\textwidth}
    \subcaption{}
    \centering
    \includegraphics[width=0.9\textwidth]{../../figure/part1/reaction_diffusion_angelfish.png}
    \label{fig:reaction_diffusion_angelfish}
  \end{minipage}
  \begin{minipage}
    {0.32\textwidth}
    \subcaption{}
    \centering
    \includegraphics[width=0.9\textwidth]{../../figure/part1/Be’nard_cell.png}
    \label{fig:Be’nard_cell}
  \end{minipage}
  \caption{様々な非平衡系でのパターン形成。\subref{fig:BZ}Belousov–Zhabotinsky反応によるスパイラルパターン\cite{BZ_reaction}。\subref{fig:reaction_diffusion_angelfish}サザナミヤッコ(キンチャクダイの一種)の体表のTuringパターン\cite{kondo1995reaction}。\subref{fig:Be’nard_cell}熱対流によるB{\'e}nard セル}
  \label{fig:pattern_formation}
\end{figure}

\section{枝分かれと樹枝状パターン}
非平衡系でのパターンの中でも枝分かれをくりかえし形成される\textbf{樹枝状パターン}は生物組織(図\ref{fig:pattern_formation_dendrite}\subref{fig:blood_vessel_nerve})や電解析出(図\ref{fig:pattern_formation_dendrite}\subref{fig:electro_deposition}),落雷のような絶縁破壊(図\ref{fig:pattern_formation_dendrite}\subref{fig:thunder}),フィヨルド(図\ref{fig:pattern_formation_dendrite}\subref{fig:fjord})のように自然界で広くみられるパターンである。$\mathcal{O}(\SI{e-4}{m})$から$\mathcal{O}(\SI{e5}{m})$という幅広いスケールで見られ,スケールも形成メカニズムも異なるが,普遍的に見られるパターンである。このようなパターンは\textbf{フラクタル(自己相似)構造}をもち,パターンの一部を相似拡大・縮小ものが元のパターンと一致するという性質を持っている。
\begin{figure}[htbp]
  \begin{tabular}{cc}
    \begin{minipage}[t]{0.45\textwidth}
      \subcaption{}
      \centering
      \includegraphics[keepaspectratio, scale=0.8]{../../figure/part1/blood_vessel_nerve.png}
      \label{fig:blood_vessel_nerve}
    \end{minipage} &
    \begin{minipage}[t]{0.45\textwidth}
      \subcaption{}
      \centering
      \includegraphics[keepaspectratio, scale=0.8]{../../figure/part1/electro_deposition.png}
      \label{fig:electro_deposition}
    \end{minipage} \\

    \begin{minipage}[t]{0.45\textwidth}
      \subcaption{}
      \centering
      \includegraphics[keepaspectratio, scale=0.8]{../../figure/part1/thunder.jpg}
      \label{fig:thunder}
    \end{minipage}            &
    \begin{minipage}[t]{0.45\textwidth}
      \subcaption{}
      \centering
      \includegraphics[keepaspectratio, scale=0.8]{../../figure/part1/fjord.jpg}
      \label{fig:fjord}
    \end{minipage}
  \end{tabular}
  \caption{自然界に見られる様々な樹枝状パターン。\subref{fig:blood_vessel_nerve}血管(左)と神経(右)。Scale Bar: $\SI{100}{\mu m}(\SI{e-4}{m})$\cite{mukouyama2002sensory}。\subref{fig:electro_deposition}電解析出による亜鉛の金属樹$\sim\mathcal{O}(\SI{e-2}{m})。$\subref{fig:thunder}落雷(Wikipedia)$\sim\mathcal{O}(\SI{e2}{m})$。\subref{fig:fjord}フィヨルド(Wikipedia)$\sim\mathcal{O}(\SI{e5}{m})$。}
  \label{fig:pattern_formation_dendrite}
\end{figure}
\section{フラクタル次元}
\label{sec:fractal_dimension}
フラクタル構造を特徴づける量として\textbf{フラクタル次元}が挙げられる。フラクタル次元の定義は様々であるが,ここでは\textbf{相似次元}を用いる。相似次元は次のように定義される。
\begin{equation}
  D_f=\lim_{\varepsilon \to 0}\frac{\log N(\varepsilon)}{\log \frac{1}{\varepsilon}}
  \label{eq:fractal_dimension_def}
\end{equation}

ここで$N(\varepsilon)$は$\varepsilon$で覆われる点の数(長さ$\varepsilon$の物差しで何個になるか)である。フラクタル次元は,図形の複雑さを表す指標であり,整数である場合はユークリッド空間における次元を表す。例えば,ユークリッド空間中では,図\ref{fig:fractal_stracture}\subref{fig:相似次元の考え方}のようにスケール(図中では$l$)の長さが$l=1,2,3$と増加すると,パターンの数$N$は増加する。これを式\ref{eq:fractal_dimension_def}で計算すると,空間次元$D=1,2,3$に一致する。図\ref{fig:fractal_stracture}\subref{fig:シェルピンスキーのギャスケット}はシェルピンスキーのギャスケットと呼ばれるパターンである。これは三角形のスケール(三角形の一辺の長さ)を半分にして,中央の三角形を取ることで構成される。スケール$\varepsilon$が$\varepsilon=1/2$になると,一辺の長さ$\varepsilon$の三角形の個数$N(\varepsilon)$が3つになる。よって式\ref{eq:fractal_dimension_def}より$D_f=\log 3/\log 2=1.5849\cdots$となる。フラクタル構造を持つ系はスケールフリー性を持ち,系の大きさ(スケール)によらず同じ構造(パターン配置など)を持つ。このような\textbf{スケール不変性}を持つ系では,あるスケールで計測できる量(パターンの数,密度等)にべき関数則が成り立つことが知られている。

\begin{figure}[htbp]
  \begin{minipage}{0.45\textwidth}
    \centering
    \subcaption{}
    \includegraphics[scale=0.5]{../../figure/part1/Fractaldimensionexample.png}
    \label{fig:相似次元の考え方}
  \end{minipage}
  \begin{minipage}{0.45\textwidth}
    \centering
    \subcaption{}
    \includegraphics[scale=0.5]{../../figure/part1/Sierpiński_triangle.png}
    \label{fig:シェルピンスキーのギャスケット}
  \end{minipage}
  \caption{相似次元の考え方。\subref{fig:相似次元の考え方}各スケール毎のパターンの個数の変化(Wikipedia)\subref{fig:シェルピンスキーのギャスケット}シェルピンスキーのギャスケット(Wikipedia)。相似次元(フラクタル次元)は$D_f=\log 3/\log 2=1.5849\ddots$。}
  \label{fig:fractal_stracture}
\end{figure}
\section{研究目的}
枝分かれパターンの形成メカニズムは現象によって異なる。\textcolor{red}{しかし,枝の分岐(Tip Splitting)は成長過程での界面の粗さが影響を与える\cite{}ことが知られており,}界面の粗さの不安定性が最終的な枝分かれ形状に影響を与えると思われる。しかし,成長界面の粗さの不安定性と最終的な枝分かれ形状の関係は十分に理解されていない。本研究では,成長界面の粗さの不安定性と枝分かれ形状の関係を明らかにすることを目的とする。モデル系として,亜鉛イオン水溶液の電界析出で生じる亜鉛の金属樹を用いる。亜鉛の金属樹はフラクタル性を示すことが知られており\cite{matsushita1984fractal},様々なパラメータでの形態変化が調べられている\cite{suda2003temperature}。本研究では電解質溶液に界面活性剤を加え,最終的に生じる樹枝状パターンがどのように変化するかを議論するとともに,パターン変化の原因であると思われる,電界析出する金属結晶の界面の粗さを平滑化させるleveling現象を検証した。また,枝分かれパターンのフラクタル性を再現するモデルである拡散律速凝集(Diffusion Limited Aggregation: DLA)モデル\cite{witten1981diffusion}をベースに,実験を再現するモデルの構築を行った。

\ifdraft{
  \bibliographystyle{../../Preamble/Physics.bst}
  \bibliography{../../Preamble/reference.bib}
}{}

\end{document}