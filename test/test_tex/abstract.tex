\documentclass[autodetect-engine,dvi=dvipdfmx,a4paper,ja=standard,oneside,openany,11pt,draft]{bxjsbook}
\usepackage{preamble}
\begin{document}
エネルギーの注入や散逸を伴う非平衡系においては,熱平衡系には対応物のない,新しいクラスの相転移現象が数多く存在することが知られている。
Landau理論に基づいた現象論的な自由エネルギー汎関数を導入し,動的な自由エネルギー最小化原理である勾配系ダイナミクスを考えることで,非平衡系における相転移現象を説明できることがある。
しかし,捕食・被食系のように,取りたい状態が互いに異なる,非相反な相互作用を伴う系においては,自由エネルギー最小化原理を適用することができない。
非相反相互作用系においては,通常のLandau理論では現れない,静的な相から動的な相への非相反相転移現象を示すことが知られており,非平衡統計力学の分野で近年注目を集めている。
自由エネルギー最小化原理を記述する勾配系ダイナミクスを非相反相互作用系に拡張した,擬勾配系ダイナミクスは,非相反相互作用する連続場の時空間パターン形成を記述し,非相反相転移現象と多様な分岐構造を示す数理モデルとして,パターン形成やアクティブマター物理学などの観点から近年研究が盛んに行われてきている。

本研究では,非保存系の擬勾配系ダイナミクスのひとつである,1 次元非相反Swift-Hohenbergモデルについて,数値計算と,振幅方程式を用いた理論的な解析を行い,非相反相転移に伴う時空間ダイナミクスの分岐構造に着目して解析を行なった。
非相反Swift-Hohenbergモデルは,2 つの秩序変数が相反・非相反な相互作用をする偏微分方程式系である。
このモデルには,不安定化を特徴づけるパラメータ,相反相互作用・非相反相互作用を特徴づけるパラメータの計3つのパラメータが存在する。
これらのパラメータを変えながら行なった数値計算により,秩序変数の時空間Fourierスペクトルに着目して相を特徴づけ,相図を作成した。
また,空間 Fourier モードに関する振幅方程式を用いて,時空間パターンの間の分岐構造を同定することにも成功している。

本研究は,所属研究室の北畑裕之教授,伊藤弘明助教と,国科温州研究院(中華人民共和国)の好村滋行教授との共同研究によって行われたものである。

\end{document}
