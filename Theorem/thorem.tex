\documentclass{ltjsarticle}
%\usepackage[top=30truemm,bottom=30truemm,left=20truemm,right=20truemm]{geometry}
\usepackage{mypackage}
\begin{document}

\title{計算メモ}
\author{中村友哉}
\maketitle
\section{Bejie Curv}
\subsection{The Definition of Bezier Curve}
Def.

\begin{equation}
  \vb*{P}(t)=\sum_{k=0}^{N-1}\binom{n}{k}t^k(1-t)^{n-k}\vb*{Q}_i
\end{equation}

ここで$\vb*{Q}_i$は制御点を表す。$\vb*{q}_i=(a_i,b_i)^t$ (縦ベクトル)と置くと、ベジェ曲線のパラメータ表示は制御点が4つの場合以下のようになる。

\begin{equation}
  \vb*{x}(t)=
  \begin{pmatrix}
    x \\ y  
  \end{pmatrix}
  =(1-t)^3\vb*{q}_1+3(1-t)^2t\vb*{q}_2+3(1-t)t^2\vb*{q}_3+t^3\vb*{q}_4
\end{equation}

\subsection{Calculating Area}
Bezier Curve 以下の面積は
\begin{equation}
  dx=\{-3(1-t)^2a_1-3(1-t)(3t-1)a_2+3t(2-3t)a_3+3t^2a_4\}dt
\end{equation}
より、以下で与えられる。
\begin{equation}
  \begin{split}
    S&=\int_{a_1}^{a_4}y(t)\dd{x(t)}\\
    &=\int_{0}^{1}\{(1-t)^3b_1+3(1-t)^2tb_2+3(1-t)t^2b_3+t^3b_4\}\\
    & \hspace{5cm}\cdot \{-3(1-t)^2a_1-3(1-t)(3t-1)a_2+3t(2-3t)a_3+3t^2a_4\}dt\\
    &=\frac{1}{20}\{b_1(-10a_1+6a_2+3a_3+a_4)+3b_2(-2a_1+a_3+a_4)\\
    & \hspace{5cm}-3b_3(a_1+a_2-2a_4)-b_4(a_1+3a_2+6a_3-10a_4)\}\\
  \end{split}
\end{equation}

\end{document}