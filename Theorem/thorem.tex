\documentclass{ltjsarticle}
\usepackage[top=30truemm,bottom=30truemm,left=20truemm,right=20truemm]{geometry}
\usepackage{physics}
\begin{document}

\title{はじめての\TeX }
\author{Taro J. Armstrong}
\maketitle
\section{はじめての\TeX がLua\TeX なんて粋だね}
%\section{はじめての\TeXがLua\TeXなんて粋だね}

%こうやって文字を打ちます。
\subsection{小見出し!}
あたり前過ぎて気に止めることもないですが、きとんと改行命令を出していなくても自動で改行します。
\\出力できてる?ちゃんと出力できてそう。
\begin{equation}
  \frac{\partial C}{\partial t}=D\frac{\partial^2 c}{\partial x^2}
\end{equation}
\begin{equation}
  F(\omega)=\frac{1}{2\pi}\int f(t) e^{-i\omega t}
\end{equation}
\begin{equation}
  E=\sqrt{m^2c^4+p^2v^2}
\end{equation}
\end{document}